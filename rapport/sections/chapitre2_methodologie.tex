\chapter{Methodologie et Approche Technique}

\section{Methodologie de Developpement}

\subsection{Cycle de Vie du Projet}

Le developpement de notre solution SIEM/SOAR suit une approche iterative basee sur la methodologie DevSecOps, adaptee aux contraintes de securite et de disponibilite de l'environnement hospitalier.

\subsubsection{Phases de Developpement}

\begin{enumerate}
    \item \textbf{Phase d'Analyse} (2 semaines)
          \begin{itemize}
              \item Audit de l'infrastructure existante
              \item Identification des sources de logs
              \item Analyse des flux reseau critiques
              \item Mapping des exigences reglementaires
          \end{itemize}

    \item \textbf{Phase de Conception} (3 semaines)
          \begin{itemize}
              \item Architecture de la solution SIEM/SOAR
              \item Definition des cas d'usage prioritaires
              \item Conception des workflows d'automatisation
              \item Specification des integrations API
          \end{itemize}

    \item \textbf{Phase d'Implementation} (6 semaines)
          \begin{itemize}
              \item Deploiement de l'infrastructure de base
              \item Configuration des connecteurs de donnees
              \item Developpement des regles de correlation
              \item Integration des composants SOAR
          \end{itemize}

    \item \textbf{Phase de Tests} (3 semaines)
          \begin{itemize}
              \item Tests de charge et performance
              \item Validation des scenarios d'attaque
              \item Tests d'integration bout en bout
              \item Audit de securite externe
          \end{itemize}

    \item \textbf{Phase de Deploiement} (2 semaines)
          \begin{itemize}
              \item Migration progressive en production
              \item Formation des equipes operationnelles
              \item Documentation technique complete
              \item Monitoring post-deploiement
          \end{itemize}
\end{enumerate}

\subsection{Methodologie de Securite}

\subsubsection{Security by Design}

L'approche "Security by Design" est integree des la conception :

\begin{itemize}
    \item \textbf{Principe de moindre privilege} : Acces minimal necessaire pour chaque composant
    \item \textbf{Defense en profondeur} : Multiples couches de securite redondantes
    \item \textbf{Fail-safe defaults} : Configuration securisee par defaut
    \item \textbf{Separation des preoccupations} : Isolation des environnements critiques
\end{itemize}

\subsubsection{Framework NIST Cybersecurity}

Notre approche s'aligne sur le framework NIST CSF :

\begin{table}[H]
    \centering
    \caption{Mapping NIST Cybersecurity Framework}
    \begin{tabular}{|l|l|l|}
        \hline
        \textbf{Fonction} & \textbf{Composant SIEM/SOAR} & \textbf{Implementation} \\
        \hline
        Identify          & Asset Discovery              & Wazuh Agent Inventory   \\
        \hline
        Protect           & Access Control               & RBAC + MFA              \\
        \hline
        Detect            & Event Correlation            & Wazuh Rules Engine      \\
        \hline
        Respond           & Incident Response            & TheHive Workflows       \\
        \hline
        Recover           & Business Continuity          & Automated Backup        \\
        \hline
    \end{tabular}
\end{table}

\section{Architecture Technique Detaillee}

\subsection{Architecture Globale du Systeme}

\subsubsection{Vue d'Ensemble}

L'architecture de notre solution SIEM/SOAR s'articule autour de quatre couches principales, chacune ayant des responsabilites specifiques et des interfaces bien definies.

\begin{figure}[H]
    \centering
    \includegraphics[width=0.9\textwidth]{images/network_security_flow.png}
    \caption{Architecture globale de la solution SIEM/SOAR hospitaliere - Flux de securite}
    \label{fig:architecture_globale}
\end{figure}

La figure \ref{fig:architecture_globale} illustre les flux de donnees et les interactions entre les differents composants de notre solution. Cette architecture garantit une collecte exhaustive des evenements de securite et leur traitement en temps reel.

\subsection{Diagrammes de Flux de Donnees}

\subsubsection{Flux de Donnees Simplifie}

Pour une comprehension initiale, la figure \ref{fig:flow_simple} presente une vue simplifiee des flux de donnees principaux :

\begin{figure}[H]
    \centering
    \includegraphics[width=0.8\textwidth]{images/flowData_simple.png}
    \caption{Diagramme de flux de donnees simplifie}
    \label{fig:flow_simple}
\end{figure}

\subsubsection{Flux de Donnees Complexe}

La figure \ref{fig:flow_complex} detaille l'ensemble des interactions et des traitements automatises :

\begin{figure}[H]
    \centering
    \includegraphics[width=0.95\textwidth]{images/flowData_Complex.png}
    \caption{Diagramme de flux de donnees complexe - Vue detaillee des processus d'automatisation}
    \label{fig:flow_complex}
\end{figure}

Ce diagramme complexe montre comment les alertes du SIEM sont automatiquement traitees par le SOAR, enrichies avec de l'intelligence sur les menaces, et transformees en actions de reponse automatisees.

\subsubsection{Couche de Collecte de Donnees}

\paragraph{Sources de Donnees}
\begin{enumerate}
    \item \textbf{Logs Systeme}
          \begin{itemize}
              \item Serveurs Windows (Event Logs)
              \item Serveurs Linux (Syslog, journald)
              \item Equipements reseau (SNMP, NetFlow)
              \item Bases de donnees (Audit logs)
          \end{itemize}

    \item \textbf{Logs Applicatifs}
          \begin{itemize}
              \item SIH (Systeme d'Information Hospitalier)
              \item PACS (Picture Archiving and Communication System)
              \item Applications web medicales
              \item Systemes de messagerie
          \end{itemize}

    \item \textbf{Logs de Securite}
          \begin{itemize}
              \item Firewalls (pfSense, FortiGate)
              \item IDS/IPS (Suricata, Snort)
              \item WAF (ModSecurity)
              \item Systemes d'authentification (LDAP, SSO)
          \end{itemize}

    \item \textbf{Donnees de Contexte}
          \begin{itemize}
              \item Threat Intelligence (MISP feeds)
              \item Vulnerabilites (NIST NVD)
              \item Asset inventory (CMDB)
              \item Configuration management
          \end{itemize}
\end{enumerate}

\paragraph{Mecanismes de Collecte}
\begin{itemize}
    \item \textbf{Wazuh Agents} : Deploiement sur endpoints Windows/Linux
    \item \textbf{Syslog forwarding} : Collecte centralisee des logs reseau
    \item \textbf{API REST} : Integration avec applications tierces
    \item \textbf{File monitoring} : Surveillance de fichiers de logs
    \item \textbf{Windows Event Logs} : Collecte native via WinRM
\end{itemize}

\subsection{Couche de Traitement et Correlation}

\subsubsection{Wazuh SIEM - Moteur de Correlation}

\paragraph{Architecture Distribuee}
\begin{itemize}
    \item \textbf{Wazuh Manager} : Serveur central de correlation (Master)
    \item \textbf{Wazuh Workers} : Serveurs de traitement distribue
    \item \textbf{Wazuh Indexer} : Cluster Elasticsearch pour stockage
    \item \textbf{Wazuh Dashboard} : Interface de visualisation Kibana
\end{itemize}

\paragraph{Regles de Correlation Personnalisees}

Les regles de correlation sont developpees pour detecter les attaques specifiques a l'environnement hospitalier :

\begin{lstlisting}[style=xmlstyle,caption=Exemple de regle Wazuh pour detection EternalBlue]
<group name="eternalblue,windows,exploit">
  <!-- EternalBlue SMB exploit detection -->
  <rule id="100001" level="12">
    <if_sid>18152</if_sid>
    <srcip>!$HOME_NET</srcip>
    <dstport>445</dstport>
    <match>SMB|CIFS</match>
    <description>EternalBlue: SMB exploit attempt from external IP</description>
    <group>attack.lateral_movement,attack.t1055</group>
  </rule>
  
  <!-- Process injection after SMB connection -->
  <rule id="100002" level="13">
    <if_matched_sid>100001</if_matched_sid>
    <same_source_ip />
    <time>same_minute</time>
    <description>EternalBlue: Process injection detected</description>
    <group>attack.privilege_escalation,attack.t1055</group>
  </rule>
</group>
\end{lstlisting}

\subsubsection{Enrichissement des Evenements}

\paragraph{Geolocalisation IP}
\begin{itemize}
    \item Base GeoIP MaxMind pour localisation geographique
    \item Detection d'acces depuis pays a risque
    \item Calcul de distance impossible (Impossible Travel)
    \item Correlation avec listes de reputation IP
\end{itemize}

\paragraph{Asset Context}
\begin{itemize}
    \item Enrichissement avec donnees CMDB
    \item Classification de criticite des assets
    \item Mapping avec utilisateurs et services
    \item Contexte business des systemes impactes
\end{itemize}

\subsection{Couche d'Orchestration SOAR}

\subsubsection{TheHive - Gestion d'Incidents}

\paragraph{Modele de Donnees}
\begin{itemize}
    \item \textbf{Alerts} : Evenements de securite bruts depuis le SIEM
    \item \textbf{Cases} : Incidents de securite confirmes necessitant investigation
    \item \textbf{Tasks} : Actions specifiques dans le cadre d'un incident
    \item \textbf{Observables} : IOCs extraits et analyses (IP, hash, domaine)
\end{itemize}

\paragraph{Workflows Automatises}
\begin{enumerate}
    \item \textbf{Triage Automatique}
          \begin{itemize}
              \item Classification par type d'attaque (MITRE ATT\&CK)
              \item Scoring de criticite base sur asset et TTP
              \item Assignment automatique selon expertise equipe
              \item Escalade basee sur SLA predefinis
          \end{itemize}

    \item \textbf{Enrichissement Contextuel}
          \begin{itemize}
              \item Recherche historique d'incidents similaires
              \item Correlation avec threat intelligence MISP
              \item Analyse comportementale utilisateur (UEBA)
              \item Evaluation d'impact business
          \end{itemize}

    \item \textbf{Reponse Automatisee}
          \begin{itemize}
              \item Isolation reseau d'endpoints compromis
              \item Blocage automatique d'IP malveillantes
              \item Revocation de sessions utilisateur
              \item Sauvegarde forensique de preuves
          \end{itemize}
\end{enumerate}

\subsubsection{Cortex - Analyse d'Observables}

\paragraph{Analyzers Deployes}
\begin{table}[H]
    \centering
    \caption{Analyzers Cortex configures pour l'environnement hospitalier}
    \begin{tabular}{|l|l|c|l|}
        \hline
        \textbf{Type} & \textbf{Analyzer} & \textbf{SLA} & \textbf{Cas d'Usage}       \\
        \hline
        IP            & VirusTotal        & 30s          & Reputation IP externe      \\
        \hline
        IP            & AbuseIPDB         & 15s          & Detection IP malveillantes \\
        \hline
        Hash          & Hybrid Analysis   & 2min         & Analyse malware sandbox    \\
        \hline
        Domain        & PassiveTotal      & 20s          & Infrastructure adversaire  \\
        \hline
        URL           & Joe Sandbox       & 5min         & Analyse comportementale    \\
        \hline
        Email         & DMARC Analyzer    & 10s          & Validation authenticity    \\
        \hline
    \end{tabular}
\end{table}

\paragraph{Responders Personnalises}
\begin{itemize}
    \item \textbf{OPNsense IP Block} : Blocage automatique au niveau firewall
    \item \textbf{Active Directory Disable} : Desactivation compte utilisateur
    \item \textbf{Email Quarantine} : Mise en quarantaine email malveillant
    \item \textbf{MISP Event Creation} : Publication IOC vers communaute
\end{itemize}

\subsection{Couche d'Integration et Automatisation}

\subsubsection{n8n - Orchestrateur de Workflows}

\paragraph{Architecture n8n}
\begin{itemize}
    \item \textbf{Execution Mode} : Queue-based avec Redis backend
    \item \textbf{Scaling} : Horizontal scaling avec load balancer
    \item \textbf{Persistence} : PostgreSQL pour etat des workflows
    \item \textbf{Security} : JWT authentication avec rotation automatique
\end{itemize}

\paragraph{Workflows Critiques Implementes}

\begin{enumerate}
    \item \textbf{Workflow EternalBlue Response}
          \begin{itemize}
              \item Trigger : Wazuh alert rule 100001
              \item Actions : Isolation reseau + analyse forensique + notification
              \item SLA : Reponse en < 60 secondes
              \item Escalade : SOC Manager si echec automatisation
          \end{itemize}

    \item \textbf{Workflow XSS Detection}
          \begin{itemize}
              \item Trigger : ModSecurity WAF block
              \item Actions : Analyse payload + bloc IP + notification developpeur
              \item SLA : Traitement en < 30 secondes
              \item Learning : Machine learning pour amelioration detection
          \end{itemize}

    \item \textbf{Workflow Malicious Website}
          \begin{itemize}
              \item Trigger : DNS sinkhole hit
              \item Actions : Investigation utilisateur + formation + rapport
              \item SLA : Investigation en < 24h
              \item Prevention : Mise a jour blacklist DNS
          \end{itemize}
\end{enumerate}

\section{Technologies et Outils Selectionnes}

\subsection{Justification des Choix Techniques}

\subsubsection{Wazuh vs Alternatives}

\begin{table}[H]
    \centering
    \caption{Comparaison des solutions SIEM open source}
    \begin{tabular}{|l|c|c|c|c|}
        \hline
        \textbf{Critere} & \textbf{Wazuh} & \textbf{OSSIM} & \textbf{ELK} & \textbf{Graylog} \\
        \hline
        Events/sec       & 100K+          & 50K            & 200K+        & 75K              \\
        \hline
        Regles natives   & 3000+          & 1500+          & Custom       & 500+             \\
        \hline
        MITRE ATT\&CK    & Natif          & Plugin         & Manual       & Plugin           \\
        \hline
        Agent-based      & Oui            & Oui            & Beats        & Sidecar          \\
        \hline
        File Integrity   & Natif          & Plugin         & Manual       & Plugin           \\
        \hline
        Cloud Ready      & Oui            & Partiel        & Oui          & Oui              \\
        \hline
        \textbf{Score}   & \textbf{9/10}  & 6/10           & 8/10         & 7/10             \\
        \hline
    \end{tabular}
\end{table}

\paragraph{Avantages de Wazuh}
\begin{itemize}
    \item \textbf{Integration native} : MITRE ATT\&CK mapping built-in
    \item \textbf{Performance} : Traitement en temps reel haute performance
    \item \textbf{Compliance} : Modules PCI DSS, HIPAA, SOX natives
    \item \textbf{Scalabilite} : Architecture distribuee avec clustering
    \item \textbf{Communaute} : Support actif et regles regulierement mises a jour
\end{itemize}

\subsubsection{TheHive/Cortex vs Alternatives}

\begin{table}[H]
    \centering
    \caption{Comparaison des plateformes SOAR}
    \begin{tabular}{|l|c|c|c|c|}
        \hline
        \textbf{Critere} & \textbf{TheHive} & \textbf{MISP} & \textbf{Demisto} & \textbf{Phantom} \\
        \hline
        Open Source      & Oui              & Oui           & Non              & Non              \\
        \hline
        API REST         & Complete         & Complete      & Limitee          & Proprietaire     \\
        \hline
        Workflow Engine  & Natif            & Basique       & Avance           & Avance           \\
        \hline
        Threat Intel     & Via MISP         & Natif         & Integre          & Integre          \\
        \hline
        Cost (5 ans)     & 0€               & 0€            & 500K€            & 750K€            \\
        \hline
        Customization    & Elevee           & Elevee        & Moyenne          & Faible           \\
        \hline
        \textbf{Score}   & \textbf{9/10}    & 7/10          & 8/10             & 7/10             \\
        \hline
    \end{tabular}
\end{table}

\subsection{Infrastructure Technique}

\subsubsection{Specifications Materielles}

\begin{table}[H]
    \centering
    \caption{Dimensionnement infrastructure SIEM/SOAR}
    \begin{tabular}{|l|c|c|c|c|}
        \hline
        \textbf{Composant} & \textbf{CPU}     & \textbf{RAM}   & \textbf{Storage} & \textbf{Network} \\
        \hline
        Wazuh Manager      & 8 vCPU           & 16 GB          & 500 GB SSD       & 10 Gbps          \\
        \hline
        Wazuh Indexer (x3) & 4 vCPU           & 32 GB          & 2 TB NVMe        & 10 Gbps          \\
        \hline
        TheHive            & 4 vCPU           & 8 GB           & 200 GB SSD       & 1 Gbps           \\
        \hline
        Cortex             & 8 vCPU           & 16 GB          & 500 GB SSD       & 1 Gbps           \\
        \hline
        MISP               & 2 vCPU           & 4 GB           & 100 GB SSD       & 1 Gbps           \\
        \hline
        n8n                & 2 vCPU           & 4 GB           & 50 GB SSD        & 1 Gbps           \\
        \hline
        \textbf{Total}     & \textbf{30 vCPU} & \textbf{84 GB} & \textbf{3.85 TB} & \textbf{-}       \\
        \hline
    \end{tabular}
\end{table}

\subsubsection{Architecture Reseau}

\paragraph{Segmentation Reseau}
\begin{itemize}
    \item \textbf{DMZ SIEM} : 192.168.100.0/24 - Composants exposes (Dashboard)
    \item \textbf{LAN SOAR} : 192.168.101.0/24 - Backend processing (Indexer, Cortex)
    \item \textbf{MGMT} : 192.168.102.0/24 - Administration et monitoring
    \item \textbf{HOSPITAL} : 192.168.15.0/24 - Reseau hospitalier source
\end{itemize}

\paragraph{Flux Reseau Autorises}
\begin{enumerate}
    \item HOSPITAL → DMZ SIEM : Syslog (514/UDP), Wazuh Agent (1514/TCP)
    \item DMZ SIEM → LAN SOAR : Elasticsearch (9200/TCP), TheHive API (9000/TCP)
    \item LAN SOAR → Internet : Threat Intel feeds (443/TCP), DNS (53/UDP)
    \item MGMT → All : SSH (22/TCP), SNMP (161/UDP), HTTPS (443/TCP)
\end{enumerate}

Cette approche methodologique et technique etablit les fondements solides pour l'implementation de notre solution SIEM/SOAR, en garantissant la robustesse, la scalabilite et la securite adaptees a l'environnement hospitalier critique.
