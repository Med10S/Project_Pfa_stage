\appendix

\chapter{Configurations Techniques}

\section{Configuration Suricata}

\subsection{Fichier Principal suricata.yaml}

\begin{lstlisting}[style=yamlstyle,caption=Configuration Suricata pour environnement hospitalier]
%YAML 1.1
---

vars:
  address-groups:
    HOME_NET: "[192.168.15.0/24,192.168.181.0/24,192.168.183.0/24]"
    EXTERNAL_NET: "!$HOME_NET"
    HTTP_SERVERS: "$HOME_NET"
    SMTP_SERVERS: "$HOME_NET"
    SQL_SERVERS: "$HOME_NET"
    DNS_SERVERS: "$HOME_NET"
    TELNET_SERVERS: "$HOME_NET"
    AIM_SERVERS: "$EXTERNAL_NET"
    DC_SERVERS: "$HOME_NET"
    DNP3_SERVER: "$HOME_NET"
    DNP3_CLIENT: "$HOME_NET"
    MODBUS_CLIENT: "$HOME_NET"
    MODBUS_SERVER: "$HOME_NET"
    ENIP_CLIENT: "$HOME_NET"
    ENIP_SERVER: "$HOME_NET"

  port-groups:
    HTTP_PORTS: "80"
    SHELLCODE_PORTS: "!80"
    ORACLE_PORTS: 1521
    SSH_PORTS: 22
    DNP3_PORTS: 20000
    MODBUS_PORTS: 502
    FILE_DATA_PORTS: "[$HTTP_PORTS,110,143]"
    FTP_PORTS: 21
    GENEVE_PORTS: 6081
    VXLAN_PORTS: 4789
    TEREDO_PORTS: 3544

runmode: autofp

max-pending-packets: 1024

outputs:
  - fast:
      enabled: yes
      filename: fast.log
      append: yes
  - eve-log:
      enabled: yes
      filetype: regular
      filename: eve.json
      types:
        - alert:
            payload: yes
            payload-printable: yes
            packet: yes
            metadata: yes
            http-body: yes
            http-body-printable: yes
            tagged-packets: yes
        - http:
            extended: yes
        - dns
        - tls:
            extended: yes
        - files:
            force-magic: no
        - smtp
        - ssh
        - stats:
            totals: yes
            threads: no
            deltas: no
        - flow

app-layer:
  protocols:
    http:
      enabled: yes
      memcap: 64mb
    tls:
      enabled: yes
      detection-ports:
        dp: 443
    ftp:
      enabled: yes
    smtp:
      enabled: yes
    ssh:
      enabled: yes
    dns:
      tcp:
        enabled: yes
        detection-ports:
          dp: 53
      udp:
        enabled: yes
        detection-ports:
          dp: 53

rule-files:
  - suricata.rules
  - /var/lib/suricata/rules/emerging-threats.rules
  - /var/lib/suricata/rules/hospital-custom.rules

classification-file: /etc/suricata/classification.config
reference-config-file: /etc/suricata/reference.config

default-rule-path: /var/lib/suricata/rules

detect:
  profile: medium
  custom-values:
    toclient-groups: 3
    toserver-groups: 25
  sgh-mpm-context: auto
  inspection-recursion-limit: 3000

mpm-algo: auto

threading:
  set-cpu-affinity: no
  cpu-affinity:
    - management-cpu-set:
        cpu: [ 0 ]
    - receive-cpu-set:
        cpu: [ 0 ]
    - worker-cpu-set:
        cpu: [ "all" ]
        mode: "exclusive"
        prio:
          low: [ 0 ]
          medium: [ "1-2" ]
          high: [ 3 ]
          default: "medium"

profiling:
  rules:
    enabled: yes
    filename: rule_perf.log
    append: yes
    sort: avgticks
    limit: 10
  keywords:
    enabled: yes
    filename: keyword_perf.log
    append: yes
  rulegroups:
    enabled: yes
    filename: rule_group_perf.log
    append: yes

pcap-file:
  checksum-checks: auto

logging:
  default-log-level: notice
  outputs:
  - console:
      enabled: yes
  - file:
      enabled: yes
      level: info
      filename: suricata.log
  - syslog:
      enabled: no
      facility: local5
      format: "[%i] <%d> -- "
\end{lstlisting}

\subsection{Regles Personnalisees pour EternalBlue}

\begin{lstlisting}[caption=Regles Suricata pour detection EternalBlue]
# EternalBlue Detection Rules - hospital-custom.rules

# SMBv1 Negotiate detection
alert smb any any -> any 445 (msg:"ETERNALBLUE SMBv1 Negotiate Attempt"; 
  flow:to_server; content:"|ff|SMB|72|"; offset:4; depth:4;
  content:"|00 00 00 2f|"; offset:0; depth:4;
  sid:2024001; rev:1; classtype:attempted-admin;)

# EternalBlue exploit packet detection
alert tcp any any -> any 445 (msg:"ETERNALBLUE MS17-010 SMB Exploit Detected"; 
  flow:to_server,established; content:"|00 00 00 2f fe 53 4d 42|"; 
  depth:8; offset:0; content:"|72 00 00 00 00 18 01 20|"; 
  distance:0; within:16; sid:2024002; rev:1; classtype:trojan-activity;)

# DoublePulsar backdoor detection
alert tcp any any -> any 445 (msg:"ETERNALBLUE DoublePulsar Backdoor Communication"; 
  flow:to_server; content:"DoublePulsar"; nocase; 
  sid:2024003; rev:1; classtype:trojan-activity;)

# High volume SMB traffic (potential brute force)
alert tcp any any -> any 445 (msg:"ETERNALBLUE High Volume SMB Traffic - Potential Attack";
  flow:to_server; threshold:type both, track by_src, count 100, seconds 60;
  sid:2024004; rev:1; classtype:attempted-dos;)

# SMB buffer overflow attempt
alert tcp any any -> any 445 (msg:"ETERNALBLUE SMB Buffer Overflow Attempt";
  flow:to_server; content:"|00 00 00|"; depth:3; offset:0;
  byte_test:4,>,1000,0,relative; content:"|a0 00|"; distance:0; within:20;
  sid:2024005; rev:1; classtype:attempted-admin;)
\end{lstlisting}

\section{Configuration Wazuh}

\subsection{Regles de Correlation EternalBlue}

\begin{lstlisting}[style=xmlstyle,caption=Regles Wazuh pour EternalBlue]
<!-- EternalBlue detection rules -->
<group name="eternalblue,windows,exploit">

  <!-- Windows Event 4625 - Failed logon after EternalBlue attempt -->
  <rule id="100001" level="12">
    <if_sid>4625</if_sid>
    <field name="win.eventdata.logonType">^3$</field>
    <field name="win.eventdata.status">0xc000006d</field>
    <description>EternalBlue: Suspicious failed network logon detected</description>
    <group>authentication_failed,eternalblue</group>
  </rule>

  <!-- SMB connection anomaly detection -->
  <rule id="100002" level="10">  
    <if_sid>18152</if_sid>
    <match>445/tcp</match>
    <regex>SYN_FLOOD|CONN_FLOOD</regex>
    <description>EternalBlue: SMB connection flood detected</description>
    <group>network_flood,eternalblue</group>
  </rule>

  <!-- Process creation after SMB exploitation -->
  <rule id="100003" level="13">
    <decoded_as>windows-eventlog</decoded_as>
    <field name="win.system.eventID">^1$</field>
    <field name="win.eventdata.image">cmd.exe|powershell.exe|rundll32.exe</field>
    <regex>CreateRemoteThread|WriteProcessMemory|VirtualAllocEx</regex>
    <description>EternalBlue: Suspicious process injection after SMB connection</description>
    <group>process_injection,eternalblue</group>
  </rule>

  <!-- Registry modification indicating persistence -->
  <rule id="100004" level="11">
    <decoded_as>windows-eventlog</decoded_as>
    <field name="win.system.eventID">^13$</field>
    <field name="win.eventdata.targetObject">Run|RunOnce|Winlogon</field>
    <time>same_minute</time>
    <if_matched_sid>100003</if_matched_sid>
    <description>EternalBlue: Registry persistence mechanism detected</description>
    <group>persistence,eternalblue</group>
  </rule>

  <!-- Network communication to suspicious IPs -->
  <rule id="100005" level="10">
    <decoded_as>windows-eventlog</decoded_as>
    <field name="win.system.eventID">^3$</field>
    <field name="win.eventdata.protocol">TCP</field>
    <regex>192.168.183.100|external_c2_list</regex>
    <time>same_hour</time>
    <if_matched_sid>100003</if_matched_sid>
    <description>EternalBlue: Suspicious network communication detected</description>
    <group>command_control,eternalblue</group>
  </rule>

</group>
\end{lstlisting}

\subsection{Configuration Webhook n8n}

\begin{lstlisting}[style=xmlstyle,caption=Configuration Webhook Wazuh vers n8n]
<!-- Integration with n8n for SOAR automation -->
<integration>
  <name>custom-eternalblue-webhook</name>
  <hook_url>http://sbihi.soar.ma:5678/webhook/eternalblue-alert</hook_url>
  <level>10</level>
  <group>eternalblue</group>
  <alert_format>json</alert_format>
</integration>

<integration>
  <name>custom-dns-integration</name>
  <hook_url>http://sbihi.soar.ma:5678/webhook/wazuh-sysmon</hook_url>
  <level>3</level>
  <group>sysmon_event_22</group>
  <alert_format>json</alert_format>
</integration>

<integration>
  <name>custom-ssh-webhook</name>
  <hook_url>http://sbihi.soar.ma:5678/webhook/wazuh-ssh</hook_url>
  <level>3</level>
  <alert_format>json</alert_format>
  <rule_id>40111,60122,5758,2502,5710,5760,5763,5503</rule_id>
</integration>
\end{lstlisting}

\section{Configuration ModSecurity}

\subsection{Regles Personnalisees XSS}

\begin{lstlisting}[caption=Regles ModSecurity pour protection XSS avancee]
# Advanced XSS Protection Rules for Hospital Environment

# IP tracking for debugging
SecRule REQUEST_HEADERS:X-Forwarded-For "@rx ." \
    "id:900100,phase:1,pass,log,\
     msg:'X-Forwarded-For header detected: %{REQUEST_HEADERS.X-Forwarded-For}'"

SecRule REQUEST_HEADERS:X-Real-IP "@rx ." \
    "id:900101,phase:1,pass,log,\
     msg:'X-Real-IP header detected: %{REQUEST_HEADERS.X-Real-IP}'"

# Docker bridge traffic detection
SecRule REMOTE_ADDR "@ipMatch 172.20.0.0/16" \
    "id:900102,phase:1,pass,log,\
     msg:'Docker bridge traffic detected from: %{REMOTE_ADDR}'"

# Real attacker IP detection
SecRule REMOTE_ADDR "@rx ^192\.168\.1\." \
    "id:900103,phase:1,pass,log,\
     msg:'Real attacker IP detected: %{REMOTE_ADDR}'"

# Advanced XSS detection with context
SecRule ARGS "@detectXSS" \
    "id:1001,phase:2,block,\
     msg:'XSS Attack Detected in Arguments',\
     logdata:'Matched Data: %{MATCHED_VAR} found in %{MATCHED_VAR_NAME}',\
     tag:'application-multi',tag:'language-multi',tag:'attack-xss',\
     severity:'CRITICAL'"

# Hospital-specific XSS patterns
SecRule ARGS "@rx (?i)(\<script[^>]*\>[\s\S]*?\</script\>|javascript:|data:text/html|vbscript:)" \
    "id:1002,phase:2,block,\
     msg:'XSS Attack - Script Injection in Hospital System',\
     tag:'attack-xss',tag:'hospital-specific',\
     severity:'HIGH'"

# Event handler XSS detection
SecRule ARGS "@rx (?i)on(abort|blur|change|click|error|focus|load|reset|submit)\s*=" \
    "id:1003,phase:2,block,\
     msg:'XSS Attack - Event Handler in Medical Application',\
     tag:'attack-xss',tag:'event-handler',\
     severity:'HIGH'"

# Encoded XSS attempts
SecRule ARGS "@rx (?i)(%3C|&lt;|\\x3c)(script|img|svg|iframe)" \
    "id:1004,phase:2,block,\
     msg:'XSS Attack - Encoded Script Tag',\
     tag:'attack-xss',tag:'encoded',\
     severity:'MEDIUM'"

# Medical form specific protection
SecRule REQUEST_URI "@contains /patient/" \
    "id:1010,phase:1,pass,nolog,\
     setvar:'tx.patient_form=1'"

SecRule ARGS "@rx (?i)(patient|medical|diagnosis|prescription)" \
    "id:1011,phase:2,pass,\
     msg:'Medical data context detected',\
     tag:'medical-data',\
     chain"
    SecRule TX:patient_form "@eq 1" \
        "msg:'Enhanced XSS protection for patient data forms',\
         tag:'patient-protection'"

# Response monitoring for successful attacks
SecRule RESPONSE_BODY "@rx (?i)(\<script|\%3cscript)" \
    "id:1020,phase:4,block,\
     msg:'XSS payload detected in response',\
     tag:'response-xss',\
     severity:'HIGH'"
\end{lstlisting}

\chapter{Scripts d'Automatisation}

\section{Script d'Attachement TheHive}

\begin{lstlisting}[style=jsstyle,caption=Script n8n pour attachement de fichiers TheHive]
// n8n_thehive_attach.js - TheHive file attachment automation

const fs = require('fs');
const path = require('path');
const FormData = require('form-data');
const axios = require('axios');

async function attachFileToTheHive(config) {
    const {
        thehive_url,
        api_key,
        alert_id,
        file_path,
        file_name,
        description = 'Automated evidence attachment'
    } = config;

    try {
        // Verify file exists
        if (!fs.existsSync(file_path)) {
            throw new Error(`File not found: ${file_path}`);
        }

        // Get file stats
        const stats = fs.statSync(file_path);
        console.log(`[+] File size: ${stats.size} bytes`);

        // Create form data
        const form = new FormData();
        form.append('attachment', fs.createReadStream(file_path), {
            filename: file_name,
            contentType: 'application/octet-stream'
        });

        // Add metadata
        form.append('description', description);
        form.append('tags', JSON.stringify(['evidence', 'automated', 'pcap']));

        // TheHive API request
        const response = await axios.post(
            `${thehive_url}/api/alert/${alert_id}/artifact`,
            form,
            {
                headers: {
                    'Authorization': `Bearer ${api_key}`,
                    'Content-Type': `multipart/form-data; boundary=${form.getBoundary()}`
                },
                timeout: 30000,
                maxContentLength: Infinity,
                maxBodyLength: Infinity
            }
        );

        console.log('[+] File attached successfully');
        console.log(`[+] Artifact ID: ${response.data._id}`);
        
        return {
            success: true,
            artifact_id: response.data._id,
            file_name: file_name,
            file_size: stats.size
        };

    } catch (error) {
        console.error('[-] Error attaching file:', error.message);
        
        if (error.response) {
            console.error('[-] API Error:', error.response.status, error.response.data);
        }
        
        return {
            success: false,
            error: error.message
        };
    }
}

// Usage in n8n Code Node
const config = {
    thehive_url: 'http://thehive.sbihi.soar.ma',
    api_key: 'HSTx8PnJZNVvHwYFGs+564VD7pfqsRAj',
    alert_id: $json.alert_id || $input.first().json.alert_id,
    file_path: $json.evidence_file || '/tmp/evidence.pcap',
    file_name: $json.file_name || 'network_evidence.pcap',
    description: `Evidence from ${$json.attack_type || 'unknown'} attack`
};

// Execute attachment
const result = await attachFileToTheHive(config);

// Return result for next node
return [result];
\end{lstlisting}

\section{Script de Blocage OPNsense}

\begin{lstlisting}[style=jsstyle,caption=Script n8n pour blocage automatique IP OPNsense]
// n8n_opnsense_final.js - OPNsense automatic IP blocking

const axios = require('axios');
const crypto = require('crypto');

class OPNsenseFirewall {
    constructor(config) {
        this.url = config.url;
        this.apiKey = config.api_key;
        this.apiSecret = config.api_secret;
        this.aliasName = config.alias_name || 'Black_list';
        
        // Create authentication header
        this.authHeader = 'Basic ' + Buffer.from(
            `${this.apiKey}:${this.apiSecret}`
        ).toString('base64');
    }

    async makeApiCall(endpoint, method = 'GET', data = null) {
        const config = {
            method: method,
            url: `${this.url}${endpoint}`,
            headers: {
                'Authorization': this.authHeader,
                'Content-Type': 'application/json'
            },
            timeout: 15000
        };

        if (data && method !== 'GET') {
            config.data = data;
        }

        try {
            const response = await axios(config);
            return response.data;
        } catch (error) {
            console.error(`[-] API call failed for ${endpoint}:`, error.message);
            throw error;
        }
    }

    async getAliasId() {
        try {
            const aliases = await this.makeApiCall('/api/firewall/alias/searchItem');
            
            for (let alias of aliases.rows || []) {
                if (alias.name === this.aliasName) {
                    console.log(`[+] Found alias ${this.aliasName} with ID: ${alias.uuid}`);
                    return alias.uuid;
                }
            }
            
            console.log(`[-] Alias ${this.aliasName} not found`);
            return null;
        } catch (error) {
            console.error('[-] Error getting alias ID:', error.message);
            throw error;
        }
    }

    async getCurrentIPs(aliasId) {
        try {
            const aliasData = await this.makeApiCall(`/api/firewall/alias/getItem/${aliasId}`);
            
            if (aliasData.alias && aliasData.alias.content) {
                const currentIPs = aliasData.alias.content.split(',')
                    .map(ip => ip.trim())
                    .filter(ip => ip.length > 0);
                
                console.log(`[+] Current IPs in blacklist: ${currentIPs.length}`);
                return currentIPs;
            }
            
            return [];
        } catch (error) {
            console.error('[-] Error getting current IPs:', error.message);
            return [];
        }
    }

    async blockIP(ipToBlock, reason = 'Automated SOAR blocking') {
        try {
            console.log(`[+] Starting IP blocking process for: ${ipToBlock}`);
            
            // Get alias ID
            const aliasId = await this.getAliasId();
            if (!aliasId) {
                throw new Error(`Alias ${this.aliasName} not found`);
            }

            // Get current IPs
            const currentIPs = await this.getCurrentIPs(aliasId);
            
            // Check if IP already blocked
            if (currentIPs.includes(ipToBlock)) {
                console.log(`[!] IP ${ipToBlock} already in blacklist`);
                return {
                    success: true,
                    action: 'already_blocked',
                    ip: ipToBlock,
                    message: 'IP was already in blacklist'
                };
            }

            // Add new IP to list
            const updatedIPs = [...currentIPs, ipToBlock];
            const payload = {
                alias: {
                    enabled: "1",
                    name: this.aliasName,
                    type: "host",
                    content: updatedIPs.join(','),
                    description: `Auto-managed blacklist - Last update: ${new Date().toISOString()}`
                }
            };

            // Update alias
            console.log('[+] Updating firewall alias...');
            const updateResponse = await this.makeApiCall(
                `/api/firewall/alias/setItem/${aliasId}`, 
                'POST', 
                payload
            );

            if (updateResponse.result !== 'saved') {
                throw new Error(`Failed to update alias: ${updateResponse.result}`);
            }

            // Apply changes
            console.log('[+] Applying firewall configuration...');
            const applyResponse = await this.makeApiCall(
                '/api/firewall/alias/reconfigure', 
                'POST'
            );

            if (applyResponse.status !== 'ok') {
                throw new Error(`Failed to apply configuration: ${applyResponse.status}`);
            }

            console.log(`[+] Successfully blocked IP: ${ipToBlock}`);
            
            // Log the action
            const logEntry = {
                timestamp: new Date().toISOString(),
                action: 'ip_blocked',
                ip: ipToBlock,
                reason: reason,
                total_blocked_ips: updatedIPs.length
            };

            return {
                success: true,
                action: 'blocked',
                ip: ipToBlock,
                total_ips: updatedIPs.length,
                log: logEntry
            };

        } catch (error) {
            console.error(`[-] Error blocking IP ${ipToBlock}:`, error.message);
            
            return {
                success: false,
                action: 'error',
                ip: ipToBlock,
                error: error.message
            };
        }
    }

    async unblockIP(ipToUnblock) {
        try {
            console.log(`[+] Starting IP unblocking process for: ${ipToUnblock}`);
            
            const aliasId = await this.getAliasId();
            if (!aliasId) {
                throw new Error(`Alias ${this.aliasName} not found`);
            }

            const currentIPs = await this.getCurrentIPs(aliasId);
            
            if (!currentIPs.includes(ipToUnblock)) {
                console.log(`[!] IP ${ipToUnblock} not found in blacklist`);
                return {
                    success: true,
                    action: 'not_found',
                    ip: ipToUnblock
                };
            }

            // Remove IP from list
            const updatedIPs = currentIPs.filter(ip => ip !== ipToUnblock);
            
            const payload = {
                alias: {
                    enabled: "1",
                    name: this.aliasName,
                    type: "host",
                    content: updatedIPs.join(','),
                    description: `Auto-managed blacklist - Last update: ${new Date().toISOString()}`
                }
            };

            // Update and apply
            await this.makeApiCall(`/api/firewall/alias/setItem/${aliasId}`, 'POST', payload);
            await this.makeApiCall('/api/firewall/alias/reconfigure', 'POST');

            console.log(`[+] Successfully unblocked IP: ${ipToUnblock}`);
            
            return {
                success: true,
                action: 'unblocked',
                ip: ipToUnblock,
                total_ips: updatedIPs.length
            };

        } catch (error) {
            console.error(`[-] Error unblocking IP ${ipToUnblock}:`, error.message);
            
            return {
                success: false,
                action: 'error',
                ip: ipToUnblock,
                error: error.message
            };
        }
    }
}

// Main execution in n8n
const opnsenseConfig = {
    url: "http://192.168.181.1",
    api_key: "YOUR_API_KEY",
    api_secret: "YOUR_API_SECRET",
    alias_name: "Black_list"
};

const firewall = new OPNsenseFirewall(opnsenseConfig);

// Get IP to block from input
const ipToBlock = $input.first().json.source_ip || $json.ip_address;
const reason = $input.first().json.alert_description || 'SOAR automated blocking';

if (!ipToBlock) {
    return [{
        success: false,
        error: 'No IP address provided'
    }];
}

// Validate IP format
const ipRegex = /^(?:(?:25[0-5]|2[0-4][0-9]|[01]?[0-9][0-9]?)\.){3}(?:25[0-5]|2[0-4][0-9]|[01]?[0-9][0-9]?)$/;
if (!ipRegex.test(ipToBlock)) {
    return [{
        success: false,
        error: `Invalid IP address format: ${ipToBlock}`
    }];
}

// Execute blocking
const result = await firewall.blockIP(ipToBlock, reason);

// Return result
return [result];
\end{lstlisting}

\chapter{Metriques et Resultats de Tests}

\section{Tableau de Bord des Performances}

\begin{table}[H]
    \centering
    \caption{Resultats des Tests de Performance par Scenario d'Attaque}
    \begin{tabular}{|l|c|c|c|c|c|}
        \hline
        \textbf{Scenario}  & \textbf{Tests} & \textbf{Detectes} & \textbf{Taux}   & \textbf{Temps Moy.} & \textbf{Faux Pos.} \\
        \hline
        EternalBlue        & 15             & 14                & 93.3\%          & 2.3s                & 2                  \\
        \hline
        XSS Attacks        & 50             & 47                & 94.0\%          & 0.1s                & 5                  \\
        \hline
        Sites Malveillants & 100            & 85                & 85.0\%          & 1.8s                & 8                  \\
        \hline
        SSH Brute Force    & 25             & 24                & 96.0\%          & 0.5s                & 1                  \\
        \hline
        \textbf{TOTAL}     & \textbf{190}   & \textbf{170}      & \textbf{89.5\%} & \textbf{1.2s}       & \textbf{16}        \\
        \hline
    \end{tabular}
\end{table}

\section{Analyse des IOCs Collectes}

\subsection{Indicateurs de Compromission EternalBlue}

\begin{lstlisting}[style=jsstyle,caption=IOCs extraits lors des tests EternalBlue]
{
  "eternalblue_iocs": {
    "attack_date": "2025-07-30T19:04:34Z",
    "source_ip": "192.168.183.100",
    "target_ip": "192.168.183.10",
    "network_indicators": [
      {
        "type": "port_scan",
        "protocol": "TCP",
        "port": 445,
        "packets": 127,
        "description": "SMB port enumeration"
      },
      {
        "type": "smb_negotiate",
        "signature": "\\xff\\x53\\x4d\\x42\\x72",
        "length": 47,
        "description": "SMBv1 negotiate request"
      },
      {
        "type": "exploit_packet",
        "signature": "\\x00\\x00\\x00\\x2f\\xfe\\x53\\x4d\\x42",
        "size": 2048,
        "description": "EternalBlue buffer overflow"
      }
    ],
    "file_indicators": [
      {
        "type": "hash",
        "algorithm": "MD5",
        "value": "c1d5cf8c43e7679b782eca6fdf9a5ad7",
        "description": "EternalBlue payload"
      },
      {
        "type": "hash", 
        "algorithm": "SHA256",
        "value": "ed01ebfbc9eb5bbea545af4d01bf5f1071661840480439c6e5babe8e080e41aa",
        "description": "DoublePulsar backdoor"
      }
    ],
    "process_indicators": [
      {
        "type": "process_creation",
        "parent": "svchost.exe",
        "child": "cmd.exe",
        "command_line": "cmd.exe /c whoami",
        "description": "Suspicious process spawn"
      },
      {
        "type": "network_connection",
        "process": "cmd.exe",
        "destination": "192.168.183.100:4444",
        "description": "Reverse shell connection"
      }
    ],
    "timeline": [
      "19:04:34 - Initial port scan",
      "19:04:35 - SMBv1 service discovery", 
      "19:04:36 - EternalBlue exploit launched",
      "19:04:37 - Buffer overflow successful",
      "19:04:38 - Shellcode execution",
      "19:04:39 - DoublePulsar installation",
      "19:04:40 - C2 communication established"
    ]
  }
}
\end{lstlisting}

\chapter{Guides d'Installation}

\section{Installation Automatisee}

\begin{lstlisting}[style=bashstyle,caption=Script d'installation complete de la stack SOAR]
#!/bin/bash
# install_soar_stack.sh - Complete SOAR stack installation

set -e

# Colors for output
RED='\033[0;31m'
GREEN='\033[0;32m'
YELLOW='\033[1;33m'
NC='\033[0m' # No Color

# Configuration
INSTALL_DIR="/opt/soar"
LOG_FILE="/var/log/soar_install.log"
DOCKER_COMPOSE_VERSION="2.20.0"

# Functions
log() {
    echo -e "${GREEN}[$(date +'%Y-%m-%d %H:%M:%S')] $1${NC}" | tee -a "$LOG_FILE"
}

error() {
    echo -e "${RED}[ERROR] $1${NC}" | tee -a "$LOG_FILE"
    exit 1
}

warning() {
    echo -e "${YELLOW}[WARNING] $1${NC}" | tee -a "$LOG_FILE"
}

check_requirements() {
    log "Checking system requirements..."
    
    # Check OS
    if [[ ! -f /etc/os-release ]]; then
        error "Unsupported operating system"
    fi
    
    source /etc/os-release
    if [[ "$ID" != "ubuntu" && "$ID" != "debian" ]]; then
        warning "Untested OS: $ID. Proceeding anyway..."
    fi
    
    # Check resources
    MEMORY_GB=$(free -g | awk '/^Mem:/{print $2}')
    if [[ $MEMORY_GB -lt 8 ]]; then
        warning "Less than 8GB RAM detected. Performance may be affected."
    fi
    
    DISK_GB=$(df -BG / | awk 'NR==2{print $4}' | sed 's/G//')
    if [[ $DISK_GB -lt 50 ]]; then
        error "Insufficient disk space. At least 50GB required."
    fi
    
    log "System requirements check completed"
}

install_docker() {
    log "Installing Docker..."
    
    if command -v docker &> /dev/null; then
        log "Docker already installed: $(docker --version)"
        return
    fi
    
    # Update package index
    apt-get update
    
    # Install prerequisites
    apt-get install -y \
        apt-transport-https \
        ca-certificates \
        curl \
        gnupg \
        lsb-release
    
    # Add Docker GPG key
    curl -fsSL https://download.docker.com/linux/ubuntu/gpg | gpg --dearmor -o /usr/share/keyrings/docker-archive-keyring.gpg
    
    # Add Docker repository
    echo "deb [arch=amd64 signed-by=/usr/share/keyrings/docker-archive-keyring.gpg] https://download.docker.com/linux/ubuntu $(lsb_release -cs) stable" | tee /etc/apt/sources.list.d/docker.list > /dev/null
    
    # Install Docker
    apt-get update
    apt-get install -y docker-ce docker-ce-cli containerd.io
    
    # Start and enable Docker
    systemctl start docker
    systemctl enable docker
    
    # Add user to docker group
    usermod -aG docker $SUDO_USER
    
    log "Docker installation completed"
}

install_docker_compose() {
    log "Installing Docker Compose..."
    
    if command -v docker-compose &> /dev/null; then
        log "Docker Compose already installed: $(docker-compose --version)"
        return
    fi
    
    # Download Docker Compose
    curl -L "https://github.com/docker/compose/releases/download/v${DOCKER_COMPOSE_VERSION}/docker-compose-$(uname -s)-$(uname -m)" -o /usr/local/bin/docker-compose
    
    # Make executable
    chmod +x /usr/local/bin/docker-compose
    
    # Create symlink
    ln -sf /usr/local/bin/docker-compose /usr/bin/docker-compose
    
    log "Docker Compose installation completed"
}

setup_directories() {
    log "Setting up directories..."
    
    # Create main directory
    mkdir -p "$INSTALL_DIR"
    cd "$INSTALL_DIR"
    
    # Create subdirectories
    mkdir -p {wazuh,suricata,modsecurity,thehive,cortex,misp,n8n}
    mkdir -p {logs,data,configs,scripts}
    mkdir -p data/{wazuh,elasticsearch,cassandra,mysql,redis}
    
    # Set permissions
    chown -R $SUDO_USER:$SUDO_USER "$INSTALL_DIR"
    
    log "Directory structure created"
}

configure_network() {
    log "Configuring network settings..."
    
    # Create docker networks
    docker network create --driver bridge soar-network --subnet=172.20.0.0/16 || true
    docker network create --driver bridge wazuh-network --subnet=172.21.0.0/16 || true
    
    # Configure firewall (if UFW is available)
    if command -v ufw &> /dev/null; then
        ufw --force enable
        ufw allow 22/tcp
        ufw allow 9000/tcp  # TheHive
        ufw allow 9001/tcp  # Cortex
        ufw allow 5678/tcp  # n8n
        ufw allow 443/tcp   # HTTPS
        ufw allow 1514/tcp  # Wazuh
        ufw allow 1515/tcp  # Wazuh
    fi
    
    log "Network configuration completed"
}

deploy_wazuh() {
    log "Deploying Wazuh SIEM..."
    
    cd "$INSTALL_DIR/wazuh"
    
    # Download Wazuh docker-compose
    curl -so wazuh-docker.tar.gz https://packages.wazuh.com/4.7/wazuh-docker.tar.gz
    tar -xvf wazuh-docker.tar.gz --strip-components=1
    
    # Generate certificates
    docker-compose -f generate-indexer-certs.yml run --rm generator
    
    # Start Wazuh
    docker-compose up -d
    
    # Wait for services
    log "Waiting for Wazuh services to start..."
    sleep 60
    
    log "Wazuh deployment completed"
}

deploy_soar_stack() {
    log "Deploying SOAR stack..."
    
    cd "$INSTALL_DIR"
    
    # Create main docker-compose file
    cat > docker-compose.yml << 'EOF'
version: '3.8'

networks:
  soar-network:
    external: true

services:
  # TheHive
  cassandra:
    image: cassandra:4.0
    container_name: cassandra
    environment:
      - CASSANDRA_CLUSTER_NAME=thehive
    volumes:
      - ./data/cassandra:/var/lib/cassandra
    networks:
      - soar-network

  elasticsearch:
    image: elasticsearch:7.17.0
    container_name: elasticsearch
    environment:
      - discovery.type=single-node
      - xpack.security.enabled=false
    volumes:
      - ./data/elasticsearch:/usr/share/elasticsearch/data
    networks:
      - soar-network

  thehive:
    image: thehiveproject/thehive:5.2
    container_name: thehive
    depends_on:
      - cassandra
      - elasticsearch
    ports:
      - "9000:9000"
    volumes:
      - ./configs/thehive:/etc/thehive
      - ./data/thehive:/opt/thp/thehive/data
    networks:
      - soar-network

  # Cortex
  cortex:
    image: thehiveproject/cortex:3.1
    container_name: cortex
    depends_on:
      - elasticsearch
    ports:
      - "9001:9001"
    volumes:
      - ./configs/cortex:/etc/cortex
      - /var/run/docker.sock:/var/run/docker.sock
    networks:
      - soar-network

  # MISP
  redis:
    image: redis:7.0
    container_name: redis
    volumes:
      - ./data/redis:/data
    networks:
      - soar-network

  mysql:
    image: mysql:8.0
    container_name: mysql
    environment:
      - MYSQL_ROOT_PASSWORD=password
      - MYSQL_DATABASE=misp
      - MYSQL_USER=misp
      - MYSQL_PASSWORD=example
    volumes:
      - ./data/mysql:/var/lib/mysql
    networks:
      - soar-network

  misp:
    image: coolacid/misp-docker:core-latest
    container_name: misp
    depends_on:
      - redis
      - mysql
    ports:
      - "4432:80"
    environment:
      - MYSQL_HOST=mysql
      - MYSQL_DATABASE=misp
      - MYSQL_USER=misp
      - MYSQL_PASSWORD=example
      - REDIS_HOST=redis
    networks:
      - soar-network

  # n8n
  n8n:
    image: n8nio/n8n:latest
    container_name: n8n
    ports:
      - "5678:5678"
    environment:
      - N8N_HOST=sbihi.soar.ma
      - N8N_PORT=5678
      - N8N_PROTOCOL=http
      - NODE_ENV=production
    volumes:
      - ./data/n8n:/home/node/.n8n
    networks:
      - soar-network
EOF

    # Start SOAR stack
    docker-compose up -d
    
    # Wait for services
    log "Waiting for SOAR services to start..."
    sleep 120
    
    log "SOAR stack deployment completed"
}

configure_integrations() {
    log "Configuring integrations..."
    
    # Configure Wazuh integrations
    WAZUH_CONFIG_PATH="$INSTALL_DIR/wazuh/single-node/config/wazuh_cluster/wazuh_manager.conf"
    
    if [[ -f "$WAZUH_CONFIG_PATH" ]]; then
        # Add webhook integrations
        cat >> "$WAZUH_CONFIG_PATH" << 'EOF'

<!-- SOAR Integrations -->
<integration>
  <name>custom-eternalblue-webhook</name>
  <hook_url>http://n8n:5678/webhook/eternalblue-alert</hook_url>
  <level>10</level>
  <group>eternalblue</group>
  <alert_format>json</alert_format>
</integration>

<integration>
  <name>custom-dns-integration</name>
  <hook_url>http://n8n:5678/webhook/wazuh-sysmon</hook_url>
  <level>3</level>
  <group>sysmon_event_22</group>
  <alert_format>json</alert_format>
</integration>
EOF
        
        # Restart Wazuh manager
        docker restart wazuh.manager
    fi
    
    log "Integration configuration completed"
}

install_additional_tools() {
    log "Installing additional tools..."
    
    # Install monitoring tools
    apt-get install -y htop iotop nethogs tcpdump wireshark-common
    
    # Install analysis tools
    apt-get install -y jq curl wget git vim
    
    log "Additional tools installation completed"
}

create_service_scripts() {
    log "Creating service management scripts..."
    
    cat > "$INSTALL_DIR/start_soar.sh" << 'EOF'
#!/bin/bash
echo "Starting SOAR Stack..."
cd /opt/soar
docker-compose up -d
echo "SOAR Stack started successfully"
EOF

    cat > "$INSTALL_DIR/stop_soar.sh" << 'EOF'
#!/bin/bash
echo "Stopping SOAR Stack..."
cd /opt/soar
docker-compose down
echo "SOAR Stack stopped"
EOF

    cat > "$INSTALL_DIR/status_soar.sh" << 'EOF'
#!/bin/bash
echo "SOAR Stack Status:"
cd /opt/soar
docker-compose ps
echo ""
echo "Resource Usage:"
docker stats --no-stream
EOF

    chmod +x "$INSTALL_DIR"/*.sh
    
    log "Service scripts created"
}

perform_health_checks() {
    log "Performing health checks..."
    
    # Check services
    SERVICES=("thehive:9000" "cortex:9001" "n8n:5678" "misp:4432")
    
    for service in "${SERVICES[@]}"; do
        IFS=':' read -r name port <<< "$service"
        if curl -f http://localhost:$port/api/status &>/dev/null || curl -f http://localhost:$port &>/dev/null; then
            log "$name service is healthy"
        else
            warning "$name service may not be fully ready"
        fi
    done
    
    log "Health checks completed"
}

display_summary() {
    log "Installation completed successfully!"
    echo ""
    echo "=== SOAR Stack Access URLs ==="
    echo "TheHive:  http://$(hostname -I | awk '{print $1}'):9000"
    echo "Cortex:   http://$(hostname -I | awk '{print $1}'):9001"
    echo "MISP:     http://$(hostname -I | awk '{print $1}'):4432"
    echo "n8n:      http://$(hostname -I | awk '{print $1}'):5678"
    echo "Wazuh:    https://$(hostname -I | awk '{print $1}'):443"
    echo ""
    echo "=== Default Credentials ==="
    echo "TheHive:  admin@thehive.local / secret"
    echo "Cortex:   admin@cortex.local / secret"  
    echo "MISP:     admin@admin.test / admin"
    echo "Wazuh:    admin / SecretPassword"
    echo ""
    echo "=== Management Commands ==="
    echo "Start:    $INSTALL_DIR/start_soar.sh"
    echo "Stop:     $INSTALL_DIR/stop_soar.sh"
    echo "Status:   $INSTALL_DIR/status_soar.sh"
    echo ""
    echo "=== Log Location ==="
    echo "Install Log: $LOG_FILE"
    echo "Service Logs: $INSTALL_DIR/logs/"
    echo ""
    echo "Please reboot the system to ensure all changes take effect."
}

# Main installation process
main() {
    log "Starting SOAR stack installation..."
    
    # Check if running as root
    if [[ $EUID -ne 0 ]]; then
        error "This script must be run as root (use sudo)"
    fi
    
    # Check if SUDO_USER is set
    if [[ -z "$SUDO_USER" ]]; then
        error "Please run this script with sudo, not as root directly"
    fi
    
    check_requirements
    install_docker
    install_docker_compose
    setup_directories
    configure_network
    deploy_wazuh
    deploy_soar_stack
    configure_integrations
    install_additional_tools
    create_service_scripts
    perform_health_checks
    display_summary
    
    log "Installation process completed successfully"
}

# Run main function
main "$@"
EOF

chmod +x install_soar_stack.sh
\end{lstlisting}

\newpage
