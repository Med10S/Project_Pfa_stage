\chapter{Structure du Projet et Présentation des Composants}

Ce chapitre présente l'organisation structurelle du projet et introduit chacun des composants majeurs de l'architecture SIEM/SOAR développée. Cette présentation suit la logique fonctionnelle de la solution, depuis les fondements conceptuels jusqu'aux tests de validation.

\section{Chapitre 1 - Contexte et Problématique}

\subsection{Analyse de l'Environnement Hospitalier}

Le premier chapitre établit les fondements du projet en analysant les spécificités de l'environnement hospitalier et les défis cybersécuritaires qui lui sont propres.

\subsubsection{Enjeux de la Cybersécurité Hospitalière}
Cette section examine :
\begin{itemize}
    \item L'augmentation des cyberattaques ciblant le secteur de la santé (+47\% selon l'ANSSI 2024)
    \item Les particularités de l'infrastructure hospitalière (criticité, hétérogénéité)
    \item Les contraintes réglementaires (HIPAA, RGPD)
    \item La valeur économique des données de santé (jusqu'à 250\$ sur le dark web)
\end{itemize}

\subsubsection{Typologie des Menaces Spécifiques}
Analyse détaillée des principales familles d'attaques :
\begin{itemize}
    \item \textbf{Ransomwares} : WannaCry , Ryuk, Lockbit
    \item \textbf{Compromission d'équipements médicaux} : Systèmes obsolètes, protocoles non sécurisés
    \item \textbf{Vecteurs d'attaque réseau} : EternalBlue (MS17-010), BlueKeep (CVE-2019-0708)
    \item \textbf{Attaques applicatives} : SQL injection, XSS, injection de commandes
\end{itemize}

\subsubsection{État de l'Art SIEM/SOAR}
Comparaison des solutions existantes :
\begin{itemize}
    \item \textbf{Solutions commerciales} : Splunk (150K EPS, 15€/GB), QRadar (100K EPS, 12€/GB), ArcSight, LogRhythm, Sentinel Azure
    \item \textbf{Solutions open source} : Wazuh (SIEM/XDR), OSSEC, ELK Stack, Graylog, OSSIM
    \item \textbf{Plateformes SOAR} : TheHive (gestion collaborative), Cortex (analyse automatisée), MISP (threat intelligence)
\end{itemize}

\subsubsection{Objectifs et Défis du Projet}
Définition des objectifs principaux :
\begin{itemize}
    \item \textbf{Détection précoce} : Temps de détection < 30 secondes
    \item \textbf{Réponse automatisée} : 80\% des incidents sans intervention humaine
    \item \textbf{Continuité de service} : Disponibilité > 99.9\%
    \item \textbf{Couverture MITRE ATT\&CK} : > 80\% des techniques
\end{itemize}

\section{Chapitre 2 - Méthodologie et Approche Technique}

\subsection{Méthodologie de Développement}

Le deuxième chapitre présente l'approche méthodologique adoptée pour le développement de la solution.

\subsubsection{Cycle de Vie du Projet DevSecOps}
Approche itérative structurée en phases :
\begin{itemize}
    \item \textbf{Phase d'Analyse} (1 semaine) : Audit infrastructure, identification sources de logs, mapping réglementaire
    \item \textbf{Phase de Conception} (3 semaines) : Architecture SIEM/SOAR, cas d'usage prioritaires, workflows d'automatisation
    \item \textbf{Phase d'Implémentation} (3 semaines) : Déploiement infrastructure, règles de corrélation, intégration SOAR
    \item \textbf{Phase de Tests} (2 semaines) : Tests de charge, validation scénarios d'attaque, audit sécurité
\end{itemize}

\subsubsection{Framework NIST Cybersecurity}
Alignement sur le framework NIST CSF pour structurer l'approche sécuritaire selon les fonctions Identify, Protect, Detect, Respond, Recover.

\newpage
\subsubsection{Architecture Technique Détaillée}
Présentation de l'architecture globale multi-couches :
\begin{itemize}
    \item \textbf{Couche de Collecte} : Sources de données (logs système, réseau, applications), agents Wazuh
    \item \textbf{Couche de Traitement et Corrélation} : Wazuh Manager, règles personnalisées, enrichissement géolocalisation
    \item \textbf{Couche d'Orchestration SOAR} : TheHive (gestion incidents), Cortex (analyse observables), MISP (threat intelligence)
    \item \textbf{Couche d'Intégration} : n8n workflows, APIs REST, automatisation réponses
\end{itemize}

\subsection{Diagrammes de Flux et Architecture}

\subsubsection{Flux de Données Simplifié}
Modélisation des flux depuis la collecte jusqu'à la réponse automatisée, avec pipeline ETL temps réel optimisé pour la réactivité.

\section{Chapitre 3 - Implémentation et Configuration}

\subsection{Déploiement de l'Infrastructure}

Le troisième chapitre détaille l'implémentation pratique de la solution dans l'environnement de laboratoire.

\subsubsection{Environnement de Laboratoire}
Architecture de test reproduisant fidèlement l'écosystème hospitalier :
\begin{itemize}
    \item \textbf{Segment Production} (192.168.15.0/24) : SIH (Windows Server 2019), PACS (Windows Server 2016), Workstations (Windows 10)
    \item \textbf{Segment Attaquant} (192.168.183.0/24) : Kali Linux, Metasploit, outils Red Team
    \item \textbf{Segment SIEM/SOAR} (192.168.3.0/24) : Wazuh, TheHive, Cortex, MISP, n8n
    \item \textbf{Segment Defense} (192.168.181.0/24) : pfSense firewall, ModSecurity WAF
\end{itemize}

\subsubsection{Configuration des Composants SIEM}
Déploiement et configuration détaillée :
\begin{itemize}
    \item \textbf{Wazuh Manager} : Serveur central de corrélation, règles personnalisées hospitalières
    \item \textbf{Wazuh Indexer} : Stockage OpenSearch des événements
    \item \textbf{Wazuh Dashboard} : Interface Kibana pour visualisation et analyse
    \item \textbf{Agents Wazuh} : Collecteurs distribués sur endpoints Windows/Linux
\end{itemize}

\subsubsection{Configuration des Composants SOAR}
Stack d'orchestration et d'automatisation :
\begin{itemize}
    \item \textbf{TheHive} : Templates hospitaliers, workflows collaboratifs, API REST
    \item \textbf{Cortex} : 100+ analyzers (VirusTotal, AbuseIPDB, URLVoid), responders personnalisés
    \item \textbf{MISP} : Feeds threat intelligence, objets MISP personnalisés pour le médical
    \item \textbf{n8n} : Workflows visuels, connecteurs API, automatisation multi-canal
\end{itemize}

\subsubsection{Détection Réseau et Applicative}
Configuration avancée des systèmes de détection :
\begin{itemize}
    \item \textbf{Suricata IDS/IPS} : 30K+ règles ET Open, règles personnalisées, moteurs parallèles
    \item \textbf{ModSecurity WAF} : OWASP CRS, règles hospitalières, machine learning anti-obfuscation
\end{itemize}

\section{Chapitre 4 - Tests et Validation}

\subsection{Méthodologie de Test Red Team/Blue Team}

Le quatrième chapitre présente la validation de la solution à travers des scénarios d'attaque contrôlés.

\subsubsection{Environnement de Test Contrôlé}
Infrastructure de test sécurisée permettant la simulation d'attaques réalistes :
\begin{itemize}
    \item \textbf{Équipe Red Team} : Kali Linux (192.168.183.2), Metasploit, payloads personnalisés
    \item \textbf{Équipe Blue Team} : Stack SOAR complète, monitoring temps réel, workflows automatisés
    \item \textbf{Métriques} : Temps de détection, précision, taux de faux positifs, temps de réponse
\end{itemize}

\subsubsection{Scénarios d'Attaque Implémentés}

\paragraph{Scénario 1 : EternalBlue (CVE-2017-0144)}
Test d'exploitation SMBv1 avec méthodologie complète :
\begin{itemize}
    \item \textbf{Reconnaissance} : Scan ports SMB, identification versions vulnérables
    \item \textbf{Exploitation} : Payload EternalBlue personnalisé, reverse engineering
    \item \textbf{Post-exploitation} : Backdoor DoublePulsar, persistance système
    \item \textbf{Détection} : Suricata signatures spécifiques, corrélation Wazuh events Windows
    \item \textbf{Réponse automatisée} : Isolation réseau via n8n, capture forensique, notifications
\end{itemize}

\paragraph{Scénario 2 : Cross-Site Scripting (XSS)}
Tests d'attaques applicatives web sur DVWA :
\begin{itemize}
    \item \textbf{Variantes testées} : Reflected XSS, Stored XSS, DOM-based XSS
    \item \textbf{Techniques de bypass} : Obfuscation, encoding, contournement WAF
    \item \textbf{Protection ModSecurity} : OWASP CRS, règles personnalisées, ML anti-obfuscation
    \item \textbf{Détection} : Analyse patterns malveillants, corrélation événements applicatifs
    \item \textbf{Réponse} : Blocage automatique, logging détaillé, alertes TheHive
\end{itemize}

\paragraph{Scénario 3 : Sites Malveillants et DNS Monitoring}
Simulation de trafic malveillant via monitoring DNS :
\begin{itemize}
    \item \textbf{Monitoring Sysmon} : Event ID 22 pour requêtes DNS, configuration SwiftOnSecurity
    \item \textbf{Intégration Wazuh} : Règle 61650 pour détection domaines malveillants
    \item \textbf{Workflow n8n} : 20 nœuds interconnectés, webhook reception, analyse automatisée
    \item \textbf{Types simulés} : Communications C2, exfiltration DNS tunneling, téléchargements malware
    \item \textbf{Réponse} : Création cases TheHive, analyse Cortex, blocage automatique domaines
\end{itemize}

\subsubsection{Métriques et Évaluation des Performances}
Analyse quantitative des résultats :
\begin{itemize}
    \item \textbf{Temps de détection} : < 5 secondes pour tous les scénarios
    \item \textbf{Taux de détection} : 100\% pour attaques connues, 90\% pour variantes
    \item \textbf{Faux positifs} : < 2\% après tuning des règles
    \item \textbf{Temps de réponse automatisée} : < 30 secondes pour containment
    \item \textbf{Couverture MITRE ATT\&CK} : 85\% des techniques testées
\end{itemize}

\subsection{Validation de l'Architecture}

\subsubsection{Tests de Charge et Performance}
Validation de la scalabilité :
\begin{itemize}
    \item \textbf{Throughput} : 50K+ événements/seconde traités sans perte
    \item \textbf{Latence} : < 100ms pour corrélation temps réel
    \item \textbf{Haute disponibilité} : Failover automatique testé et validé
\end{itemize}


\section{Cohérence Architecturale et Intégration}

Cette organisation en chapitres reflète la démarche méthodologique adoptée, partant de l'analyse du contexte hospitalier vers la validation pratique de la solution. Chaque composant s'intègre dans une architecture globale cohérente, facilitant :

\begin{itemize}
    \item \textbf{La maintenance} : Architecture modulaire et documentée
    \item \textbf{L'évolution} : Composants extensibles et configurables
    \item \textbf{L'adaptation} : Templates et workflows personnalisables par établissement
    \item \textbf{La réplication} : Documentation complète pour reproduction
\end{itemize}

La validation par des tests Red Team/Blue Team démontre l'efficacité opérationnelle de la solution dans la détection et la réponse aux cyberattaques ciblant les infrastructures hospitalières, tout en respectant les contraintes de continuité de service et de conformité réglementaire.

\newpage
