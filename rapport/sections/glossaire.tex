\chapter*{Glossaire}
\addcontentsline{toc}{chapter}{Glossaire}

\begin{description}

    \item[API (Application Programming Interface)]
          Interface de programmation qui permet à différentes applications de communiquer entre elles via des requêtes standardisées.

    \item[CASB (Cloud Access Security Broker)]
          Outil de sécurité qui s'interpose entre les utilisateurs et les applications cloud pour appliquer des politiques de sécurité.

    \item[CTI (Cyber Threat Intelligence)]
          Information sur les menaces cybernétiques actuelles et émergentes qui aide les organisations à prendre des décisions de sécurité éclairées.

    \item[DoublePulsar]
          Backdoor utilisée par l'exploit EternalBlue pour maintenir l'accès persistant à un système compromis.

    \item[EDR (Endpoint Detection and Response)]
          Solution de sécurité qui surveille continuellement les endpoints pour détecter et répondre aux menaces.

    \item[EternalBlue]
          Exploit développé par la NSA qui exploite une vulnérabilité dans le protocole SMBv1 de Microsoft (CVE-2017-0144).

    \item[IOC (Indicator of Compromise)]
          Artefact ou observation sur un réseau ou un système d'exploitation qui indique une intrusion informatique.

    \item[IPS (Intrusion Prevention System)]
          Système de prévention d'intrusion qui surveille le trafic réseau et bloque automatiquement les activités malveillantes.

    \item[MITRE ATT\&CK]
          Framework de connaissance développé par MITRE Corporation qui catalogue les tactiques, techniques et procédures utilisées par les adversaires.

    \item[NIDS (Network Intrusion Detection System)]
          Système de détection d'intrusion réseau qui surveille le trafic pour identifier les activités suspectes.

    \item[OSINT (Open Source Intelligence)]
          Collecte et analyse d'informations à partir de sources publiquement disponibles à des fins de renseignement.

    \item[PCAP (Packet Capture)]
          Format de fichier utilisé pour stocker les données de paquets réseau capturés par des outils de surveillance.

    \item[Playbook]
          Document ou script qui définit une série d'étapes standardisées pour répondre à un incident de sécurité spécifique.

    \item[RBAC (Role-Based Access Control)]
          Méthode de contrôle d'accès qui restreint l'accès au système en fonction du rôle de l'utilisateur dans l'organisation.

    \item[REST (Representational State Transfer)]
          Style architectural pour les services web qui utilise les méthodes HTTP standard pour les opérations CRUD.

    \item[SIEM (Security Information and Event Management)]
          Technologie qui collecte, agrège et analyse les données de sécurité en temps réel pour détecter les menaces.

    \item[SMB (Server Message Block)]
          Protocole de communication réseau utilisé pour partager des fichiers, imprimantes et ports série entre les nœuds d'un réseau.

    \item[SNORT]
          Système de détection d'intrusion réseau open source capable d'effectuer l'analyse du trafic en temps réel.

    \item[SOC (Security Operations Center)]
          Centre opérationnel centralisé qui supervise et améliore la posture de sécurité d'une organisation.

    \item[SOAR (Security Orchestration, Automation and Response)]
          Plateforme qui combine trois capacités logicielles principales : orchestration et automatisation des tâches de sécurité, et plateforme de réponse aux incidents.

    \item[STIX (Structured Threat Information eXpression)]
          Langage standardisé pour la représentation d'informations sur les menaces cybernétiques.

    \item[TAXII (Trusted Automated eXchange of Intelligence Information)]
          Spécification pour l'échange automatisé d'informations sur les menaces cybernétiques.

    \item[TTP (Tactics, Techniques, and Procedures)]
          Modèles de comportement d'un acteur malveillant, décrivant comment il mène ses opérations.

    \item[UEBA (User and Entity Behavior Analytics)]
          Processus de cybersécurité qui utilise l'analyse de données pour détecter les anomalies de comportement.

    \item[WAF (Web Application Firewall)]
          Pare-feu applicatif qui protège les applications web en filtrant, surveillant et bloquant le trafic HTTP malveillant.

    \item[XDR (Extended Detection and Response)]
          Approche de sécurité qui intègre plusieurs produits de sécurité dans un système de détection et de réponse unifié.

    \item[YARA]
          Outil permettant d'identifier et de classifier des échantillons de malware basé sur des descriptions textuelles.

    \item[Zero-Day]
          Vulnérabilité de sécurité informatique qui est exploitée par des attaquants avant qu'un correctif soit disponible.

\end{description}
