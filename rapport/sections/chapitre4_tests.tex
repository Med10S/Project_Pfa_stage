\chapter{Tests et Validation}

\section{Scenarios de Tests de Securite}

\subsection{Methodologie de Test}

\subsubsection{Approche Red Team / Blue Team}

Notre strategie de validation s'appuie sur une methodologie Red Team / Blue Team adaptee a l'environnement hospitalier, ou les contraintes de continuite de service imposent des tests non destructifs.

\paragraph{Equipe Red Team (Offensive)}
\begin{itemize}
    \item \textbf{Objectif} : Simuler des attaques realistes contre l'infrastructure hospitaliere
    \item \textbf{Contraintes} : Tests non intrusifs, environnement de laboratoire isole
    \item \textbf{Outils} : Kali Linux, Metasploit, Custom payloads
    \item \textbf{Scenarios} : EternalBlue, XSS, Sites malveillants, Brute force
\end{itemize}

\paragraph{Equipe Blue Team (Defensive)}
\begin{itemize}
    \item \textbf{Objectif} : Detecter, analyser et repondre aux attaques simulees
    \item \textbf{Outils} : Wazuh SIEM, TheHive SOAR, Cortex, MISP
    \item \textbf{Metriques} : Temps de detection, precision, taux de faux positifs
    \item \textbf{Reponse} : Workflows automatises, escalation, containment
\end{itemize}

\subsubsection{Environnement de Test Controle}

\begin{table}[H]
    \centering
    \caption{Infrastructure de test pour validation SIEM/SOAR}
    \begin{tabular}{|l|l|c|l|}
        \hline
        \textbf{Composant} & \textbf{IP}   & \textbf{OS}  & \textbf{Role}            \\
        \hline
        Attacker Machine   & 192.168.183.2 & Kali Linux   & Red Team Platform        \\
        \hline
        SIH Server         & 192.168.15.10 & Windows 2019 & Target - Hospital IS     \\
        \hline
        PACS Server        & 192.168.15.20 & Windows 2016 & Target - Medical Imaging \\
        \hline
        User Workstation   & 192.168.15.30 & Windows 10   & Target - End User        \\
        \hline
        Web Server         & 192.168.181.2 & Ubuntu 20.04 & Malicious Website        \\
        \hline
        Wazuh Manager      & 192.168.3.10  & Ubuntu 22.04 & SIEM Central             \\
        \hline
        TheHive            & 192.168.3.10  & Ubuntu 22.04 & SOAR Platform            \\
        \hline
    \end{tabular}
\end{table}

\subsection{Scenario 1 : Test EternalBlue (MS17-010)}

\subsubsection{Objectifs du Test}

\begin{itemize}
    \item Valider la detection de l'exploit EternalBlue sur systemes Windows vulnerables
    \item Tester la reactivite des workflows automatises de reponse
    \item Mesurer les performances de correlation d{'}evenements
    \item Evaluer l'efficacite de l'isolation automatique de systemes compromis
\end{itemize}

\subsubsection{Configuration du Test}

\paragraph{Serveur Cible - Machine Vulnerable}
\begin{lstlisting}[style=bashstyle,caption=Configuration machine vulnerable pour test EternalBlue]
# Machine cible : Windows 7 non patche (192.168.15.20)
# Vulnerable a MS17-010 (EternalBlue) par defaut

# Configuration minimale pour test :
# 1. Windows 7 SP1 original (non patche)
#    - SMBv1 active par defaut
#    - Vulnerable a CVE-2017-0144 (EternalBlue)
#    - Aucun patch de securite applique

# 2. Configuration reseau de base
#    - Adresse IP statique : 192.168.15.20/24
#    - Passerelle : 192.168.15.1
#    - DNS : 192.168.15.1

# 3. Services SMB actifs
#    - Port 445/tcp ouvert (Server Message Block)
#    - Port 139/tcp ouvert (NetBIOS Session Service)
#    - Partages administratifs actifs (C$, ADMIN$)

# 4. Aucune configuration logging specifique
#    - Machine vierge sans agent de monitoring
#    - Pas d'installation Wazuh (detection assuree par Suricata)
#    - Logs Windows par defaut uniquement

# 5. Simulation environnement hospitalier
#    - Nom machine : PACS-SERVER-01
#    - Workgroup : HOSPITAL
#    - Utilisateur local : Administrator (mot de passe faible)

# Note : La detection EternalBlue est assuree par Suricata
#        qui surveille le trafic reseau SMB sur le segment
#        192.168.15.0/24 -> 192.168.183.0/24 (attaquant)
\end{lstlisting}

\paragraph{Methodologie d'Analyse et Reverse Engineering}

Pour developper une detection precise d'EternalBlue, nous avons adopte une approche methodologique en trois phases :

\subparagraph{Phase 1 : Capture Manuelle du Trafic d{'}Attaque}

Dans un premier temps, nous avons execute l'attaque EternalBlue avec \textbf{msfconsole} tout en capturant manuellement le trafic reseau via \texttt{tcpdump} et \texttt{Wireshark} :

\begin{itemize}
    \item \textbf{Exploit utilise} : \texttt{windows/smb/ms17\_010\_eternalblue}
    \item \textbf{Cible} : 192.168.15.20 (Windows 7 vulnerable)
    \item \textbf{Attaquant} : 192.168.183.2 (Kali Linux)
    \item \textbf{Capture} : Trafic SMB complet sur port 445 sauvegarde en PCAP
\end{itemize}

\subparagraph{Phase 2 : Reverse Engineering des Patterns d{'}Attaque}

L{'}analyse detaillee des captures PCAP nous a permis d{'}identifier les etapes critiques de l{'}exploitation EternalBlue :

\begin{enumerate}
    \item \textbf{Grooming des messages SMB} : Preparation de la memoire du noyau par envoi de paquets SMB specifiques pour organiser le heap
    \item \textbf{Surchargement memoire} : Saturation deliberee de la memoire disponible via allocation massive de buffers
    \item \textbf{Liberation d{'}espace memoire} : Coupure brutale de connexions pour liberer des zones memoire ciblees
    \item \textbf{Buffer overflow dans SRVNET\_BUFFER} : Exploitation de la vulnerabilite pour modifier le pointeur vers la fonction de traitement SMB
    \item \textbf{Execution de code malveillant} : Detournement du flux d{'}execution vers le shellcode injecte
\end{enumerate}

\subparagraph{Phase 3 : Creation des Regles Suricata Personnalisees}

Sur la base de cette analyse, nous avons developpe des regles Suricata multi-phases capables de detecter chaque etape de l{'}attaque. Ces regles analysent les patterns binaires specifiques observes dans les captures et utilisent des flowbits pour correler les differentes phases d{'}exploitation.

\subsubsection{Analyse des Patterns d{'}Attaque EternalBlue}

\paragraph{Patterns Identifies par Reverse Engineering}

L{'}analyse detaillee des captures PCAP a revele les signatures specifiques de chaque phase d{'}exploitation :

\begin{lstlisting}[style=bashstyle,caption=Patterns critiques identifies dans l'attaque EternalBlue]
# PHASE 1: SMB Grooming - Preparation du heap memoire
# Pattern observe: Messages SMB3 avec sequences specifiques
Offset 0: |00 00 10 35 ff 53 4d 42 33|  # Header SMB3 avec taille 0x1035
Offset 9: |41 41 41 41| (repete)          # Pattern AAAA pour grooming

# PHASE 2: Memory Saturation - Surchargement memoire
# Pattern observe: Paquets SMB surdimensionnes (>4000 bytes)
# Contient des sequences repetitives pour saturer les buffers

# PHASE 3: Connection Release - Liberation d{'}espace memoire
# Pattern observe: FIN/RST immediats apres grooming
# Permet de liberer des zones memoire specifiques du heap

# PHASE 4: SRVNET_BUFFER Overflow - Exploitation critique
# Pattern observe: Buffer overflow visant le pointeur de fonction
Offset variable: |fe 53 4d 42|           # SMB3 signature specifique
Suivi de: Sequences calculees pour overflow du pointeur SRVNET

# PHASE 5: Code Execution - Detournement du flux
# Pattern observe: Shellcode execution via pointeur corrompu
# Detection: Reponses SMB anormales indiquant prise de controle
\end{lstlisting}

\paragraph{Regles Suricata Developpees}

Base sur cette analyse, nous avons cree des regles de detection multi-phases :

\begin{lstlisting}[style=bashstyle,caption=Extrait des regles Suricata personnalisees pour EternalBlue]
# PHASE 1: Detection du SMB Grooming
alert tcp any any -> any 445 (
    msg:"ETERNALBLUE PHASE 1 - SMB3 Grooming Pattern Detected";
    content:"|00 00 10 35 ff 53 4d 42 33|"; offset:0; depth:9;
    content:"|41 41 41 41|"; distance:0; within:100;
    flowbits:set,eternalblue.grooming.detected;
    sid:9000001; rev:1;
)

# PHASE 2: Detection du Memory Saturation
alert tcp any any -> any 445 (
    msg:"ETERNALBLUE PHASE 2 - Memory Saturation Attack";
    dsize:>4000;
    content:"|ff|SMB"; offset:4; depth:4;
    flowbits:isset,eternalblue.grooming.detected;
    flowbits:set,eternalblue.saturation.detected;
    sid:9000002; rev:1;
)

# PHASE 3: Detection du SRVNET_BUFFER Overflow
alert tcp any any -> any 445 (
    msg:"ETERNALBLUE PHASE 3 - SRVNET_BUFFER Overflow Attempt";
    content:"|fe 53 4d 42|"; offset:4; depth:4;
    byte_test:2,>,1500,2;
    flowbits:isset,eternalblue.saturation.detected;
    flowbits:set,eternalblue.overflow.detected;
    sid:9000003; rev:1; priority:1;
)

# CORRELATION: Chaine d{'}attaque complete
alert tcp any any -> any any (
    msg:"ETERNALBLUE CRITICAL - Complete Attack Chain Detected";
    flowbits:isset,eternalblue.grooming.detected;
    flowbits:isset,eternalblue.overflow.detected;
    threshold:type limit, track by_src, seconds 600, count 1;
    sid:9000020; rev:1; priority:1;
)
\end{lstlisting}

\subsubsection{Validation des Regles de Detection}

\paragraph{Tests de Performance des Regles Suricata}

Apres implementation des regles personnalisees, nous avons valide leur efficacite :

\begin{table}[H]
    \centering
    \caption{Performance des regles EternalBlue personnalisees}
    \begin{tabular}{|l|c|c|c|}
        \hline
        \textbf{Phase Detectee}     & \textbf{Temps Detection} & \textbf{Precision} & \textbf{Faux Positifs} \\
        \hline
        SMB Grooming (Phase 1)      & 0.8s                     & 100\%              & 0                      \\
        \hline
        Memory Saturation (Phase 2) & 1.2s                     & 100\%              & 1                      \\
        \hline
        SRVNET Overflow (Phase 3)   & 1.5s                     & 100\%              & 0                      \\
        \hline
        Correlation Complete        & 2.1s                     & 100\%              & 0                      \\
        \hline
    \end{tabular}
\end{table}

\paragraph{Chronologie de Detection Optimisee}

\begin{table}[H]
    \centering
    \caption{Timeline de detection EternalBlue avec regles personnalisees}
    \begin{tabular}{|l|c|l|l|}
        \hline
        \textbf{Timestamp} & \textbf{Delai} & \textbf{Evenement}      & \textbf{Source}              \\
        \hline
        19:04:34.120       & T+0s           & SMB Grooming detecte    & Suricata Rule 9000001        \\
        \hline
        19:04:34.920       & T+0.8s         & Memory Saturation       & Suricata Rule 9000002        \\
        \hline
        19:04:35.620       & T+1.5s         & SRVNET Overflow         & Suricata Rule 9000003        \\
        \hline
        19:04:36.220       & T+2.1s         & Correlation EternalBlue & Suricata Rule 9000020        \\
        \hline
        19:04:36.450       & T+2.3s         & Extraction PCAP         & Script intelligent-extractor \\
        \hline
        19:04:36.780       & T+2.7s         & TheHive alert created   & n8n Webhook                  \\
        \hline
        19:04:37.100       & T+3.0s         & IP blocking triggered   & OPNsense API                 \\
        \hline
        19:04:37.890       & T+3.8s         & Medical staff notified  & SMTP Gateway                 \\
        \hline
    \end{tabular}
\end{table}

\paragraph{Avantages de l'Approche Reverse Engineering}

Cette methodologie nous a permis d'obtenir :

\begin{itemize}
    \item \textbf{Detection precoce} : Identification des patterns des les premieres phases (grooming)
    \item \textbf{Precision elevee} : 0 faux positifs sur les regles critiques
    \item \textbf{Correlation fiable} : Suivi complet de la chaine d{'}attaque via flowbits
    \item \textbf{Extraction contextualisee} : PCAP captures avec metadonnees d{'}attaque
    \item \textbf{Reponse adaptee} : Escalation basee sur la severite reelle de la phase detectee
\end{itemize}


\subsection{Scenario 2 : Tests d{'}Attaques XSS}


\subsection{Objectifs et Methodologie}

Pour l'étude des attaques XSS, nous avons utilisé l'application web \textbf{DVWA (Damn Vulnerable Web Application)}, une plateforme open source conçue pour tester et apprendre les vulnérabilités courantes des applications web, dont les failles XSS (Cross-Site Scripting).

\paragraph{Présentation de DVWA}
DVWA propose plusieurs modules de vulnérabilités, dont XSS (reflected, stored, DOM-based), SQL injection, CSRF, etc. L'application permet de choisir différents niveaux de difficulté et d'observer le comportement d'une application web face à des attaques réelles.

\paragraph{Configuration de Test}
\begin{itemize}
    \item \textbf{Application cible} : DVWA (Damn Vulnerable Web Application)
    \item \textbf{Déploiement} : Conteneur Docker (voir \texttt{docker-compose.yml})
    \item \textbf{Protection} : ModSecurity (avec OWASP CRS) en reverse proxy devant DVWA
    \item \textbf{Outils d{'}attaque} : Scripts Python automatisés, payloads XSS classiques et avancés
\end{itemize}

\paragraph{Configuration ModSecurity}
ModSecurity a été configuré en mode \texttt{blocking} avec les règles OWASP CRS pour détecter et bloquer les attaques XSS. Les logs sont centralisés et analysés automatiquement (voir scripts \texttt{monitor-xss.sh}, \texttt{xss-analyzer.py}).

\paragraph{Types de XSS Testés}
\begin{enumerate}
    \item \textbf{XSS Reflected} : Injection via paramètres GET/POST sur DVWA
    \item \textbf{XSS Stockée} : Injection dans les champs persistants (ex : commentaires)
    \item \textbf{XSS DOM-based} : Exploitation via manipulation du DOM côté client
\end{enumerate}

\subsubsection{Resultats de Detection}

\begin{table}[H]
    \centering
    \caption{Performance de detection XSS avec ModSecurity}
    \begin{tabular}{|l|c|c|c|}
        \hline
        \textbf{Type XSS} & \textbf{Tests} & \textbf{Detectes} & \textbf{Bloques} \\
        \hline
        Reflective        & 7              & 7                 & 7                \\
        \hline
        Stockee           & 6              & 6                 & 6                \\
        \hline
        DOM-based         & 5              & 5                 & 5                \\
        \hline
        Contextuel        & 6              & 6                 & 6                \\
        \hline
        \textbf{Total}    & \textbf{24}    & \textbf{24}       & \textbf{24}      \\
        \hline
    \end{tabular}
\end{table}

\paragraph{Metriques de Performance}
\begin{itemize}
    \item \textbf{Taux de detection} : 100\% (24/24 payloads)
    \item \textbf{Temps de detection moyen} : 0.12 secondes
    \item \textbf{Taux de blocage} : 100\%
    \item \textbf{Faux positifs} : 5 sur trafic legitime
    \item \textbf{Impact performance} : < 2ms latence
\end{itemize}




\subsection{Scenario 3 : Test Sites Web Malveillants}

Cette section présente la méthodologie d'analyse comportementale des accès aux sites web malveillants. L'approche implémentée se base sur la surveillance DNS via Sysmon Event ID 22, avec intégration Wazuh et workflows n8n automatisés.

\subsubsection{Configuration de l'Infrastructure de Surveillance DNS}

Notre approche d'analyse des sites web malveillants repose sur la surveillance passive des requêtes DNS, qui constitue un point de contrôle stratégique dans la détection des communications vers des domaines malveillants.

\textbf{Architecture de surveillance :}
\begin{itemize}
    \item \textbf{Collecte} : Sysmon Event ID 22 (DNS Query) sur les postes de travail
    \item \textbf{Traitement} : Agent Wazuh avec règles personnalisées
    \item \textbf{Orchestration} : Workflows n8n pour l'automation
    \item \textbf{Analyse} : Cortex avec analyseurs MISP et VirusTotal
    \item \textbf{Réponse} : TheHive pour la gestion d'incidents
\end{itemize}

\subsubsection{Configuration Sysmon}

La surveillance DNS s'appuie sur une configuration Sysmon spécialisée. Le fichier \texttt{sysmonconfig.xml} est configuré pour capturer les événements DNS avec filtrage intelligent :
\newpage
\begin{lstlisting}[style=XMLStyle, caption=Configuration Sysmon pour DNS, label=lst:sysmon-dns]
<!--SYSMON EVENT ID 22 : DNS QUERY [DnsQuery]-->
<RuleGroup name="" groupRelation="or">
    <DnsQuery onmatch="exclude">
        <!--Filtrage du bruit reseau-->
        <QueryName condition="end with">.arpa.</QueryName>
        <QueryName condition="end with">.microsoft.com</QueryName>
        <QueryName condition="end with">.windows.com</QueryName>
        <!--Exclusion des CDNs legitimes-->
        <QueryName condition="end with">.akadns.net</QueryName>
        <QueryName condition="end with">.cloudfront.net</QueryName>
    </DnsQuery>
</RuleGroup>
\end{lstlisting}

Cette configuration permet de réduire le bruit tout en conservant les requêtes vers des domaines potentiellement malveillants.

\subsubsection{Configuration Agent Wazuh}

L'agent Wazuh est configuré pour collecter les événements Sysmon et appliquer des règles de détection personnalisées :

\begin{lstlisting}[style=XMLStyle, caption=Configuration Agent Wazuh, label=lst:wazuh-agent-dns]
<localfile>
    <location>Microsoft-Windows-Sysmon/Operational</location>
    <log_format>eventchannel</log_format>
</localfile>

<client_buffer>
    <disabled>no</disabled>
    <queue_size>15000</queue_size>
    <events_per_second>1000</events_per_second>
</client_buffer>

<client>
    <server>
        <address>192.168.15.3</address>
        <port>1514</port>
        <protocol>tcp</protocol>
    </server>
    <crypto_method>aes</crypto_method>
</client>
\end{lstlisting}

\subsubsection{Règles de Détection Wazuh}

Des règles personnalisées analysent les événements DNS pour détecter les accès suspects :

\begin{lstlisting}[style=XMLStyle, caption=Règle Wazuh DNS, label=lst:wazuh-rule-dns]
<rule id="61650" level="8">
    <if_sid>61649</if_sid>
    <field name="win.system.eventID">22</field>
    <description>Sysmon - Event ID 22: DNSEvent (DNS query)</description>
    <group>sysmon,sysmon_eid20_detections,windows,sysmon_event_22</group>
</rule>
\end{lstlisting}

\subsubsection{Workflow n8n d'Automatisation}

Le workflow n8n orchestre la chaîne complète de traitement des alertes DNS. La structure JSON complète du workflow comprend 20 nœuds interconnectés :

\textbf{1. Réception Webhook :}
\begin{itemize}
    \item Endpoint : \texttt{/webhook/wazuh-sysmon}
    \item Méthode : POST
    \item Traitement des alertes Wazuh en temps réel
\end{itemize}

\textbf{2. Traitement de l'Alerte :}
Le nœud \textit{Process Alert} extrait et formate les données DNS :

\begin{lstlisting}[style=JSStyle, caption=Extraction données DNS, label=lst:dns-extract]
// Extraction des informations DNS
const eventdata = win.eventdata || {};
const dnsQuery = {};

if (eventdata.queryName) {
    dnsQuery.domain = eventdata.queryName;
    description += `- **DNS Query**: \`${eventdata.queryName}\`\n`;
}

if (eventdata.queryResults) {
    dnsQuery.result = eventdata.queryResults;
    // Extraction des IPs resolues
    const ipMatches = eventdata.queryResults.match(/::ffff:(\\d+\\.\\d+\\.\\d+\\.\\d+)/g) || [];
    const resolvedIPs = ipMatches.map(ip => ip.replace('::ffff:', ''));
    dnsQuery.ips = resolvedIPs;
}
\end{lstlisting}

\textbf{3. Création d'Alerte TheHive :}
Chaque requête DNS suspecte génère automatiquement une alerte dans TheHive avec :
\begin{itemize}
    \item Sévérité basée sur le niveau Wazuh (1-3 scale)
    \item Tags automatiques (sysmon, dns-query, niveau de sévérité)
    \item TLP:AMBER par défaut
    \item Description formatée en Markdown
\end{itemize}

\textbf{4. Création d'Observables :}
Le domaine DNS devient un observable de type \texttt{domain} dans TheHive, enrichi avec :
\begin{itemize}
    \item IPs résolues en contexte
    \item Informations processus (PID, GUID, utilisateur)
    \item Résultats de requête complets
\end{itemize}

\textbf{5. Analyse Cortex :}
L'observable est automatiquement soumis aux analyseurs Cortex configurés via l'ID d'analyseur \texttt{797f393dd998e724b49b040c71d26e9f::cortex}.

\textbf{6. Traitement des Résultats :}
Le nœud \textit{Process Observable Results} évalue les rapports d'analyse selon plusieurs critères :

\begin{lstlisting}[style=JSStyle, caption=Évaluation des menaces, label=lst:threat-eval]
function analyzeReports(reports) {
    let highestThreatLevel = "info";
    let hasEvents = false;
    const findings = [];
    
    // Traitement des taxonomies MISP
    if (report?.taxonomies && Array.isArray(report.taxonomies)) {
        report.taxonomies.forEach(taxonomy => {
            if (taxonomy.value && !taxonomy.value.includes("0 events")) {
                hasEvents = true;
                updateThreatLevel(taxonomy.level);
            }
        });
    }
    
    return { threatLevel: highestThreatLevel, hasEvents, findings };
}
\end{lstlisting}

\textbf{7. Logique de Décision :}
Selon les résultats d'analyse :
\begin{itemize}
    \item \textbf{Vrai positif} : Promotion vers cas TheHive + notifications
    \item \textbf{Faux positif} : Marquage de l'alerte comme ignorée
\end{itemize}

\textbf{8. Notifications Automatisées :}
\begin{itemize}
    \item \textbf{Email HTML} : Rapport détaillé vers \texttt{soc-team@sbihi.soar.ma}
    \item \textbf{Telegram} : Notification instantanée avec boutons d'action vers \texttt{@SOC\_Team}
    \item \textbf{Liens directs} : Accès rapide aux alertes et cas dans TheHive
\end{itemize}

\subsubsection{Gestion des Défaillances}

Le workflow intègre une surveillance de l'état des analyseurs avec notifications automatiques en cas de panne :

\begin{lstlisting}[style=JSStyle, caption=Détection panne analyseur, label=lst:analyzer-fail]
// Verification du statut d'execution
if ($json.status === "Failure") {
    // Generation notifications d'alerte service
    return {
        html: alertHTML,
        markdown: alertMarkdown
    };
}
\end{lstlisting}

\subsubsection{Exemple de Test et Résultats}

\textbf{Scénario de Test :}
Un utilisateur effectue une requête DNS vers un domaine malveillant simulé \texttt{http://23.227.163.110/locker.php}, qui déclenche la chaîne de détection complète.

\textbf{Données d'Exemple Capturées :}

\begin{lstlisting}[style=JSONStyle, caption=Alerte DNS Wazuh, label=lst:dns-alert]
{
  "alert": {
    "timestamp": "2025-07-23T23:21:30.683+0000",
    "rule": {
      "level": 8,
      "description": "Sysmon - Event ID 22: DNSEvent (DNS query)",
      "id": "61650",
      "groups": ["sysmon", "sysmon_eid20_detections", "windows"]
    },
    "agent": {
      "id": "002",
      "name": "win10",
      "ip": "192.168.1.11"
    },
    "data": {
      "win": {
        "eventdata": {
          "queryName": "http://23.227.163.110/locker.php",
          "processId": "960",
          "user": "DESKTOP-75QULTD\\pc",
          "image": "C:\\Tools\\internet_detector\\internet_detector.exe"
        }
      }
    }
  }
}
\end{lstlisting}

\subsubsection{Flux de Traitement Automatisé}

\textbf{1. Détection initiale :}
\begin{itemize}
    \item Sysmon Event ID 22 capturé
    \item Règle Wazuh 61650 déclenchée (niveau 8)
    \item Webhook n8n activé automatiquement
\end{itemize}

\textbf{2. Enrichissement :}
\begin{itemize}
    \item Création observable TheHive type \texttt{domain}
    \item Soumission automatique à Cortex
    \item Analyse MISP et VirusTotal
\end{itemize}

\textbf{3. Décision automatique :}
\begin{itemize}
    \item Évaluation niveau de menace basée sur les taxonomies
    \item Promotion vers cas si score élevé
    \item Marquage faux positif si score faible
\end{itemize}

\textbf{4. Notifications :}
\begin{itemize}
    \item Email HTML détaillé vers SOC
    \item Alerte Telegram instantanée
    \item Liens directs vers interfaces TheHive
\end{itemize}

\subsubsection{Métriques de Performance}

\textbf{Temps de Réponse :}
\begin{itemize}
    \item \textbf{Détection DNS} : < 1 seconde
    \item \textbf{Alerte TheHive} : < 5 secondes
    \item \textbf{Analyse Cortex} : 30-60 secondes
    \item \textbf{Notification finale} : < 90 secondes
\end{itemize}

\textbf{Fiabilité :}
\begin{itemize}
    \item \textbf{Taux de faux positifs} : < 5\%
    \item \textbf{Couverture DNS} : 99.8\%
    \item \textbf{Disponibilité analyseurs} : 99.5\%
    \item \textbf{Délai notification} : < 2 minutes
\end{itemize}

\subsubsection{Avantages de l'Approche DNS}

\textbf{Avantages Stratégiques :}
\begin{itemize}
    \item \textbf{Point de contrôle unique} : Surveillance centralisée des résolutions DNS
    \item \textbf{Détection précoce} : Interception avant établissement connexion
    \item \textbf{Couverture étendue} : Tous les processus système inclus
    \item \textbf{Performance optimale} : Impact minimal sur les performances utilisateur
\end{itemize}

\textbf{Capacités de Détection :}
\begin{itemize}
    \item Communications C2 malware
    \item Tentatives phishing
    \item Exfiltration de données
    \item Domaines compromis
    \item Tunneling DNS
\end{itemize}

\subsubsection{Intégration Threat Intelligence}

Le système exploite plusieurs sources de renseignements :

\textbf{Sources MISP :}
\begin{itemize}
    \item Indicateurs IoC automatiquement corrélés
    \item Taxonomies standard appliquées
    \item Scoring basé sur la fréquence d'observation
\end{itemize}

\textbf{Enrichissement VirusTotal :}
\begin{itemize}
    \item Réputation domaine en temps réel
    \item Historique des détections
    \item Métadonnées DNS supplémentaires
\end{itemize}

Cette approche DNS offre une couverture de sécurité complète avec une automatisation poussée, permettant une détection rapide et une réponse coordonnée aux menaces via sites web malveillants.

Ce chapitre démontre une approche complète de tests et validation de notre solution SIEM/SOAR, avec des scénarios réalistes adaptés à l'environnement hospitalier et des métriques de performance détaillées.
